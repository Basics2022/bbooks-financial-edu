%% Generated by Sphinx.
\def\sphinxdocclass{jupyterBook}
\documentclass[letterpaper,10pt,english]{jupyterBook}
\ifdefined\pdfpxdimen
   \let\sphinxpxdimen\pdfpxdimen\else\newdimen\sphinxpxdimen
\fi \sphinxpxdimen=.75bp\relax
\ifdefined\pdfimageresolution
    \pdfimageresolution= \numexpr \dimexpr1in\relax/\sphinxpxdimen\relax
\fi
%% let collapsible pdf bookmarks panel have high depth per default
\PassOptionsToPackage{bookmarksdepth=5}{hyperref}
%% turn off hyperref patch of \index as sphinx.xdy xindy module takes care of
%% suitable \hyperpage mark-up, working around hyperref-xindy incompatibility
\PassOptionsToPackage{hyperindex=false}{hyperref}
%% memoir class requires extra handling
\makeatletter\@ifclassloaded{memoir}
{\ifdefined\memhyperindexfalse\memhyperindexfalse\fi}{}\makeatother

\PassOptionsToPackage{warn}{textcomp}

\catcode`^^^^00a0\active\protected\def^^^^00a0{\leavevmode\nobreak\ }
\usepackage{cmap}
\usepackage{fontspec}
\defaultfontfeatures[\rmfamily,\sffamily,\ttfamily]{}
\usepackage{amsmath,amssymb,amstext}
\usepackage{polyglossia}
\setmainlanguage{english}



\setmainfont{FreeSerif}[
  Extension      = .otf,
  UprightFont    = *,
  ItalicFont     = *Italic,
  BoldFont       = *Bold,
  BoldItalicFont = *BoldItalic
]
\setsansfont{FreeSans}[
  Extension      = .otf,
  UprightFont    = *,
  ItalicFont     = *Oblique,
  BoldFont       = *Bold,
  BoldItalicFont = *BoldOblique,
]
\setmonofont{FreeMono}[
  Extension      = .otf,
  UprightFont    = *,
  ItalicFont     = *Oblique,
  BoldFont       = *Bold,
  BoldItalicFont = *BoldOblique,
]



\usepackage[Bjarne]{fncychap}
\usepackage[,numfigreset=1,mathnumfig]{sphinx}

\fvset{fontsize=\small}
\usepackage{geometry}


% Include hyperref last.
\usepackage{hyperref}
% Fix anchor placement for figures with captions.
\usepackage{hypcap}% it must be loaded after hyperref.
% Set up styles of URL: it should be placed after hyperref.
\urlstyle{same}

\addto\captionsenglish{\renewcommand{\contentsname}{Introduction}}

\usepackage{sphinxmessages}



        % Start of preamble defined in sphinx-jupyterbook-latex %
         \usepackage[Latin,Greek]{ucharclasses}
        \usepackage{unicode-math}
        % fixing title of the toc
        \addto\captionsenglish{\renewcommand{\contentsname}{Contents}}
        \hypersetup{
            pdfencoding=auto,
            psdextra
        }
        % End of preamble defined in sphinx-jupyterbook-latex %
        

\title{Financial Education - basics}
\date{Jul 03, 2025}
\release{}
\author{basics}
\newcommand{\sphinxlogo}{\vbox{}}
\renewcommand{\releasename}{}
\makeindex
\begin{document}

\pagestyle{empty}
\sphinxmaketitle
\pagestyle{plain}
\sphinxtableofcontents
\pagestyle{normal}
\phantomsection\label{\detokenize{intro::doc}}


\sphinxAtStartPar
This book is meant to collect some notes about financial instruments and methods for financial education, and mainly focused asset allocation.

\sphinxAtStartPar
This material is part of the \sphinxhref{https://basics2022.github.io/bbooks}{\sphinxstylestrong{basics\sphinxhyphen{}books project}}. It is also available as a \DUrole{xref,download,myst}{.pdf document}.

\begin{sphinxadmonition}{note}{Main goal}

\sphinxAtStartPar
The ultimate goal of this material is to develop an understanding of how to manage personal savings efficiently, in line with one’s own reasonable objectives.

\sphinxAtStartPar
To achieve this, some intermediate goals include:
\begin{itemize}
\item {} 
\sphinxAtStartPar
gaining knowledge of the \sphinxstylestrong{macroeconomic environment}

\item {} 
\sphinxAtStartPar
familiarizing with some of the most common \sphinxstylestrong{investment tools} (mainly bonds and stocks);

\item {} 
\sphinxAtStartPar
getting used to some \sphinxstylestrong{common\sphinxhyphen{}sense} and \sphinxstylestrong{investing principles}: minimizing certain costs when conditions are equal, risk/reward, diversification, liquidity, and other constraints/inefficiencies

\item {} 
\sphinxAtStartPar
learning \sphinxstylestrong{what not to do}

\item {} 
\sphinxAtStartPar
and once the poor choices have been ruled out, evaluating the reasonable options for building and managing an investment portfolio, using mainly {\hyperref[\detokenize{ch/assets/etfs:fin-edu-assets-etfs}]{\sphinxcrossref{\DUrole{std,std-ref}{\sphinxstylestrong{ETF}s}}}} as a natural choice of a (usually) liquid asset providing diversification at low cost, even for small capitals.

\end{itemize}
\end{sphinxadmonition}
\begin{itemize}
\item {} 
\sphinxAtStartPar
Introduction

\begin{itemize}
\item {} 
\sphinxAtStartPar
{\hyperref[\detokenize{ch/summary::doc}]{\sphinxcrossref{Summary}}}

\item {} 
\sphinxAtStartPar
{\hyperref[\detokenize{ch/references::doc}]{\sphinxcrossref{References}}}

\end{itemize}
\end{itemize}
\begin{itemize}
\item {} 
\sphinxAtStartPar
Macroeconomic Context for Investing

\begin{itemize}
\item {} 
\sphinxAtStartPar
{\hyperref[\detokenize{ch/actors::doc}]{\sphinxcrossref{Actors}}}

\item {} 
\sphinxAtStartPar
{\hyperref[\detokenize{ch/inflation::doc}]{\sphinxcrossref{Inflation}}}

\item {} 
\sphinxAtStartPar
{\hyperref[\detokenize{code/notebooks/inflation::doc}]{\sphinxcrossref{Inflation}}}

\item {} 
\sphinxAtStartPar
{\hyperref[\detokenize{ch/policy::doc}]{\sphinxcrossref{Policy}}}

\end{itemize}
\end{itemize}
\begin{itemize}
\item {} 
\sphinxAtStartPar
Investing Principles

\begin{itemize}
\item {} 
\sphinxAtStartPar
{\hyperref[\detokenize{ch/principles/intro_nb::doc}]{\sphinxcrossref{Introduction to principles of investing}}}

\item {} 
\sphinxAtStartPar
{\hyperref[\detokenize{code/notebooks/rebalancing::doc}]{\sphinxcrossref{Rebalancing}}}

\item {} 
\sphinxAtStartPar
{\hyperref[\detokenize{code/notebooks/sequence-risk::doc}]{\sphinxcrossref{Sequence risk}}}

\end{itemize}
\end{itemize}
\begin{itemize}
\item {} 
\sphinxAtStartPar
Asset classes

\begin{itemize}
\item {} 
\sphinxAtStartPar
{\hyperref[\detokenize{ch/assets/intro::doc}]{\sphinxcrossref{Introduction to asset classes}}}

\item {} 
\sphinxAtStartPar
{\hyperref[\detokenize{ch/assets/bonds::doc}]{\sphinxcrossref{Bonds}}}

\item {} 
\sphinxAtStartPar
{\hyperref[\detokenize{ch/assets/equity::doc}]{\sphinxcrossref{Equity}}}

\item {} 
\sphinxAtStartPar
{\hyperref[\detokenize{ch/assets/etfs::doc}]{\sphinxcrossref{ETFs}}}

\end{itemize}
\end{itemize}
\begin{itemize}
\item {} 
\sphinxAtStartPar
Asset allocation

\begin{itemize}
\item {} 
\sphinxAtStartPar
{\hyperref[\detokenize{ch/investing/intro::doc}]{\sphinxcrossref{Introduction to investing}}}

\item {} 
\sphinxAtStartPar
{\hyperref[\detokenize{ch/investing/mpt::doc}]{\sphinxcrossref{Modern Portfolio Theory}}}

\item {} 
\sphinxAtStartPar
{\hyperref[\detokenize{ch/investing/capm::doc}]{\sphinxcrossref{Capital Asset Pricing Model}}}

\item {} 
\sphinxAtStartPar
{\hyperref[\detokenize{ch/investing/strategic-tactical::doc}]{\sphinxcrossref{Strategic and Tactical Asset Allocation}}}

\item {} 
\sphinxAtStartPar
{\hyperref[\detokenize{ch/investing/rebalancing::doc}]{\sphinxcrossref{Rebalancing}}}

\end{itemize}
\end{itemize}
\begin{itemize}
\item {} 
\sphinxAtStartPar
Extra

\begin{itemize}
\item {} 
\sphinxAtStartPar
{\hyperref[\detokenize{ch/extra/intro::doc}]{\sphinxcrossref{Extra and Random}}}

\item {} 
\sphinxAtStartPar
{\hyperref[\detokenize{ch/extra/euristhics::doc}]{\sphinxcrossref{Euristhics and historical correlations}}}

\end{itemize}
\end{itemize}
\begin{itemize}
\item {} 
\sphinxAtStartPar
People

\begin{itemize}
\item {} 
\sphinxAtStartPar
{\hyperref[\detokenize{ch/people/list::doc}]{\sphinxcrossref{Resources, People and Firms}}}

\item {} 
\sphinxAtStartPar
{\hyperref[\detokenize{ch/people/the_bull_guests::doc}]{\sphinxcrossref{The Bull}}}

\end{itemize}
\end{itemize}

\sphinxstepscope


\part{Introduction}

\sphinxstepscope


\chapter{Summary}
\label{\detokenize{ch/summary:summary}}\label{\detokenize{ch/summary:fin-edu-summary}}\label{\detokenize{ch/summary::doc}}\subsubsection*{Introduction}

\sphinxAtStartPar
Financial goals; money; inflation (BC and inflation target);
\subsubsection*{Asset classes}
\subsubsection*{Asset allocation}

\sphinxstepscope


\chapter{References}
\label{\detokenize{ch/references:references}}\label{\detokenize{ch/references::doc}}
\sphinxAtStartPar
Here some references to othere sources, in order to reasonably organize the contents of this book
\subsubsection*{Investment and Portfolio Management \sphinxhyphen{} RICE \sphinxhyphen{} coursera \sphinxhyphen{} A.Ozoguz, J.Foote}
\subsubsection*{Global Financial Markets}
\begin{itemize}
\item {} 
\sphinxAtStartPar
Intro and Review of Elementary Finance Tools

\item {} 
\sphinxAtStartPar
Financial system and financial assets: fixed income, equity and derivatives

\item {} 
\sphinxAtStartPar
Organization of financial markets and securities trading

\end{itemize}
\subsubsection*{Portfolio Selection and Risk Management}
\begin{itemize}
\item {} 
\sphinxAtStartPar
Intro and R/R: R/R trade\sphinxhyphen{}off

\item {} 
\sphinxAtStartPar
Ptf construction and diversification

\item {} 
\sphinxAtStartPar
Investor choices: utility functions, mean\sphinxhyphen{}variance preferences

\item {} 
\sphinxAtStartPar
Optimal capital allocation and portfolio choice: mean\sphinxhyphen{}variance optimization (Modern Portfolio Theory)

\item {} 
\sphinxAtStartPar
Equilibrium asset princing models: CAPM, return\sphinxhyphen{}beta; multi\sphinxhyphen{}factor models (e.g. Fama\sphinxhyphen{}French)

\end{itemize}
\subsubsection*{Biases and Portfolio Selection}
\begin{itemize}
\item {} 
\sphinxAtStartPar
Efficient Market Hypotesis (EMH), and anomalies

\item {} 
\sphinxAtStartPar
Biases and realistic preferences

\item {} 
\sphinxAtStartPar
Inefficient markets: equity premium, volatility puzzle (?), long\sphinxhyphen{}run reversal to the mean, value effect, momentum

\item {} 
\sphinxAtStartPar
Investor behavior

\end{itemize}
\subsubsection*{Investment Strategies and Portfolio Analysis}
\begin{itemize}
\item {} 
\sphinxAtStartPar
Performance measurement and benchmarking

\item {} 
\sphinxAtStartPar
Active vs passive investing: \(R^*\) risk\sphinxhyphen{}adjusted return measurements: Sharpe, Sortino, Treynor’ratio, Jensens’alpha,…;comparing rhe \(R^*\)

\item {} 
\sphinxAtStartPar
Performance evaluation: style analysis and performance attribution

\end{itemize}
\subsubsection*{Capstone: Build a Winning Investment Portfolio}

\sphinxAtStartPar
Using software for building ptf and assess its properties
\begin{itemize}
\item {} 
\sphinxAtStartPar
…

\end{itemize}

\sphinxstepscope


\part{Macroeconomic Context for Investing}

\sphinxstepscope


\chapter{Actors}
\label{\detokenize{ch/actors:actors}}\label{\detokenize{ch/actors:fin-edu-actors}}\label{\detokenize{ch/actors::doc}}

\section{People}
\label{\detokenize{ch/actors:people}}\label{\detokenize{ch/actors:fin-edu-actors-people}}

\section{Private companies}
\label{\detokenize{ch/actors:private-companies}}\label{\detokenize{ch/actors:fin-edu-actors-firms}}

\section{Government \sphinxhyphen{} public}
\label{\detokenize{ch/actors:government-public}}\label{\detokenize{ch/actors:fin-edu-actors-government}}

\section{Banks}
\label{\detokenize{ch/actors:banks}}\label{\detokenize{ch/actors:fin-edu-actors-banks}}

\subsection{Central banks}
\label{\detokenize{ch/actors:central-banks}}\label{\detokenize{ch/actors:fin-edu-actors-banks-cb}}

\subsection{Investment banks}
\label{\detokenize{ch/actors:investment-banks}}\label{\detokenize{ch/actors:fin-edu-actors-banks-inv}}

\section{Foreign regions}
\label{\detokenize{ch/actors:foreign-regions}}\label{\detokenize{ch/actors:fin-edu-actors-banks-foreign}}
\sphinxstepscope


\chapter{Inflation}
\label{\detokenize{ch/inflation:inflation}}\label{\detokenize{ch/inflation:fin-edu-inflation}}\label{\detokenize{ch/inflation::doc}}\begin{itemize}
\item {} 
\sphinxAtStartPar
Definition and measurements of inflation

\item {} 
\sphinxAtStartPar
Example: Italy (ISTAT). NIC, FOI, IPCA; sectors: weights, and sector IPC

\item {} 
\sphinxAtStartPar
Who controls inflation, and how? One of {\hyperref[\detokenize{ch/actors:fin-edu-actors-banks-cb}]{\sphinxcrossref{\DUrole{std,std-ref}{CB}}}} goals: inflation target; tools: monetary policy

\item {} 
\sphinxAtStartPar
Origin of inflation?
\begin{itemize}
\item {} 
\sphinxAtStartPar
short\sphinxhyphen{}, medium\sphinxhyphen{}run: cost\sphinxhyphen{}push, demand\sphinxhyphen{}pull, built\sphinxhyphen{}in (triangle model)

\item {} 
\sphinxAtStartPar
long\sphinxhyphen{}run: “monetary always and everywhere a monetary phenomenon” M.Friedman

\end{itemize}

\end{itemize}

\sphinxstepscope


\chapter{Inflation}
\label{\detokenize{code/notebooks/inflation:inflation}}\label{\detokenize{code/notebooks/inflation::doc}}

\section{Load libraries and import data}
\label{\detokenize{code/notebooks/inflation:load-libraries-and-import-data}}
\begin{sphinxuseclass}{cell}\begin{sphinxVerbatimInput}

\begin{sphinxuseclass}{cell_input}
\begin{sphinxVerbatim}[commandchars=\\\{\}]
\PYG{c+c1}{\PYGZsh{} Load libraries}
\PYG{k+kn}{import} \PYG{n+nn}{numpy} \PYG{k}{as} \PYG{n+nn}{np}
\PYG{k+kn}{import} \PYG{n+nn}{pandas} \PYG{k}{as} \PYG{n+nn}{pd}

\PYG{k+kn}{import} \PYG{n+nn}{plotly}\PYG{n+nn}{.}\PYG{n+nn}{graph\PYGZus{}objects} \PYG{k}{as} \PYG{n+nn}{go}
\PYG{k+kn}{import} \PYG{n+nn}{plotly}\PYG{n+nn}{.}\PYG{n+nn}{express} \PYG{k}{as} \PYG{n+nn}{px}
\end{sphinxVerbatim}

\end{sphinxuseclass}\end{sphinxVerbatimInput}

\end{sphinxuseclass}
\begin{sphinxuseclass}{cell}\begin{sphinxVerbatimInput}

\begin{sphinxuseclass}{cell_input}
\begin{sphinxVerbatim}[commandchars=\\\{\}]
\PYG{c+c1}{\PYGZsh{} Import data}
\PYG{c+c1}{\PYGZsh{} conditioning for being in Colab or not}
\PYG{k}{if} \PYG{l+s+s1}{\PYGZsq{}}\PYG{l+s+s1}{google.colab}\PYG{l+s+s1}{\PYGZsq{}} \PYG{o+ow}{in} \PYG{n+nb}{str}\PYG{p}{(}\PYG{n}{get\PYGZus{}ipython}\PYG{p}{(}\PYG{p}{)}\PYG{p}{)}\PYG{p}{:}
    \PYG{k+kn}{from} \PYG{n+nn}{google}\PYG{n+nn}{.}\PYG{n+nn}{colab} \PYG{k+kn}{import} \PYG{n}{drive}
    \PYG{n}{drive}\PYG{o}{.}\PYG{n}{mount}\PYG{p}{(}\PYG{l+s+s1}{\PYGZsq{}}\PYG{l+s+s1}{/content/drive}\PYG{l+s+s1}{\PYGZsq{}}\PYG{p}{)}
    \PYG{n}{folder} \PYG{o}{=} \PYG{l+s+s1}{\PYGZsq{}}\PYG{l+s+s1}{/content/drive/MyDrive/basics\PYGZhy{}books/repos/financial\PYGZhy{}edu/bbooks\PYGZhy{}financial\PYGZhy{}edu/code/data/}\PYG{l+s+s1}{\PYGZsq{}}
\PYG{k}{else}\PYG{p}{:}
    \PYG{n}{folder} \PYG{o}{=} \PYG{l+s+s1}{\PYGZsq{}}\PYG{l+s+s1}{\PYGZsq{}}

\PYG{c+c1}{\PYGZsh{}\PYGZgt{} Files}
\PYG{c+c1}{\PYGZsh{} FOI\PYGZhy{}prices: 2016\PYGZhy{}.../2025\PYGZhy{}...}
\PYG{c+c1}{\PYGZsh{} IPCA\PYGZhy{}weights: 2018/2025}
\PYG{c+c1}{\PYGZsh{} IPCA\PYGZhy{}prices:  2018\PYGZhy{}.../2025\PYGZhy{}...}
\PYG{c+c1}{\PYGZsh{} NIC\PYGZhy{}weights: 2018/2025}
\PYG{c+c1}{\PYGZsh{} NIC\PYGZhy{}prices:  2018\PYGZhy{}.../2025\PYGZhy{}...}
\PYG{n}{filen} \PYG{o}{=} \PYG{p}{\PYGZob{}}
    \PYG{l+s+s1}{\PYGZsq{}}\PYG{l+s+s1}{FOI\PYGZhy{}prices}\PYG{l+s+s1}{\PYGZsq{}}  \PYG{p}{:} \PYG{n}{folder}\PYG{o}{+}\PYG{l+s+s1}{\PYGZsq{}}\PYG{l+s+s1}{monthly\PYGZhy{}FOI.xlsx}\PYG{l+s+s1}{\PYGZsq{}} \PYG{p}{,}
    \PYG{l+s+s1}{\PYGZsq{}}\PYG{l+s+s1}{IPCA\PYGZhy{}weights}\PYG{l+s+s1}{\PYGZsq{}}\PYG{p}{:} \PYG{n}{folder}\PYG{o}{+}\PYG{l+s+s1}{\PYGZsq{}}\PYG{l+s+s1}{Classificazione Ecoicop (4 cifre) (IT1,168\PYGZus{}6\PYGZus{}DF\PYGZus{}DCSP\PYGZus{}IPCA3\PYGZus{}1,1.0).xlsx}\PYG{l+s+s1}{\PYGZsq{}}\PYG{p}{,}
    \PYG{l+s+s1}{\PYGZsq{}}\PYG{l+s+s1}{IPCA\PYGZhy{}prices}\PYG{l+s+s1}{\PYGZsq{}} \PYG{p}{:} \PYG{n}{folder}\PYG{o}{+}\PYG{l+s+s1}{\PYGZsq{}}\PYG{l+s+s1}{Classificazione Ecoicop (4 cifre) (IT1,168\PYGZus{}760\PYGZus{}DF\PYGZus{}DCSP\PYGZus{}IPCA1B2015\PYGZus{}1,1.0).xlsx}\PYG{l+s+s1}{\PYGZsq{}}\PYG{p}{,}
    \PYG{l+s+s1}{\PYGZsq{}}\PYG{l+s+s1}{NIC\PYGZhy{}weights}\PYG{l+s+s1}{\PYGZsq{}} \PYG{p}{:} \PYG{n}{folder}\PYG{o}{+}\PYG{l+s+s1}{\PYGZsq{}}\PYG{l+s+s1}{Classificazione Ecoicop (5 cifre) (IT1,167\PYGZus{}743\PYGZus{}DF\PYGZus{}DCSP\PYGZus{}NIC3B2015\PYGZus{}3,1.0).xlsx}\PYG{l+s+s1}{\PYGZsq{}}\PYG{p}{,}
    \PYG{l+s+s1}{\PYGZsq{}}\PYG{l+s+s1}{NIC\PYGZhy{}prices}\PYG{l+s+s1}{\PYGZsq{}}  \PYG{p}{:} \PYG{n}{folder}\PYG{o}{+}\PYG{l+s+s1}{\PYGZsq{}}\PYG{l+s+s1}{Classificazione Ecoicop (5 cifre) (IT1,167\PYGZus{}744\PYGZus{}DF\PYGZus{}DCSP\PYGZus{}NIC1B2015\PYGZus{}4,1.0).xlsx}\PYG{l+s+s1}{\PYGZsq{}}
\PYG{p}{\PYGZcb{}}
\end{sphinxVerbatim}

\end{sphinxuseclass}\end{sphinxVerbatimInput}

\end{sphinxuseclass}
\begin{sphinxuseclass}{cell}\begin{sphinxVerbatimInput}

\begin{sphinxuseclass}{cell_input}
\begin{sphinxVerbatim}[commandchars=\\\{\}]
\PYG{n}{df} \PYG{o}{=} \PYG{p}{\PYGZob{}}
    \PYG{l+s+s1}{\PYGZsq{}}\PYG{l+s+s1}{prices}\PYG{l+s+s1}{\PYGZsq{}} \PYG{p}{:} \PYG{n}{pd}\PYG{o}{.}\PYG{n}{read\PYGZus{}excel}\PYG{p}{(}\PYG{n}{filen}\PYG{p}{[}\PYG{l+s+s1}{\PYGZsq{}}\PYG{l+s+s1}{IPCA\PYGZhy{}prices}\PYG{l+s+s1}{\PYGZsq{}}\PYG{p}{]}\PYG{p}{,} \PYG{n}{sheet\PYGZus{}name}\PYG{o}{=}\PYG{l+s+s1}{\PYGZsq{}}\PYG{l+s+s1}{data}\PYG{l+s+s1}{\PYGZsq{}}\PYG{p}{,} \PYG{n}{decimal}\PYG{o}{=}\PYG{l+s+s1}{\PYGZsq{}}\PYG{l+s+s1}{,}\PYG{l+s+s1}{\PYGZsq{}}\PYG{p}{)}\PYG{p}{,}
    \PYG{l+s+s1}{\PYGZsq{}}\PYG{l+s+s1}{weights}\PYG{l+s+s1}{\PYGZsq{}}\PYG{p}{:} \PYG{n}{pd}\PYG{o}{.}\PYG{n}{read\PYGZus{}excel}\PYG{p}{(}\PYG{n}{filen}\PYG{p}{[}\PYG{l+s+s1}{\PYGZsq{}}\PYG{l+s+s1}{IPCA\PYGZhy{}weights}\PYG{l+s+s1}{\PYGZsq{}}\PYG{p}{]}\PYG{p}{,} \PYG{n}{sheet\PYGZus{}name}\PYG{o}{=}\PYG{l+s+s1}{\PYGZsq{}}\PYG{l+s+s1}{data}\PYG{l+s+s1}{\PYGZsq{}}\PYG{p}{,} \PYG{n}{decimal}\PYG{o}{=}\PYG{l+s+s1}{\PYGZsq{}}\PYG{l+s+s1}{,}\PYG{l+s+s1}{\PYGZsq{}}\PYG{p}{)}
\PYG{p}{\PYGZcb{}}

\PYG{k}{for} \PYG{n}{kdf}\PYG{p}{,} \PYG{n}{idf} \PYG{o+ow}{in} \PYG{n}{df}\PYG{o}{.}\PYG{n}{items}\PYG{p}{(}\PYG{p}{)}\PYG{p}{:}
    \PYG{n}{idf}\PYG{o}{.}\PYG{n}{columns} \PYG{o}{=} \PYG{n}{idf}\PYG{o}{.}\PYG{n}{columns}\PYG{o}{.}\PYG{n}{str}\PYG{o}{.}\PYG{n}{strip}\PYG{p}{(}\PYG{p}{)}  \PYG{c+c1}{\PYGZsh{} strip whitespaces}
    \PYG{n}{idf} \PYG{o}{=} \PYG{n}{idf}\PYG{o}{.}\PYG{n}{set\PYGZus{}index}\PYG{p}{(}\PYG{p}{[}\PYG{l+s+s1}{\PYGZsq{}}\PYG{l+s+s1}{Tempo}\PYG{l+s+s1}{\PYGZsq{}}\PYG{p}{]}\PYG{p}{)}
    \PYG{n}{idf} \PYG{o}{=} \PYG{n}{idf}\PYG{o}{.}\PYG{n}{transpose}\PYG{p}{(}\PYG{p}{)}
    \PYG{n}{idf} \PYG{o}{=} \PYG{n}{idf}\PYG{o}{.}\PYG{n}{rename}\PYG{p}{(}\PYG{n}{columns}\PYG{o}{=}\PYG{p}{\PYGZob{}}\PYG{l+s+s1}{\PYGZsq{}}\PYG{l+s+s1}{index}\PYG{l+s+s1}{\PYGZsq{}}\PYG{p}{:} \PYG{l+s+s1}{\PYGZsq{}}\PYG{l+s+s1}{Tempo}\PYG{l+s+s1}{\PYGZsq{}}\PYG{p}{\PYGZcb{}}\PYG{p}{)}
\end{sphinxVerbatim}

\end{sphinxuseclass}\end{sphinxVerbatimInput}
\begin{sphinxVerbatimOutput}

\begin{sphinxuseclass}{cell_output}
\begin{sphinxVerbatim}[commandchars=\\\{\}]
\PYG{g+gt}{\PYGZhy{}\PYGZhy{}\PYGZhy{}\PYGZhy{}\PYGZhy{}\PYGZhy{}\PYGZhy{}\PYGZhy{}\PYGZhy{}\PYGZhy{}\PYGZhy{}\PYGZhy{}\PYGZhy{}\PYGZhy{}\PYGZhy{}\PYGZhy{}\PYGZhy{}\PYGZhy{}\PYGZhy{}\PYGZhy{}\PYGZhy{}\PYGZhy{}\PYGZhy{}\PYGZhy{}\PYGZhy{}\PYGZhy{}\PYGZhy{}\PYGZhy{}\PYGZhy{}\PYGZhy{}\PYGZhy{}\PYGZhy{}\PYGZhy{}\PYGZhy{}\PYGZhy{}\PYGZhy{}\PYGZhy{}\PYGZhy{}\PYGZhy{}\PYGZhy{}\PYGZhy{}\PYGZhy{}\PYGZhy{}\PYGZhy{}\PYGZhy{}\PYGZhy{}\PYGZhy{}\PYGZhy{}\PYGZhy{}\PYGZhy{}\PYGZhy{}\PYGZhy{}\PYGZhy{}\PYGZhy{}\PYGZhy{}\PYGZhy{}\PYGZhy{}\PYGZhy{}\PYGZhy{}\PYGZhy{}\PYGZhy{}\PYGZhy{}\PYGZhy{}\PYGZhy{}\PYGZhy{}\PYGZhy{}\PYGZhy{}\PYGZhy{}\PYGZhy{}\PYGZhy{}\PYGZhy{}\PYGZhy{}\PYGZhy{}\PYGZhy{}\PYGZhy{}}
\PYG{n+ne}{FileNotFoundError}\PYG{g+gWhitespace}{                         }Traceback (most recent call last)
\PYG{n}{Cell} \PYG{n}{In}\PYG{p}{[}\PYG{l+m+mi}{3}\PYG{p}{]}\PYG{p}{,} \PYG{n}{line} \PYG{l+m+mi}{2}
\PYG{g+gWhitespace}{      }\PYG{l+m+mi}{1} \PYG{n}{df} \PYG{o}{=} \PYG{p}{\PYGZob{}}
\PYG{n+ne}{\PYGZhy{}\PYGZhy{}\PYGZhy{}\PYGZhy{}\PYGZgt{} }\PYG{l+m+mi}{2}     \PYG{l+s+s1}{\PYGZsq{}}\PYG{l+s+s1}{prices}\PYG{l+s+s1}{\PYGZsq{}} \PYG{p}{:} \PYG{n}{pd}\PYG{o}{.}\PYG{n}{read\PYGZus{}excel}\PYG{p}{(}\PYG{n}{filen}\PYG{p}{[}\PYG{l+s+s1}{\PYGZsq{}}\PYG{l+s+s1}{IPCA\PYGZhy{}prices}\PYG{l+s+s1}{\PYGZsq{}}\PYG{p}{]}\PYG{p}{,} \PYG{n}{sheet\PYGZus{}name}\PYG{o}{=}\PYG{l+s+s1}{\PYGZsq{}}\PYG{l+s+s1}{data}\PYG{l+s+s1}{\PYGZsq{}}\PYG{p}{,} \PYG{n}{decimal}\PYG{o}{=}\PYG{l+s+s1}{\PYGZsq{}}\PYG{l+s+s1}{,}\PYG{l+s+s1}{\PYGZsq{}}\PYG{p}{)}\PYG{p}{,}
\PYG{g+gWhitespace}{      }\PYG{l+m+mi}{3}     \PYG{l+s+s1}{\PYGZsq{}}\PYG{l+s+s1}{weights}\PYG{l+s+s1}{\PYGZsq{}}\PYG{p}{:} \PYG{n}{pd}\PYG{o}{.}\PYG{n}{read\PYGZus{}excel}\PYG{p}{(}\PYG{n}{filen}\PYG{p}{[}\PYG{l+s+s1}{\PYGZsq{}}\PYG{l+s+s1}{IPCA\PYGZhy{}weights}\PYG{l+s+s1}{\PYGZsq{}}\PYG{p}{]}\PYG{p}{,} \PYG{n}{sheet\PYGZus{}name}\PYG{o}{=}\PYG{l+s+s1}{\PYGZsq{}}\PYG{l+s+s1}{data}\PYG{l+s+s1}{\PYGZsq{}}\PYG{p}{,} \PYG{n}{decimal}\PYG{o}{=}\PYG{l+s+s1}{\PYGZsq{}}\PYG{l+s+s1}{,}\PYG{l+s+s1}{\PYGZsq{}}\PYG{p}{)}
\PYG{g+gWhitespace}{      }\PYG{l+m+mi}{4} \PYG{p}{\PYGZcb{}}
\PYG{g+gWhitespace}{      }\PYG{l+m+mi}{6} \PYG{k}{for} \PYG{n}{kdf}\PYG{p}{,} \PYG{n}{idf} \PYG{o+ow}{in} \PYG{n}{df}\PYG{o}{.}\PYG{n}{items}\PYG{p}{(}\PYG{p}{)}\PYG{p}{:}
\PYG{g+gWhitespace}{      }\PYG{l+m+mi}{7}     \PYG{n}{idf}\PYG{o}{.}\PYG{n}{columns} \PYG{o}{=} \PYG{n}{idf}\PYG{o}{.}\PYG{n}{columns}\PYG{o}{.}\PYG{n}{str}\PYG{o}{.}\PYG{n}{strip}\PYG{p}{(}\PYG{p}{)}  \PYG{c+c1}{\PYGZsh{} strip whitespaces}

\PYG{n+nn}{File \PYGZti{}/.local/lib/python3.8/site\PYGZhy{}packages/pandas/io/excel/\PYGZus{}base.py:478,} in \PYG{n+ni}{read\PYGZus{}excel}\PYG{n+nt}{(io, sheet\PYGZus{}name, header, names, index\PYGZus{}col, usecols, dtype, engine, converters, true\PYGZus{}values, false\PYGZus{}values, skiprows, nrows, na\PYGZus{}values, keep\PYGZus{}default\PYGZus{}na, na\PYGZus{}filter, verbose, parse\PYGZus{}dates, date\PYGZus{}parser, date\PYGZus{}format, thousands, decimal, comment, skipfooter, storage\PYGZus{}options, dtype\PYGZus{}backend)}
\PYG{g+gWhitespace}{    }\PYG{l+m+mi}{476} \PYG{k}{if} \PYG{o+ow}{not} \PYG{n+nb}{isinstance}\PYG{p}{(}\PYG{n}{io}\PYG{p}{,} \PYG{n}{ExcelFile}\PYG{p}{)}\PYG{p}{:}
\PYG{g+gWhitespace}{    }\PYG{l+m+mi}{477}     \PYG{n}{should\PYGZus{}close} \PYG{o}{=} \PYG{k+kc}{True}
\PYG{n+ne}{\PYGZhy{}\PYGZhy{}\PYGZgt{} }\PYG{l+m+mi}{478}     \PYG{n}{io} \PYG{o}{=} \PYG{n}{ExcelFile}\PYG{p}{(}\PYG{n}{io}\PYG{p}{,} \PYG{n}{storage\PYGZus{}options}\PYG{o}{=}\PYG{n}{storage\PYGZus{}options}\PYG{p}{,} \PYG{n}{engine}\PYG{o}{=}\PYG{n}{engine}\PYG{p}{)}
\PYG{g+gWhitespace}{    }\PYG{l+m+mi}{479} \PYG{k}{elif} \PYG{n}{engine} \PYG{o+ow}{and} \PYG{n}{engine} \PYG{o}{!=} \PYG{n}{io}\PYG{o}{.}\PYG{n}{engine}\PYG{p}{:}
\PYG{g+gWhitespace}{    }\PYG{l+m+mi}{480}     \PYG{k}{raise} \PYG{n+ne}{ValueError}\PYG{p}{(}
\PYG{g+gWhitespace}{    }\PYG{l+m+mi}{481}         \PYG{l+s+s2}{\PYGZdq{}}\PYG{l+s+s2}{Engine should not be specified when passing }\PYG{l+s+s2}{\PYGZdq{}}
\PYG{g+gWhitespace}{    }\PYG{l+m+mi}{482}         \PYG{l+s+s2}{\PYGZdq{}}\PYG{l+s+s2}{an ExcelFile \PYGZhy{} ExcelFile already has the engine set}\PYG{l+s+s2}{\PYGZdq{}}
\PYG{g+gWhitespace}{    }\PYG{l+m+mi}{483}     \PYG{p}{)}

\PYG{n+nn}{File \PYGZti{}/.local/lib/python3.8/site\PYGZhy{}packages/pandas/io/excel/\PYGZus{}base.py:1496,} in \PYG{n+ni}{ExcelFile.\PYGZus{}\PYGZus{}init\PYGZus{}\PYGZus{}}\PYG{n+nt}{(self, path\PYGZus{}or\PYGZus{}buffer, engine, storage\PYGZus{}options)}
\PYG{g+gWhitespace}{   }\PYG{l+m+mi}{1494}     \PYG{n}{ext} \PYG{o}{=} \PYG{l+s+s2}{\PYGZdq{}}\PYG{l+s+s2}{xls}\PYG{l+s+s2}{\PYGZdq{}}
\PYG{g+gWhitespace}{   }\PYG{l+m+mi}{1495} \PYG{k}{else}\PYG{p}{:}
\PYG{n+ne}{\PYGZhy{}\PYGZgt{} }\PYG{l+m+mi}{1496}     \PYG{n}{ext} \PYG{o}{=} \PYG{n}{inspect\PYGZus{}excel\PYGZus{}format}\PYG{p}{(}
\PYG{g+gWhitespace}{   }\PYG{l+m+mi}{1497}         \PYG{n}{content\PYGZus{}or\PYGZus{}path}\PYG{o}{=}\PYG{n}{path\PYGZus{}or\PYGZus{}buffer}\PYG{p}{,} \PYG{n}{storage\PYGZus{}options}\PYG{o}{=}\PYG{n}{storage\PYGZus{}options}
\PYG{g+gWhitespace}{   }\PYG{l+m+mi}{1498}     \PYG{p}{)}
\PYG{g+gWhitespace}{   }\PYG{l+m+mi}{1499}     \PYG{k}{if} \PYG{n}{ext} \PYG{o+ow}{is} \PYG{k+kc}{None}\PYG{p}{:}
\PYG{g+gWhitespace}{   }\PYG{l+m+mi}{1500}         \PYG{k}{raise} \PYG{n+ne}{ValueError}\PYG{p}{(}
\PYG{g+gWhitespace}{   }\PYG{l+m+mi}{1501}             \PYG{l+s+s2}{\PYGZdq{}}\PYG{l+s+s2}{Excel file format cannot be determined, you must specify }\PYG{l+s+s2}{\PYGZdq{}}
\PYG{g+gWhitespace}{   }\PYG{l+m+mi}{1502}             \PYG{l+s+s2}{\PYGZdq{}}\PYG{l+s+s2}{an engine manually.}\PYG{l+s+s2}{\PYGZdq{}}
\PYG{g+gWhitespace}{   }\PYG{l+m+mi}{1503}         \PYG{p}{)}

\PYG{n+nn}{File \PYGZti{}/.local/lib/python3.8/site\PYGZhy{}packages/pandas/io/excel/\PYGZus{}base.py:1371,} in \PYG{n+ni}{inspect\PYGZus{}excel\PYGZus{}format}\PYG{n+nt}{(content\PYGZus{}or\PYGZus{}path, storage\PYGZus{}options)}
\PYG{g+gWhitespace}{   }\PYG{l+m+mi}{1368} \PYG{k}{if} \PYG{n+nb}{isinstance}\PYG{p}{(}\PYG{n}{content\PYGZus{}or\PYGZus{}path}\PYG{p}{,} \PYG{n+nb}{bytes}\PYG{p}{)}\PYG{p}{:}
\PYG{g+gWhitespace}{   }\PYG{l+m+mi}{1369}     \PYG{n}{content\PYGZus{}or\PYGZus{}path} \PYG{o}{=} \PYG{n}{BytesIO}\PYG{p}{(}\PYG{n}{content\PYGZus{}or\PYGZus{}path}\PYG{p}{)}
\PYG{n+ne}{\PYGZhy{}\PYGZgt{} }\PYG{l+m+mi}{1371} \PYG{k}{with} \PYG{n}{get\PYGZus{}handle}\PYG{p}{(}
\PYG{g+gWhitespace}{   }\PYG{l+m+mi}{1372}     \PYG{n}{content\PYGZus{}or\PYGZus{}path}\PYG{p}{,} \PYG{l+s+s2}{\PYGZdq{}}\PYG{l+s+s2}{rb}\PYG{l+s+s2}{\PYGZdq{}}\PYG{p}{,} \PYG{n}{storage\PYGZus{}options}\PYG{o}{=}\PYG{n}{storage\PYGZus{}options}\PYG{p}{,} \PYG{n}{is\PYGZus{}text}\PYG{o}{=}\PYG{k+kc}{False}
\PYG{g+gWhitespace}{   }\PYG{l+m+mi}{1373} \PYG{p}{)} \PYG{k}{as} \PYG{n}{handle}\PYG{p}{:}
\PYG{g+gWhitespace}{   }\PYG{l+m+mi}{1374}     \PYG{n}{stream} \PYG{o}{=} \PYG{n}{handle}\PYG{o}{.}\PYG{n}{handle}
\PYG{g+gWhitespace}{   }\PYG{l+m+mi}{1375}     \PYG{n}{stream}\PYG{o}{.}\PYG{n}{seek}\PYG{p}{(}\PYG{l+m+mi}{0}\PYG{p}{)}

\PYG{n+nn}{File \PYGZti{}/.local/lib/python3.8/site\PYGZhy{}packages/pandas/io/common.py:868,} in \PYG{n+ni}{get\PYGZus{}handle}\PYG{n+nt}{(path\PYGZus{}or\PYGZus{}buf, mode, encoding, compression, memory\PYGZus{}map, is\PYGZus{}text, errors, storage\PYGZus{}options)}
\PYG{g+gWhitespace}{    }\PYG{l+m+mi}{859}         \PYG{n}{handle} \PYG{o}{=} \PYG{n+nb}{open}\PYG{p}{(}
\PYG{g+gWhitespace}{    }\PYG{l+m+mi}{860}             \PYG{n}{handle}\PYG{p}{,}
\PYG{g+gWhitespace}{    }\PYG{l+m+mi}{861}             \PYG{n}{ioargs}\PYG{o}{.}\PYG{n}{mode}\PYG{p}{,}
   \PYG{p}{(}\PYG{o}{.}\PYG{o}{.}\PYG{o}{.}\PYG{p}{)}
\PYG{g+gWhitespace}{    }\PYG{l+m+mi}{864}             \PYG{n}{newline}\PYG{o}{=}\PYG{l+s+s2}{\PYGZdq{}}\PYG{l+s+s2}{\PYGZdq{}}\PYG{p}{,}
\PYG{g+gWhitespace}{    }\PYG{l+m+mi}{865}         \PYG{p}{)}
\PYG{g+gWhitespace}{    }\PYG{l+m+mi}{866}     \PYG{k}{else}\PYG{p}{:}
\PYG{g+gWhitespace}{    }\PYG{l+m+mi}{867}         \PYG{c+c1}{\PYGZsh{} Binary mode}
\PYG{n+ne}{\PYGZhy{}\PYGZhy{}\PYGZgt{} }\PYG{l+m+mi}{868}         \PYG{n}{handle} \PYG{o}{=} \PYG{n+nb}{open}\PYG{p}{(}\PYG{n}{handle}\PYG{p}{,} \PYG{n}{ioargs}\PYG{o}{.}\PYG{n}{mode}\PYG{p}{)}
\PYG{g+gWhitespace}{    }\PYG{l+m+mi}{869}     \PYG{n}{handles}\PYG{o}{.}\PYG{n}{append}\PYG{p}{(}\PYG{n}{handle}\PYG{p}{)}
\PYG{g+gWhitespace}{    }\PYG{l+m+mi}{871} \PYG{c+c1}{\PYGZsh{} Convert BytesIO or file objects passed with an encoding}

\PYG{n+ne}{FileNotFoundError}: [Errno 2] No such file or directory: \PYGZsq{}Classificazione Ecoicop (4 cifre) (IT1,168\PYGZus{}760\PYGZus{}DF\PYGZus{}DCSP\PYGZus{}IPCA1B2015\PYGZus{}1,1.0).xlsx\PYGZsq{}
\end{sphinxVerbatim}

\end{sphinxuseclass}\end{sphinxVerbatimOutput}

\end{sphinxuseclass}
\begin{sphinxuseclass}{cell}\begin{sphinxVerbatimInput}

\begin{sphinxuseclass}{cell_input}
\begin{sphinxVerbatim}[commandchars=\\\{\}]
\PYG{n}{df}\PYG{p}{[}\PYG{l+s+s1}{\PYGZsq{}}\PYG{l+s+s1}{prices}\PYG{l+s+s1}{\PYGZsq{}}\PYG{p}{]}\PYG{o}{.}\PYG{n}{head}\PYG{p}{(}\PYG{p}{)}
\end{sphinxVerbatim}

\end{sphinxuseclass}\end{sphinxVerbatimInput}
\begin{sphinxVerbatimOutput}

\begin{sphinxuseclass}{cell_output}
\begin{sphinxVerbatim}[commandchars=\\\{\}]
                                               Tempo 2018\PYGZhy{}01 2018\PYGZhy{}02 2018\PYGZhy{}03  \PYGZbs{}
0                             [00] Indice generale     100.6   100.1   102.4   
1  [01] \PYGZhy{}\PYGZhy{} prodotti alimentari e bevande analcoli...   103.9     103   103.2   
2                        [011] Prodotti alimentari       104   103.2   103.4   
3                            [0111] Pane e cereali     101.6   100.6   101.1   
4                                     [0112] Carni     102.9   102.4   102.6   

  2018\PYGZhy{}04 2018\PYGZhy{}05 2018\PYGZhy{}06 2018\PYGZhy{}07 2018\PYGZhy{}08 2018\PYGZhy{}09  ... 2024\PYGZhy{}09 2024\PYGZhy{}10  \PYGZbs{}
0   102.9   103.2   103.4     102   101.8   103.5  ...     123   123.4   
1   103.6   104.3   103.9   103.1   103.1     103  ...   130.8   132.3   
2   103.7   104.5   104.2   103.2   103.2   103.2  ...   131.2   132.9   
3   101.8   101.4     102   101.7   102.1   101.3  ...   127.2   127.6   
4   102.9   102.7   102.8   102.8   102.7     103  ...   125.6   125.9   

  2024\PYGZhy{}11 2024\PYGZhy{}12 2025\PYGZhy{}01 2025\PYGZhy{}02 2025\PYGZhy{}03 2025\PYGZhy{}04 2025\PYGZhy{}05 2025\PYGZhy{}06  
0   123.3   123.4   122.4   122.5   124.4   124.9   124.8     125  
1   133.3   132.6   133.8   133.7   133.8   134.8   135.4   135.7  
2   133.8     133   134.1     134     134   134.9   135.5      ..  
3   127.8   127.8   128.5   128.2   128.4   129.2   129.3      ..  
4   126.7     127   127.9     128   128.7   129.5   130.2      ..  

[5 rows x 91 columns]
\end{sphinxVerbatim}

\end{sphinxuseclass}\end{sphinxVerbatimOutput}

\end{sphinxuseclass}
\begin{sphinxuseclass}{cell}\begin{sphinxVerbatimInput}

\begin{sphinxuseclass}{cell_input}
\begin{sphinxVerbatim}[commandchars=\\\{\}]
\PYG{n}{df}\PYG{p}{[}\PYG{l+s+s1}{\PYGZsq{}}\PYG{l+s+s1}{weights}\PYG{l+s+s1}{\PYGZsq{}}\PYG{p}{]}\PYG{o}{.}\PYG{n}{head}\PYG{p}{(}\PYG{p}{)}
\end{sphinxVerbatim}

\end{sphinxuseclass}\end{sphinxVerbatimInput}
\begin{sphinxVerbatimOutput}

\begin{sphinxuseclass}{cell_output}
\begin{sphinxVerbatim}[commandchars=\\\{\}]
                                               Tempo     2015     2016  \PYGZbs{}
0                             [00] Indice generale    1000000  1000000   
1  [01] \PYGZhy{}\PYGZhy{} prodotti alimentari e bevande analcoli...   175648   176326   
2                        [011] Prodotti alimentari     162005   162805   
3                            [0111] Pane e cereali      30036    30342   
4                                     [0112] Carni      41803    40944   

      2017     2018     2019     2020     2021     2022     2023     2024  \PYGZbs{}
0  1000000  1000000  1000000  1000000  1000000  1000000  1000000  1000000   
1   175240   175418   173257   172583   205912   194554   181443   181801   
2   161810   161903   159432   158644   189091   179008   166582   167112   
3    29853    29558    29717    29778    35767    33586    31994    31422   
4    40876    39914    39286    38162    45695    43159    39770    40253   

      2025  
0  1000000  
1   181425  
2   166336  
3    31513  
4    40080  
\end{sphinxVerbatim}

\end{sphinxuseclass}\end{sphinxVerbatimOutput}

\end{sphinxuseclass}
\begin{sphinxuseclass}{cell}\begin{sphinxVerbatimInput}

\begin{sphinxuseclass}{cell_input}
\begin{sphinxVerbatim}[commandchars=\\\{\}]
\PYG{c+c1}{\PYGZsh{} Set \PYGZsq{}Tempo\PYGZsq{} as index}
\PYG{c+c1}{\PYGZsh{} df = df.set\PYGZus{}index(\PYGZsq{}Tempo\PYGZsq{})}

\PYG{c+c1}{\PYGZsh{} Transpose the DataFrame}
\PYG{n}{df\PYGZus{}T} \PYG{o}{=} \PYG{n}{df}\PYG{o}{.}\PYG{n}{transpose}\PYG{p}{(}\PYG{p}{)}

\PYG{c+c1}{\PYGZsh{} Create a figure}
\PYG{n}{fig} \PYG{o}{=} \PYG{n}{go}\PYG{o}{.}\PYG{n}{Figure}\PYG{p}{(}\PYG{p}{)}

\PYG{c+c1}{\PYGZsh{} Add a trace for each row (now a column in df\PYGZus{}T)}
\PYG{k}{for} \PYG{n}{column} \PYG{o+ow}{in} \PYG{n}{df\PYGZus{}T}\PYG{o}{.}\PYG{n}{columns}\PYG{p}{:}
    \PYG{n}{fig}\PYG{o}{.}\PYG{n}{add\PYGZus{}trace}\PYG{p}{(}\PYG{n}{go}\PYG{o}{.}\PYG{n}{Scatter}\PYG{p}{(}
        \PYG{n}{x}\PYG{o}{=}\PYG{n}{df\PYGZus{}T}\PYG{o}{.}\PYG{n}{index}\PYG{p}{,}
        \PYG{n}{y}\PYG{o}{=}\PYG{n}{df\PYGZus{}T}\PYG{p}{[}\PYG{n}{column}\PYG{p}{]}\PYG{p}{,}
        \PYG{n}{mode}\PYG{o}{=}\PYG{l+s+s1}{\PYGZsq{}}\PYG{l+s+s1}{lines}\PYG{l+s+s1}{\PYGZsq{}}\PYG{p}{,}
        \PYG{n}{name}\PYG{o}{=}\PYG{n}{column}\PYG{p}{[}\PYG{p}{:}\PYG{l+m+mi}{50}\PYG{p}{]}
    \PYG{p}{)}\PYG{p}{)}

\PYG{c+c1}{\PYGZsh{} Customize layout}
\PYG{n}{fig}\PYG{o}{.}\PYG{n}{update\PYGZus{}layout}\PYG{p}{(}
    \PYG{n}{title}\PYG{o}{=}\PYG{l+s+s1}{\PYGZsq{}}\PYG{l+s+s1}{IPCA price indices \PYGZhy{} 2015:100}\PYG{l+s+s1}{\PYGZsq{}}\PYG{p}{,}
    \PYG{n}{xaxis\PYGZus{}title}\PYG{o}{=}\PYG{l+s+s1}{\PYGZsq{}}\PYG{l+s+s1}{Time}\PYG{l+s+s1}{\PYGZsq{}}\PYG{p}{,}
    \PYG{n}{yaxis\PYGZus{}title}\PYG{o}{=}\PYG{l+s+s1}{\PYGZsq{}}\PYG{l+s+s1}{Value}\PYG{l+s+s1}{\PYGZsq{}}\PYG{p}{,}
    \PYG{c+c1}{\PYGZsh{} xaxis\PYGZus{}tickangle=\PYGZhy{}45,}
    \PYG{c+c1}{\PYGZsh{} hovermode=\PYGZsq{}x unified\PYGZsq{},}
    \PYG{c+c1}{\PYGZsh{} template=\PYGZsq{}plotly\PYGZus{}white\PYGZsq{},}
    \PYG{n}{height}\PYG{o}{=}\PYG{l+m+mi}{600}\PYG{p}{,}
    \PYG{n}{width}\PYG{o}{=}\PYG{l+m+mi}{1000}\PYG{p}{,}
    \PYG{c+c1}{\PYGZsh{} legend=dict(orientation=\PYGZdq{}v\PYGZdq{}, x=1.02, y=1)}
\PYG{p}{)}

\PYG{n}{fig}\PYG{o}{.}\PYG{n}{show}\PYG{p}{(}\PYG{p}{)}
\end{sphinxVerbatim}

\end{sphinxuseclass}\end{sphinxVerbatimInput}
\begin{sphinxVerbatimOutput}

\begin{sphinxuseclass}{cell_output}
\end{sphinxuseclass}\end{sphinxVerbatimOutput}

\end{sphinxuseclass}
\begin{sphinxuseclass}{cell}\begin{sphinxVerbatimInput}

\begin{sphinxuseclass}{cell_input}
\begin{sphinxVerbatim}[commandchars=\\\{\}]
\PYG{c+c1}{\PYGZsh{}\PYGZgt{} Extract code and label}
\PYG{n}{ddf} \PYG{o}{=} \PYG{p}{\PYGZob{}}\PYG{p}{\PYGZcb{}}

\PYG{n}{ddf} \PYG{o}{=} \PYG{n}{pd}\PYG{o}{.}\PYG{n}{DataFrame}\PYG{p}{(}\PYG{p}{)}
\PYG{n}{ddf}\PYG{p}{[}\PYG{l+s+s1}{\PYGZsq{}}\PYG{l+s+s1}{code}\PYG{l+s+s1}{\PYGZsq{}} \PYG{p}{]} \PYG{o}{=} \PYG{n}{df}\PYG{p}{[}\PYG{l+s+s1}{\PYGZsq{}}\PYG{l+s+s1}{weights}\PYG{l+s+s1}{\PYGZsq{}}\PYG{p}{]}\PYG{p}{[}\PYG{l+s+s1}{\PYGZsq{}}\PYG{l+s+s1}{Tempo}\PYG{l+s+s1}{\PYGZsq{}}\PYG{p}{]}\PYG{o}{.}\PYG{n}{str}\PYG{o}{.}\PYG{n}{extract}\PYG{p}{(}\PYG{l+s+sa}{r}\PYG{l+s+s1}{\PYGZsq{}}\PYG{l+s+s1}{(}\PYG{l+s+s1}{\PYGZbs{}}\PYG{l+s+s1}{[}\PYG{l+s+s1}{\PYGZbs{}}\PYG{l+s+s1}{d+}\PYG{l+s+s1}{\PYGZbs{}}\PYG{l+s+s1}{])}\PYG{l+s+s1}{\PYGZsq{}}\PYG{p}{)}
\PYG{n}{ddf}\PYG{p}{[}\PYG{l+s+s1}{\PYGZsq{}}\PYG{l+s+s1}{label}\PYG{l+s+s1}{\PYGZsq{}}\PYG{p}{]} \PYG{o}{=} \PYG{n}{df}\PYG{p}{[}\PYG{l+s+s1}{\PYGZsq{}}\PYG{l+s+s1}{weights}\PYG{l+s+s1}{\PYGZsq{}}\PYG{p}{]}\PYG{p}{[}\PYG{l+s+s1}{\PYGZsq{}}\PYG{l+s+s1}{Tempo}\PYG{l+s+s1}{\PYGZsq{}}\PYG{p}{]}\PYG{o}{.}\PYG{n}{str}\PYG{o}{.}\PYG{n}{strip}\PYG{p}{(}\PYG{p}{)}

\PYG{c+c1}{\PYGZsh{}\PYGZgt{} Determine parent code}
\PYG{k}{def} \PYG{n+nf}{get\PYGZus{}parent\PYGZus{}code}\PYG{p}{(}\PYG{n}{code}\PYG{p}{)}\PYG{p}{:}
    \PYG{k}{if} \PYG{n}{code} \PYG{o+ow}{is} \PYG{k+kc}{None} \PYG{o+ow}{or} \PYG{n}{pd}\PYG{o}{.}\PYG{n}{isna}\PYG{p}{(}\PYG{n}{code}\PYG{p}{)}\PYG{p}{:}
        \PYG{k}{return} \PYG{k+kc}{None}
    \PYG{n}{code\PYGZus{}str} \PYG{o}{=} \PYG{n+nb}{str}\PYG{p}{(}\PYG{n}{code}\PYG{p}{)}\PYG{o}{.}\PYG{n}{strip}\PYG{p}{(}\PYG{l+s+s1}{\PYGZsq{}}\PYG{l+s+s1}{[]}\PYG{l+s+s1}{\PYGZsq{}}\PYG{p}{)}
    \PYG{k}{if} \PYG{n+nb}{len}\PYG{p}{(}\PYG{n}{code\PYGZus{}str}\PYG{p}{)} \PYG{o}{\PYGZlt{}}\PYG{o}{=} \PYG{l+m+mi}{2}\PYG{p}{:}
        \PYG{k}{return} \PYG{k+kc}{None}  \PYG{c+c1}{\PYGZsh{} no parent}
    \PYG{k}{elif} \PYG{n+nb}{len}\PYG{p}{(}\PYG{n}{code\PYGZus{}str}\PYG{p}{)} \PYG{o}{==} \PYG{l+m+mi}{3}\PYG{p}{:}
        \PYG{k}{return} \PYG{l+s+sa}{f}\PYG{l+s+s2}{\PYGZdq{}}\PYG{l+s+s2}{[}\PYG{l+s+si}{\PYGZob{}}\PYG{n}{code\PYGZus{}str}\PYG{p}{[}\PYG{p}{:}\PYG{l+m+mi}{2}\PYG{p}{]}\PYG{l+s+si}{\PYGZcb{}}\PYG{l+s+s2}{]}\PYG{l+s+s2}{\PYGZdq{}}
    \PYG{k}{elif} \PYG{n+nb}{len}\PYG{p}{(}\PYG{n}{code\PYGZus{}str}\PYG{p}{)} \PYG{o}{==} \PYG{l+m+mi}{4}\PYG{p}{:}
        \PYG{k}{return} \PYG{l+s+sa}{f}\PYG{l+s+s2}{\PYGZdq{}}\PYG{l+s+s2}{[}\PYG{l+s+si}{\PYGZob{}}\PYG{n}{code\PYGZus{}str}\PYG{p}{[}\PYG{p}{:}\PYG{l+m+mi}{3}\PYG{p}{]}\PYG{l+s+si}{\PYGZcb{}}\PYG{l+s+s2}{]}\PYG{l+s+s2}{\PYGZdq{}}
    \PYG{k}{else}\PYG{p}{:}
        \PYG{k}{return} \PYG{k+kc}{None}

\PYG{c+c1}{\PYGZsh{}\PYGZgt{} Determine parent code}
\PYG{k}{def} \PYG{n+nf}{get\PYGZus{}value}\PYG{p}{(}\PYG{n}{code}\PYG{p}{)}\PYG{p}{:}
    \PYG{k}{if} \PYG{n}{code} \PYG{o+ow}{is} \PYG{k+kc}{None} \PYG{o+ow}{or} \PYG{n}{pd}\PYG{o}{.}\PYG{n}{isna}\PYG{p}{(}\PYG{n}{code}\PYG{p}{)}\PYG{p}{:}
        \PYG{k}{return} \PYG{k+kc}{None}
    \PYG{n}{code\PYGZus{}str} \PYG{o}{=} \PYG{n+nb}{str}\PYG{p}{(}\PYG{n}{code}\PYG{p}{)}\PYG{o}{.}\PYG{n}{strip}\PYG{p}{(}\PYG{l+s+s1}{\PYGZsq{}}\PYG{l+s+s1}{[]}\PYG{l+s+s1}{\PYGZsq{}}\PYG{p}{)}
    \PYG{k}{if} \PYG{n+nb}{len}\PYG{p}{(}\PYG{n}{code\PYGZus{}str}\PYG{p}{)} \PYG{o}{==} \PYG{l+m+mi}{4}\PYG{p}{:}
        \PYG{k}{return} \PYG{n+nb}{int}\PYG{p}{(}\PYG{l+m+mi}{1}\PYG{p}{)}
    \PYG{k}{else}\PYG{p}{:}
        \PYG{k}{return} \PYG{n+nb}{int}\PYG{p}{(}\PYG{l+m+mi}{0}\PYG{p}{)}

\PYG{n}{ddf}\PYG{p}{[}\PYG{l+s+s1}{\PYGZsq{}}\PYG{l+s+s1}{parent}\PYG{l+s+s1}{\PYGZsq{}}\PYG{p}{]} \PYG{o}{=} \PYG{n}{ddf}\PYG{p}{[}\PYG{l+s+s1}{\PYGZsq{}}\PYG{l+s+s1}{code}\PYG{l+s+s1}{\PYGZsq{}}\PYG{p}{]}\PYG{o}{.}\PYG{n}{map}\PYG{p}{(} \PYG{n}{get\PYGZus{}parent\PYGZus{}code} \PYG{p}{)}
\PYG{c+c1}{\PYGZsh{} ddf[\PYGZsq{}value\PYGZsq{}] = ddf[\PYGZsq{}code\PYGZsq{}].map( get\PYGZus{}value )}
\PYG{n}{ddf}\PYG{p}{[}\PYG{l+s+s1}{\PYGZsq{}}\PYG{l+s+s1}{value}\PYG{l+s+s1}{\PYGZsq{}}\PYG{p}{]} \PYG{o}{=} \PYG{n}{df}\PYG{p}{[}\PYG{l+s+s1}{\PYGZsq{}}\PYG{l+s+s1}{weights}\PYG{l+s+s1}{\PYGZsq{}}\PYG{p}{]}\PYG{p}{[}\PYG{l+s+s1}{\PYGZsq{}}\PYG{l+s+s1}{2025}\PYG{l+s+s1}{\PYGZsq{}}\PYG{p}{]} \PYG{o}{/} \PYG{l+m+mf}{1e4}

\PYG{c+c1}{\PYGZsh{}\PYGZgt{} Drop [00] Indice generale}
\PYG{n}{ddf} \PYG{o}{=} \PYG{n}{ddf}\PYG{p}{[}\PYG{n}{ddf}\PYG{p}{[}\PYG{l+s+s1}{\PYGZsq{}}\PYG{l+s+s1}{code}\PYG{l+s+s1}{\PYGZsq{}}\PYG{p}{]} \PYG{o}{!=} \PYG{l+s+s1}{\PYGZsq{}}\PYG{l+s+s1}{[00]}\PYG{l+s+s1}{\PYGZsq{}}\PYG{p}{]}

\PYG{n}{ddf}\PYG{o}{.}\PYG{n}{head}\PYG{p}{(}\PYG{l+m+mi}{30}\PYG{p}{)}

\PYG{c+c1}{\PYGZsh{} \PYGZsh{} Step 4: Map parent code to label}
\PYG{c+c1}{\PYGZsh{} code\PYGZus{}label\PYGZus{}map = dict(zip(df[\PYGZsq{}code\PYGZsq{}], df[\PYGZsq{}label\PYGZsq{}]))}
\PYG{c+c1}{\PYGZsh{} df[\PYGZsq{}parent\PYGZus{}label\PYGZsq{}] = df[\PYGZsq{}parent\PYGZus{}code\PYGZsq{}].map(code\PYGZus{}label\PYGZus{}map)}

\PYG{c+c1}{\PYGZsh{} \PYGZsh{} Step 5: Full parent id}
\PYG{c+c1}{\PYGZsh{} df[\PYGZsq{}parent\PYGZsq{}] = df[\PYGZsq{}parent\PYGZus{}code\PYGZsq{}] + \PYGZsq{} \PYGZsq{} + df[\PYGZsq{}parent\PYGZus{}label\PYGZsq{}]}
\PYG{c+c1}{\PYGZsh{} df[\PYGZsq{}parent\PYGZsq{}] = df[\PYGZsq{}parent\PYGZsq{}].where(df[\PYGZsq{}parent\PYGZus{}code\PYGZsq{}].notnull())  \PYGZsh{} top\PYGZhy{}level parent stays NaN}
\end{sphinxVerbatim}

\end{sphinxuseclass}\end{sphinxVerbatimInput}
\begin{sphinxVerbatimOutput}

\begin{sphinxuseclass}{cell_output}
\begin{sphinxVerbatim}[commandchars=\\\{\}]
      code                                              label parent    value
1     [01]  [01] \PYGZhy{}\PYGZhy{} prodotti alimentari e bevande analcoliche   None  18.1425
2    [011]                          [011] Prodotti alimentari   [01]  16.6336
3   [0111]                              [0111] Pane e cereali  [011]   3.1513
4   [0112]                                       [0112] Carni  [011]   4.0080
5   [0113]                     [0113] Pesci e prodotti ittici  [011]   1.2412
6   [0114]                      [0114] Latte, formaggi e uova  [011]   2.5141
7   [0115]                                [0115] Oli e grassi  [011]   0.6514
8   [0116]                                      [0116] Frutta  [011]   1.4763
9   [0117]                                    [0117] Vegetali  [011]   2.4341
10  [0118]  [0118] Zucchero, confetture, miele, cioccolato...  [011]   0.8321
11  [0119]                  [0119] Prodotti alimentari n.a.c.  [011]   0.3251
12   [012]                          [012] Bevande analcoliche   [01]   1.5089
13  [0121]                           [0121] Caffè, tè e cacao  [012]   0.5413
14  [0122]  [0122] Acque minerali, bevande analcoliche, su...  [012]   0.9676
15    [02]               [02] \PYGZhy{}\PYGZhy{} bevande alcoliche e tabacchi   None   3.1911
16   [021]                            [021] Bevande alcoliche   [02]   1.0174
17  [0211]                                    [0211] Alcolici  [021]   0.1377
18  [0212]                                        [0212] Vini  [021]   0.5587
19  [0213]                                       [0213] Birre  [021]   0.3210
20   [022]                                     [022] Tabacchi   [02]   2.1737
21    [03]                  [03] \PYGZhy{}\PYGZhy{} abbigliamento e calzature   None   6.7911
22   [031]                                [031] Abbigliamento   [03]   5.7829
23  [0312]                                   [0312] Indumenti  [031]   5.2618
24  [0313]  [0313] Altri articoli d\PYGZsq{}abbigliamento e access...  [031]   0.2103
25  [0314]  [0314] Servizi di lavanderia, riparazione e no...  [031]   0.3108
26   [032]                                    [032] Calzature   [03]   1.0082
27  [0321]                   [0321] Scarpe ed altre calzature  [032]   0.9866
28  [0322]            [0322] Riparazione e noleggio calzature  [032]   0.0216
29    [04]  [04] \PYGZhy{}\PYGZhy{} abitazione, acqua, elettricità e combu...   None  12.6003
30   [041]                 [041] Affitti reali per abitazione   [04]   2.9507
\end{sphinxVerbatim}

\end{sphinxuseclass}\end{sphinxVerbatimOutput}

\end{sphinxuseclass}
\begin{sphinxuseclass}{cell}\begin{sphinxVerbatimInput}

\begin{sphinxuseclass}{cell_input}
\begin{sphinxVerbatim}[commandchars=\\\{\}]
\PYG{n}{fig} \PYG{o}{=} \PYG{n}{px}\PYG{o}{.}\PYG{n}{sunburst}\PYG{p}{(}
    \PYG{n}{ddf}\PYG{p}{,}
    \PYG{n}{names}\PYG{o}{=}\PYG{l+s+s1}{\PYGZsq{}}\PYG{l+s+s1}{code}\PYG{l+s+s1}{\PYGZsq{}}\PYG{p}{,}
    \PYG{n}{parents}\PYG{o}{=}\PYG{l+s+s1}{\PYGZsq{}}\PYG{l+s+s1}{parent}\PYG{l+s+s1}{\PYGZsq{}}\PYG{p}{,}
    \PYG{n}{values}\PYG{o}{=}\PYG{l+s+s1}{\PYGZsq{}}\PYG{l+s+s1}{value}\PYG{l+s+s1}{\PYGZsq{}}\PYG{p}{,}
    \PYG{n}{hover\PYGZus{}data} \PYG{o}{=} \PYG{p}{[}\PYG{l+s+s1}{\PYGZsq{}}\PYG{l+s+s1}{label}\PYG{l+s+s1}{\PYGZsq{}}\PYG{p}{]}\PYG{p}{,}
    \PYG{n}{title}\PYG{o}{=}\PYG{l+s+s1}{\PYGZsq{}}\PYG{l+s+s1}{IPCA weights \PYGZhy{} 2025}\PYG{l+s+s1}{\PYGZsq{}}\PYG{p}{,}
    \PYG{n}{branchvalues}\PYG{o}{=}\PYG{l+s+s2}{\PYGZdq{}}\PYG{l+s+s2}{total}\PYG{l+s+s2}{\PYGZdq{}}  \PYG{c+c1}{\PYGZsh{} \PYGZdq{}toatal\PYGZdq{} or \PYGZdq{}remainder\PYGZdq{}}
\PYG{p}{)}
\PYG{n}{fig}\PYG{o}{.}\PYG{n}{show}\PYG{p}{(}\PYG{p}{)}
\end{sphinxVerbatim}

\end{sphinxuseclass}\end{sphinxVerbatimInput}
\begin{sphinxVerbatimOutput}

\begin{sphinxuseclass}{cell_output}
\end{sphinxuseclass}\end{sphinxVerbatimOutput}

\end{sphinxuseclass}
\begin{sphinxuseclass}{cell}\begin{sphinxVerbatimInput}

\begin{sphinxuseclass}{cell_input}
\begin{sphinxVerbatim}[commandchars=\\\{\}]
\PYG{k+kn}{import} \PYG{n+nn}{plotly}\PYG{n+nn}{.}\PYG{n+nn}{graph\PYGZus{}objects} \PYG{k}{as} \PYG{n+nn}{go}

\PYG{n}{years} \PYG{o}{=} \PYG{p}{[}\PYG{n+nb}{str}\PYG{p}{(}\PYG{n}{y}\PYG{p}{)} \PYG{k}{for} \PYG{n}{y} \PYG{o+ow}{in} \PYG{n+nb}{range}\PYG{p}{(}\PYG{l+m+mi}{2018}\PYG{p}{,} \PYG{l+m+mi}{2026}\PYG{p}{)}\PYG{p}{]}  \PYG{c+c1}{\PYGZsh{} list of year strings}

\PYG{k}{for} \PYG{n}{year} \PYG{o+ow}{in} \PYG{n}{years}\PYG{p}{:}
   \PYG{n}{ddf}\PYG{p}{[}\PYG{n+nb}{str}\PYG{p}{(}\PYG{n}{year}\PYG{p}{)}\PYG{p}{]} \PYG{o}{=} \PYG{n}{df}\PYG{p}{[}\PYG{l+s+s1}{\PYGZsq{}}\PYG{l+s+s1}{weights}\PYG{l+s+s1}{\PYGZsq{}}\PYG{p}{]}\PYG{p}{[}\PYG{n+nb}{str}\PYG{p}{(}\PYG{n}{year}\PYG{p}{)}\PYG{p}{]} \PYG{o}{/} \PYG{l+m+mf}{1e4}


\PYG{c+c1}{\PYGZsh{} Create the initial figure for the first year}
\PYG{n}{fig} \PYG{o}{=} \PYG{n}{go}\PYG{o}{.}\PYG{n}{Figure}\PYG{p}{(}\PYG{p}{)}

\PYG{n}{fig}\PYG{o}{.}\PYG{n}{add\PYGZus{}trace}\PYG{p}{(}\PYG{n}{go}\PYG{o}{.}\PYG{n}{Sunburst}\PYG{p}{(}
    \PYG{n}{ids}\PYG{o}{=}\PYG{n}{ddf}\PYG{p}{[}\PYG{l+s+s1}{\PYGZsq{}}\PYG{l+s+s1}{code}\PYG{l+s+s1}{\PYGZsq{}}\PYG{p}{]}\PYG{p}{,}
    \PYG{n}{labels}\PYG{o}{=}\PYG{n}{ddf}\PYG{p}{[}\PYG{l+s+s1}{\PYGZsq{}}\PYG{l+s+s1}{code}\PYG{l+s+s1}{\PYGZsq{}}\PYG{p}{]}\PYG{p}{,}
    \PYG{n}{parents}\PYG{o}{=}\PYG{n}{ddf}\PYG{p}{[}\PYG{l+s+s1}{\PYGZsq{}}\PYG{l+s+s1}{parent}\PYG{l+s+s1}{\PYGZsq{}}\PYG{p}{]}\PYG{p}{,}
    \PYG{n}{values}\PYG{o}{=}\PYG{n}{ddf}\PYG{p}{[}\PYG{l+s+s1}{\PYGZsq{}}\PYG{l+s+s1}{2025}\PYG{l+s+s1}{\PYGZsq{}}\PYG{p}{]}\PYG{p}{,}  \PYG{c+c1}{\PYGZsh{} initial year}
    \PYG{n}{hoverinfo}\PYG{o}{=}\PYG{l+s+s1}{\PYGZsq{}}\PYG{l+s+s1}{label+value+text}\PYG{l+s+s1}{\PYGZsq{}}\PYG{p}{,}
    \PYG{n}{branchvalues}\PYG{o}{=}\PYG{l+s+s2}{\PYGZdq{}}\PYG{l+s+s2}{total}\PYG{l+s+s2}{\PYGZdq{}}\PYG{p}{,}  \PYG{c+c1}{\PYGZsh{} \PYGZdq{}toatal\PYGZdq{} or \PYGZdq{}remainder\PYGZdq{}}
    \PYG{n}{sort}\PYG{o}{=}\PYG{k+kc}{False}\PYG{p}{,}
    \PYG{n}{text}\PYG{o}{=}\PYG{n}{ddf}\PYG{p}{[}\PYG{l+s+s1}{\PYGZsq{}}\PYG{l+s+s1}{label}\PYG{l+s+s1}{\PYGZsq{}}\PYG{p}{]}\PYG{p}{,}
\PYG{p}{)}\PYG{p}{)}
\PYG{n}{fig}\PYG{o}{.}\PYG{n}{update\PYGZus{}layout}\PYG{p}{(}\PYG{n}{title}\PYG{o}{=}\PYG{l+s+s2}{\PYGZdq{}}\PYG{l+s+s2}{IPCA weights \PYGZhy{} 2025}\PYG{l+s+s2}{\PYGZdq{}}\PYG{p}{)}

\PYG{c+c1}{\PYGZsh{} Create one frame per year}
\PYG{n}{frames} \PYG{o}{=} \PYG{p}{[}\PYG{p}{]}
\PYG{k}{for} \PYG{n}{year} \PYG{o+ow}{in} \PYG{n}{years}\PYG{p}{:}
    \PYG{n}{frames}\PYG{o}{.}\PYG{n}{append}\PYG{p}{(}\PYG{n}{go}\PYG{o}{.}\PYG{n}{Frame}\PYG{p}{(}
        \PYG{n}{data}\PYG{o}{=}\PYG{p}{[}\PYG{n}{go}\PYG{o}{.}\PYG{n}{Sunburst}\PYG{p}{(}
            \PYG{n}{ids}\PYG{o}{=}\PYG{n}{ddf}\PYG{p}{[}\PYG{l+s+s1}{\PYGZsq{}}\PYG{l+s+s1}{code}\PYG{l+s+s1}{\PYGZsq{}}\PYG{p}{]}\PYG{p}{,}
            \PYG{n}{labels}\PYG{o}{=}\PYG{n}{ddf}\PYG{p}{[}\PYG{l+s+s1}{\PYGZsq{}}\PYG{l+s+s1}{code}\PYG{l+s+s1}{\PYGZsq{}}\PYG{p}{]}\PYG{p}{,}
            \PYG{n}{parents}\PYG{o}{=}\PYG{n}{ddf}\PYG{p}{[}\PYG{l+s+s1}{\PYGZsq{}}\PYG{l+s+s1}{parent}\PYG{l+s+s1}{\PYGZsq{}}\PYG{p}{]}\PYG{p}{,}
            \PYG{n}{values}\PYG{o}{=}\PYG{n}{ddf}\PYG{p}{[}\PYG{n}{year}\PYG{p}{]}\PYG{p}{,}
            \PYG{n}{branchvalues}\PYG{o}{=}\PYG{l+s+s2}{\PYGZdq{}}\PYG{l+s+s2}{total}\PYG{l+s+s2}{\PYGZdq{}}\PYG{p}{,}  \PYG{c+c1}{\PYGZsh{} \PYGZdq{}toatal\PYGZdq{} or \PYGZdq{}remainder\PYGZdq{}}
            \PYG{n}{sort}\PYG{o}{=}\PYG{k+kc}{False}\PYG{p}{,}
            \PYG{n}{text}\PYG{o}{=}\PYG{n}{ddf}\PYG{p}{[}\PYG{l+s+s1}{\PYGZsq{}}\PYG{l+s+s1}{label}\PYG{l+s+s1}{\PYGZsq{}}\PYG{p}{]}
        \PYG{p}{)}\PYG{p}{]}\PYG{p}{,}
        \PYG{n}{name}\PYG{o}{=}\PYG{n}{year}\PYG{p}{,}
        \PYG{n}{layout}\PYG{o}{=}\PYG{n}{go}\PYG{o}{.}\PYG{n}{Layout}\PYG{p}{(}\PYG{n}{title\PYGZus{}text}\PYG{o}{=}\PYG{l+s+sa}{f}\PYG{l+s+s2}{\PYGZdq{}}\PYG{l+s+s2}{IPCA weights \PYGZhy{} }\PYG{l+s+si}{\PYGZob{}}\PYG{n}{year}\PYG{l+s+si}{\PYGZcb{}}\PYG{l+s+s2}{\PYGZdq{}}\PYG{p}{)}
    \PYG{p}{)}\PYG{p}{)}

\PYG{n}{fig}\PYG{o}{.}\PYG{n}{frames} \PYG{o}{=} \PYG{n}{frames}

\PYG{c+c1}{\PYGZsh{} Add slider steps for each year}
\PYG{n}{steps} \PYG{o}{=} \PYG{p}{[}\PYG{p}{]}
\PYG{k}{for} \PYG{n}{year} \PYG{o+ow}{in} \PYG{n}{years}\PYG{p}{:}
    \PYG{n}{steps}\PYG{o}{.}\PYG{n}{append}\PYG{p}{(}\PYG{n+nb}{dict}\PYG{p}{(}
        \PYG{n}{method}\PYG{o}{=}\PYG{l+s+s1}{\PYGZsq{}}\PYG{l+s+s1}{animate}\PYG{l+s+s1}{\PYGZsq{}}\PYG{p}{,}
        \PYG{n}{args}\PYG{o}{=}\PYG{p}{[}\PYG{p}{[}\PYG{n}{year}\PYG{p}{]}\PYG{p}{,}  \PYG{c+c1}{\PYGZsh{} frame name}
              \PYG{n+nb}{dict}\PYG{p}{(}\PYG{n}{mode}\PYG{o}{=}\PYG{l+s+s1}{\PYGZsq{}}\PYG{l+s+s1}{immediate}\PYG{l+s+s1}{\PYGZsq{}}\PYG{p}{,}
                   \PYG{n}{frame}\PYG{o}{=}\PYG{n+nb}{dict}\PYG{p}{(}\PYG{n}{duration}\PYG{o}{=}\PYG{l+m+mi}{500}\PYG{p}{,} \PYG{n}{redraw}\PYG{o}{=}\PYG{k+kc}{True}\PYG{p}{)}\PYG{p}{,}
                   \PYG{n}{transition}\PYG{o}{=}\PYG{n+nb}{dict}\PYG{p}{(}\PYG{n}{duration}\PYG{o}{=}\PYG{l+m+mi}{300}\PYG{p}{)}\PYG{p}{)}\PYG{p}{]}\PYG{p}{,}
        \PYG{n}{label}\PYG{o}{=}\PYG{n}{year}
    \PYG{p}{)}\PYG{p}{)}

\PYG{c+c1}{\PYGZsh{} Layout with slider}
\PYG{n}{fig}\PYG{o}{.}\PYG{n}{update\PYGZus{}layout}\PYG{p}{(}
    \PYG{n}{width}\PYG{o}{=}\PYG{l+m+mi}{800}\PYG{p}{,} \PYG{n}{height}\PYG{o}{=}\PYG{l+m+mi}{800}\PYG{p}{,}
    \PYG{n}{margin}\PYG{o}{=}\PYG{n+nb}{dict}\PYG{p}{(}\PYG{n}{t}\PYG{o}{=}\PYG{l+m+mi}{50}\PYG{p}{,} \PYG{n}{l}\PYG{o}{=}\PYG{l+m+mi}{0}\PYG{p}{,} \PYG{n}{r}\PYG{o}{=}\PYG{l+m+mi}{0}\PYG{p}{,} \PYG{n}{b}\PYG{o}{=}\PYG{l+m+mi}{0}\PYG{p}{)}\PYG{p}{,}
    \PYG{n}{sliders}\PYG{o}{=}\PYG{p}{[}\PYG{n+nb}{dict}\PYG{p}{(}
        \PYG{n}{active}\PYG{o}{=}\PYG{n}{years}\PYG{o}{.}\PYG{n}{index}\PYG{p}{(}\PYG{l+s+s1}{\PYGZsq{}}\PYG{l+s+s1}{2025}\PYG{l+s+s1}{\PYGZsq{}}\PYG{p}{)}\PYG{p}{,}
        \PYG{n}{currentvalue}\PYG{o}{=}\PYG{p}{\PYGZob{}}\PYG{l+s+s2}{\PYGZdq{}}\PYG{l+s+s2}{prefix}\PYG{l+s+s2}{\PYGZdq{}}\PYG{p}{:} \PYG{l+s+s2}{\PYGZdq{}}\PYG{l+s+s2}{Year: }\PYG{l+s+s2}{\PYGZdq{}}\PYG{p}{\PYGZcb{}}\PYG{p}{,}
        \PYG{n}{pad}\PYG{o}{=}\PYG{p}{\PYGZob{}}\PYG{l+s+s2}{\PYGZdq{}}\PYG{l+s+s2}{t}\PYG{l+s+s2}{\PYGZdq{}}\PYG{p}{:} \PYG{l+m+mi}{50}\PYG{p}{\PYGZcb{}}\PYG{p}{,}
        \PYG{n}{steps}\PYG{o}{=}\PYG{n}{steps}
    \PYG{p}{)}\PYG{p}{]}\PYG{p}{,}
\PYG{p}{)}

\PYG{n}{fig}\PYG{o}{.}\PYG{n}{show}\PYG{p}{(}\PYG{p}{)}
\end{sphinxVerbatim}

\end{sphinxuseclass}\end{sphinxVerbatimInput}
\begin{sphinxVerbatimOutput}

\begin{sphinxuseclass}{cell_output}
\end{sphinxuseclass}\end{sphinxVerbatimOutput}

\end{sphinxuseclass}
\sphinxstepscope


\chapter{Policy}
\label{\detokenize{ch/policy:policy}}\label{\detokenize{ch/policy:fin-edu-policy}}\label{\detokenize{ch/policy::doc}}

\begin{savenotes}\sphinxattablestart
\centering
\begin{tabulary}{\linewidth}[t]{|T|T|T|}
\hline

\sphinxAtStartPar

&\sphinxstyletheadfamily 
\sphinxAtStartPar
Monetary Policy
&\sphinxstyletheadfamily 
\sphinxAtStartPar
Fiscal Policy
\\
\hline
\sphinxAtStartPar
Controlled by
&
\sphinxAtStartPar
CB
&
\sphinxAtStartPar
Government
\\
\hline
\sphinxAtStartPar
Main tools
&
\sphinxAtStartPar
IR, Money supply
&
\sphinxAtStartPar
Taxes, Spending, Transfers
\\
\hline
\sphinxAtStartPar
Speed
&
\sphinxAtStartPar
Usually faster
&
\sphinxAtStartPar
Politically slower, debated
\\
\hline
\sphinxAtStartPar
Focus
&
\sphinxAtStartPar
Inflation, lliquidity, credit
&
\sphinxAtStartPar
Employment, Income distribution
\\
\hline
\sphinxAtStartPar
Independence
&
\sphinxAtStartPar

&
\sphinxAtStartPar

\\
\hline
\end{tabulary}
\par
\sphinxattableend\end{savenotes}


\section{Monetary policy}
\label{\detokenize{ch/policy:monetary-policy}}\label{\detokenize{ch/policy:fin-edu-policy-monetary}}

\section{Fiscal policy}
\label{\detokenize{ch/policy:fiscal-policy}}\label{\detokenize{ch/policy:fin-edu-policy-fiscal}}
\sphinxstepscope


\part{Investing Principles}

\sphinxstepscope


\chapter{Introduction to principles of investing}
\label{\detokenize{ch/principles/intro_nb:introduction-to-principles-of-investing}}\label{\detokenize{ch/principles/intro_nb:fin-edu-principles-intro-nb}}\label{\detokenize{ch/principles/intro_nb::doc}}
\sphinxAtStartPar
Investing is a core part of personal financial management—it’s how individuals navigate uncertainty to meet their financial goals under real\sphinxhyphen{}world constraints. The most basic objective is to preserve the real value of wealth, protecting it against {\hyperref[\detokenize{ch/inflation:fin-edu-inflation}]{\sphinxcrossref{\DUrole{std,std-ref}{inflation}}}}; more ambitious goals include growing capital to fund retirement, education, or other life plans.

\sphinxAtStartPar
Sound investing requires understanding {\hyperref[\detokenize{ch/principles/intro_nb:fin-edu-principles-return}]{\sphinxcrossref{\DUrole{std,std-ref}{return}}}} and {\hyperref[\detokenize{ch/principles/intro_nb:fin-edu-principles-risk}]{\sphinxcrossref{\DUrole{std,std-ref}{risk}}}} of available assets, and the fundamental {\hyperref[\detokenize{ch/principles/intro_nb:fin-edu-principles-rr}]{\sphinxcrossref{\DUrole{std,std-ref}{R/R trade off}}}}. It also demands attention to \sphinxstylestrong{constraints} such as \sphinxstyleemphasis{liquidity} needs, \sphinxstyleemphasis{time horizon}, \sphinxstyleemphasis{acceptable volatility}, and \sphinxstyleemphasis{risk tolerance}. One of the main principle is {\hyperref[\detokenize{ch/principles/intro_nb:fin-edu-principles-diversification}]{\sphinxcrossref{\DUrole{std,std-ref}{diversification}}}} \sphinxhyphen{} which can reduce risk and, in some cases, enhance returns.

\sphinxAtStartPar
This section introduces the core concepts needed to build a robust investment strategy: how {\hyperref[\detokenize{ch/principles/intro_nb:fin-edu-principles-time-compunding}]{\sphinxcrossref{\DUrole{std,std-ref}{compound returns}}}} shape long\sphinxhyphen{}term growth, how {\hyperref[\detokenize{ch/principles/intro_nb:fin-edu-principles-time-volatility-drag}]{\sphinxcrossref{\DUrole{std,std-ref}{volatility drag}}}} reduces expected performance, and how a clear, principle\sphinxhyphen{}based approaches \sphinxhyphen{} like {\hyperref[\detokenize{ch/principles/intro_nb:fin-edu-principles-rebalancing}]{\sphinxcrossref{\DUrole{std,std-ref}{rebalancing}}}} \sphinxhyphen{} may improve performance under uncertainties.

\sphinxAtStartPar
Given its set of constraints, an informed and intelligent agent, see {\hyperref[\detokenize{ch/principles/intro_nb:fin-edu-principles-asset-allocation}]{\sphinxcrossref{\DUrole{std,std-ref}{Portfolio construction}}}} would take actions that try to maximise return for a given accepted risk, or minimize risk for a given desired return: this behavior can be summarized in choosing actions on a \sphinxstyleemphasis{Pareto front}, i.e. within the set of all Pareto efficient solutions.
\subsubsection*{Sections}


\begin{savenotes}\sphinxattablestart
\centering
\begin{tabulary}{\linewidth}[t]{|T|T|}
\hline
\sphinxstyletheadfamily 
\sphinxAtStartPar
\sphinxstylestrong{Section}
&\sphinxstyletheadfamily 
\sphinxAtStartPar
\sphinxstylestrong{Key Concepts}
\\
\hline
\sphinxAtStartPar
\sphinxstylestrong{1. {\hyperref[\detokenize{ch/principles/intro_nb:fin-edu-principles-return}]{\sphinxcrossref{\DUrole{std,std-ref}{Return}}}}}
&
\sphinxAtStartPar

\\
\hline
\sphinxAtStartPar
\sphinxstylestrong{2. {\hyperref[\detokenize{ch/principles/intro_nb:fin-edu-principles-risk}]{\sphinxcrossref{\DUrole{std,std-ref}{Risk}}}}}
&
\sphinxAtStartPar

\\
\hline
\sphinxAtStartPar
\sphinxstylestrong{3. {\hyperref[\detokenize{ch/principles/intro_nb:fin-edu-principles-rr}]{\sphinxcrossref{\DUrole{std,std-ref}{Risk\sphinxhyphen{}Return Trade\sphinxhyphen{}Off}}}}}
&
\sphinxAtStartPar

\\
\hline
\sphinxAtStartPar
\sphinxstylestrong{4. {\hyperref[\detokenize{ch/principles/intro_nb:fin-edu-principles-diversification}]{\sphinxcrossref{\DUrole{std,std-ref}{Diversification}}}}}
&
\sphinxAtStartPar

\\
\hline
\sphinxAtStartPar
\sphinxstylestrong{5. {\hyperref[\detokenize{ch/principles/intro_nb:fin-edu-principles-asset-allocation}]{\sphinxcrossref{\DUrole{std,std-ref}{Portfolio Construction}}}}}
&
\sphinxAtStartPar

\\
\hline
\sphinxAtStartPar
\sphinxstylestrong{6. {\hyperref[\detokenize{ch/principles/intro_nb:fin-edu-principles-time}]{\sphinxcrossref{\DUrole{std,std-ref}{Time and Compounding}}}}}
&
\sphinxAtStartPar
Compounding and volatility drag
\\
\hline
\sphinxAtStartPar
\sphinxstylestrong{7. {\hyperref[\detokenize{ch/principles/intro_nb:fin-edu-principles-investing}]{\sphinxcrossref{\DUrole{std,std-ref}{Disciplined Investing}}}}}
&
\sphinxAtStartPar
PIC/PAC, rebalancing,…
\\
\hline
\end{tabulary}
\par
\sphinxattableend\end{savenotes}


\section{Return}
\label{\detokenize{ch/principles/intro_nb:return}}\label{\detokenize{ch/principles/intro_nb:fin-edu-principles-return}}
\sphinxAtStartPar
Return is the reward for investing. It can come from \sphinxstylestrong{capital gain} (price increase of assets bought), or \sphinxstylestrong{periodic cashflows}, like interest (from bonds), or dividends (from stocks). Some assets produce predictable return (either nominal, or real), other assets have less predictable returns. Any asset has some level of uncertainty, or {\hyperref[\detokenize{ch/principles/intro_nb:fin-edu-principles-rr}]{\sphinxcrossref{\DUrole{std,std-ref}{risk}}}}%
\begin{footnote}[1]\sphinxAtStartFootnote
Even the most safe assets could undergo some (really) \sphinxstylestrong{rare}, but usually (really) \sphinxstylestrong{catastrophic events}. Just as an example, it’s hard to imagine what could happen even to bonds issued by the most (perceived and priced) safe government or institution, in case of its participation in a war.
%
\end{footnote}.

\sphinxAtStartPar
Most returns are quoted on a \sphinxstylestrong{per\sphinxhyphen{}period} basis \sphinxhyphen{} usually annually \sphinxhyphen{} and expressed as the percentage of the reward over the initial amount of the investment.
\label{ch/principles/intro_nb:example-0}
\begin{sphinxadmonition}{note}{Example 7.1.1 (1\sphinxhyphen{}period returns)}



\sphinxAtStartPar
An amount of \(1000\)€ in saving account returning \(2\%\) per year, returns \(0.02 \cdot 1000\text{€} = 20\)€ at the end of the year, so that the amount in the saving account becomes \(1000\text{€} \cdot 1.02 = 1020\text{€}\). Usually some {\hyperref[\detokenize{ch/principles/intro_nb:fin-edu-principles-return-costs}]{\sphinxcrossref{\DUrole{std,std-ref}{costs}}}} must be also considered.
…
\end{sphinxadmonition}
\label{ch/principles/intro_nb:example-1}
\begin{sphinxadmonition}{note}{Example 7.1.2 (1\sphinxhyphen{}period return of equity investment)}



\sphinxAtStartPar
…given costs, dividends, taxes, buying and selling prices, evaluate return…
\end{sphinxadmonition}

\sphinxAtStartPar
For a many\sphinxhyphen{}year investment, single\sphinxhyphen{}period returns {\hyperref[\detokenize{ch/principles/intro_nb:fin-edu-principles-time-compunding}]{\sphinxcrossref{\DUrole{std,std-ref}{\sphinxstylestrong{compound}}}}} over time.


\subsection{Costs}
\label{\detokenize{ch/principles/intro_nb:costs}}\label{\detokenize{ch/principles/intro_nb:fin-edu-principles-return-costs}}
\sphinxAtStartPar
While return are uncertain, at least to a certain level, usually costs \sphinxhyphen{} fees, expenses, taxes \sphinxhyphen{} or part of them, are certain. With equal other conditions, the intelligent investor should reduce costs (known), as higher costs reduce returns w/o changing the level of risk.


\section{Risk}
\label{\detokenize{ch/principles/intro_nb:risk}}\label{\detokenize{ch/principles/intro_nb:fin-edu-principles-risk}}
\sphinxAtStartPar
Risk measures uncertainty and its effects, combining probability of events and consequences of specific events. \sphinxstyleemphasis{All the assets have some systematic and some specific risks}
.



\sphinxAtStartPar
Key measures (\sphinxstyleemphasis{should give info about magnitude, frequency/probability, and duration}) include:
\begin{itemize}
\item {} 
\sphinxAtStartPar
standard deviation or \sphinxstylestrong{volatility}: how much returns may deviate from their expected value),

\item {} 
\sphinxAtStartPar
max loss (usually 100\% can’t be neglected for catastrophic although rare events), value at risk (VaR, max loss with a given probability), drawdown (maximum peak\sphinxhyphen{}to\sphinxhyphen{}trough loss)

\item {} 
\sphinxAtStartPar
time\sphinxhyphen{}to\sphinxhyphen{}recover (time to recover drawdowns, in a temporal perspective)

\end{itemize}

\sphinxAtStartPar
Usually, risk metrics measure uncertainty, without discerning from positive and negative events: these metrics perceive a higher\sphinxhyphen{}than\sphinxhyphen{}expected return as a risk as well. Some metrics instead, see \sphinxstyleemphasis{Sortino ratio} in {\hyperref[\detokenize{ch/principles/intro_nb:fin-edu-principles-rr}]{\sphinxcrossref{\DUrole{std,std-ref}{risk\sphinxhyphen{}return}}}} section, aims at quantifying only negative events as risk.
\label{ch/principles/intro_nb:example-2}
\begin{sphinxadmonition}{note}{Example 7.2.1 (Value at Risk)}



\sphinxAtStartPar
A 1\sphinxhyphen{}year 5\% VaR of \(1000\text{€}\) of an investment it means that there’s 5\% probability of losing at least \(1000\text{€}\) in a year with that investment.
\end{sphinxadmonition}


\section{Risk\sphinxhyphen{}Return Trade Off}
\label{\detokenize{ch/principles/intro_nb:risk-return-trade-off}}\label{\detokenize{ch/principles/intro_nb:fin-edu-principles-rr}}
\begin{sphinxadmonition}{note}{“There’s no free lunch”}

\sphinxAtStartPar
Higher expected returns usually come with higher risk.
\end{sphinxadmonition}

\begin{sphinxadmonition}{note}{…but high risk doesn’t imply high expected return}

\sphinxAtStartPar
Very stupid actions usually implies poor return with high risk. Just as an example, playing Russian roulette for fun implies an expected return worse than an alternative “do\sphinxhyphen{}nothing and have an ice\sphinxhyphen{}cream instead” scenario (at least, if your goal is not to kill yourself, and your return function does not positively weight this outcome) with higher uncertainty on the final status of your health.

\sphinxAtStartPar
Sometimes the same could happen if one plays doing trading with some random meme\sphinxhyphen{}stocks or shit\sphinxhyphen{}coins.
\end{sphinxadmonition}

\sphinxAtStartPar
\sphinxstylestrong{Risk\sphinxhyphen{}adjusted return} provides an indication of the expected return per unit of risk. Common metrics are:
\begin{itemize}
\item {} 
\sphinxAtStartPar
\sphinxstylestrong{Sharpe ratio}, comparing excess return and volatility compared with a “risk\sphinxhyphen{}free” asset \sphinxhyphen{} or a benchmark
\begin{equation*}
\begin{split}S := \dfrac{\mathbb{E}[R-R_b]}{\sqrt{\text{var}[R-R_b]}}\end{split}
\end{equation*}
\item {} 
\sphinxAtStartPar
\sphinxstylestrong{Sortino ratio}
\begin{equation*}
\begin{split}So := \dfrac{\mathbb{E}[R] - T}{\text{DR}} \ ,\end{split}
\end{equation*}
\sphinxAtStartPar
with \(T\) target return, and \(\text{DR}\) the downside deviation, i.e. the deviation w.r.t the target return evaluated only for returns \(r\) lower than the target return \(T\)
\begin{equation*}
\begin{split}\text{DR}^2 = \int_{r=-\infty}^{T} (T-r)^2 \, f(r) \, dr \ ,\end{split}
\end{equation*}
\sphinxAtStartPar
being \(f(r)\) the probability density function of the (continuous) random variable \(R\) representing return

\end{itemize}
\label{ch/principles/intro_nb:example-3}
\begin{sphinxadmonition}{note}{Example 7.3.1}


\end{sphinxadmonition}


\section{Diversification}
\label{\detokenize{ch/principles/intro_nb:diversification}}\label{\detokenize{ch/principles/intro_nb:fin-edu-principles-diversification}}
\sphinxAtStartPar
Diversification spreads risk across different investments so no single event can ruin your portfolio. Diversification works well with assets that are not \sphinxhyphen{} or at least they’re loosely \sphinxhyphen{} correlated: in this case, diversification could increase return per unit of risk.


\section{Portfolio Construction}
\label{\detokenize{ch/principles/intro_nb:portfolio-construction}}\label{\detokenize{ch/principles/intro_nb:fin-edu-principles-asset-allocation}}

\section{Time}
\label{\detokenize{ch/principles/intro_nb:time}}\label{\detokenize{ch/principles/intro_nb:fin-edu-principles-time}}

\subsection{Compound Return}
\label{\detokenize{ch/principles/intro_nb:compound-return}}\label{\detokenize{ch/principles/intro_nb:fin-edu-principles-time-compunding}}
\begin{sphinxuseclass}{cell}
\begin{sphinxuseclass}{tag_hide-input}\begin{sphinxVerbatimOutput}

\begin{sphinxuseclass}{cell_output}
\begin{sphinxVerbatim}[commandchars=\\\{\}]
(Text(0.5, 0, \PYGZsq{}t\PYGZsq{}), None)
\end{sphinxVerbatim}

\noindent\sphinxincludegraphics{{762bf809b74094ddaaa2537222a9e4fe8c7a634837a512931af6778e3b19d5c5}.png}

\end{sphinxuseclass}\end{sphinxVerbatimOutput}

\end{sphinxuseclass}
\end{sphinxuseclass}

\subsubsection{Volatility Drag}
\label{\detokenize{ch/principles/intro_nb:volatility-drag}}\label{\detokenize{ch/principles/intro_nb:fin-edu-principles-time-volatility-drag}}
\begin{sphinxuseclass}{cell}
\begin{sphinxuseclass}{tag_hide-input}\begin{sphinxVerbatimOutput}

\begin{sphinxuseclass}{cell_output}
\noindent\sphinxincludegraphics{{94ef836114fb7373b6ffb9316d5293b45be4c15d84e0ff97fbaabdc442ae172d}.png}

\end{sphinxuseclass}\end{sphinxVerbatimOutput}

\end{sphinxuseclass}
\end{sphinxuseclass}
\sphinxAtStartPar
\sphinxstylestrong{todo}
\begin{itemize}
\item {} 
\sphinxAtStartPar
\sphinxstyleemphasis{“Time and risk?” Listen to The Logic of Risk}

\end{itemize}


\section{Disciplined Investing}
\label{\detokenize{ch/principles/intro_nb:disciplined-investing}}\label{\detokenize{ch/principles/intro_nb:fin-edu-principles-investing}}

\subsection{Rebalancing}
\label{\detokenize{ch/principles/intro_nb:rebalancing}}\label{\detokenize{ch/principles/intro_nb:fin-edu-principles-rebalancing}}
\sphinxAtStartPar
\sphinxhref{https://colab.research.google.com/drive/1Mi3\_9T7XN7xUl9XNfsdkMfqQTqRFzG8L?authuser=1\#scrollTo=QUq8nMHq3bb5}{Colab Notebook, rebalancing.ipynb}


\subsubsection{Rebalancing premium}
\label{\detokenize{ch/principles/intro_nb:rebalancing-premium}}\label{\detokenize{ch/principles/intro_nb:fin-edu-principles-rebalancing-premium}}

\bigskip\hrule\bigskip


\sphinxstepscope


\chapter{Rebalancing}
\label{\detokenize{code/notebooks/rebalancing:rebalancing}}\label{\detokenize{code/notebooks/rebalancing::doc}}
\sphinxAtStartPar
In this Notwbook, two strategies on a 2\sphinxhyphen{}asset portfolio are discussed and compared:
\begin{itemize}
\item {} 
\sphinxAtStartPar
\sphinxstylestrong{rebalanced portfolio}, after each period

\item {} 
\sphinxAtStartPar
\sphinxstylestrong{buy\sphinxhyphen{}and\sphinxhyphen{}hold portfolio}, without rebalancing

\end{itemize}



\sphinxAtStartPar
Effects of rebalancing and conditions for \sphinxstylestrong{rebalancing premium} are discussed: sometimes the expected log\sphinxhyphen{}return of the rebalanced portfolio may exceed the expected log\sphinxhyphen{}return of each single asset.


\section{Libraries, parameters and useful functions}
\label{\detokenize{code/notebooks/rebalancing:libraries-parameters-and-useful-functions}}
\sphinxAtStartPar
Libraries are imported and useful functions to treat conics below are defined here


\subsection{Libraries}
\label{\detokenize{code/notebooks/rebalancing:libraries}}

\subsection{Parameters}
\label{\detokenize{code/notebooks/rebalancing:parameters}}

\subsection{Functions for conics}
\label{\detokenize{code/notebooks/rebalancing:functions-for-conics}}

\section{Rebalanced portfolio}
\label{\detokenize{code/notebooks/rebalancing:rebalanced-portfolio}}
\sphinxAtStartPar
Let the \sphinxstylestrong{1\sphinxhyphen{}period return} of the assets be normal (\sphinxstylestrong{todo} \sphinxstyleemphasis{is this necessary? Can’t one rely on central limit theorem? How long the summation must be for convergence to normal distribution, in presence of \sphinxstylestrong{heavy\sphinxhyphen{}tails} distribution? If one can’t rely on central limit theorem, let use numerical methods to investigate the effect of heavy tails distributions}),
\begin{equation*}
\begin{split}\mathbf{r} \sim \mathscr{N} \left( \boldsymbol{\mu}, \boldsymbol{\sigma}^2 \right) \ .\end{split}
\end{equation*}
\sphinxAtStartPar
Compound return of the portfolio has \sphinxstylestrong{expected value}
\begin{equation*}
\begin{split}\mu^c_p = \mathbb{E}[r^c_p] = \mathbf{w}^T \boldsymbol{\mu} - \dfrac{1}{2}\mathbf{w}^T \boldsymbol{\sigma}^2 \, \mathbf{w}\end{split}
\end{equation*}
\sphinxAtStartPar
and \sphinxstylestrong{variance}
\begin{equation*}
\begin{split}\begin{aligned}
\sigma_{r^c_p}^2
& = \mathbb{E}\left[(r^c_p - \mu^c_p )^2 \right] = \dots = \mathbf{w}^T \boldsymbol{\sigma} \mathbf{w}
\end{aligned}\end{split}
\end{equation*}

\subsection{Shannon demon}
\label{\detokenize{code/notebooks/rebalancing:shannon-demon}}
\sphinxAtStartPar
Sometimes the expected value of the compound return of the rebalanced portfolio can be larger than the expected return of each asset class.
\begin{equation*}
\begin{split}\mu^c_k = \mathbb{E}[r^c_k]  = \mu_k - \dfrac{\sigma_k^2}{2}\end{split}
\end{equation*}
\sphinxAtStartPar
\sphinxstylestrong{Example: 2\sphinxhyphen{}asset portfolio.} As an example, the expected value of the compund return of a 2\sphinxhyphen{}asset rebalanced portfolio,
\begin{equation*}
\begin{split}\mathbb{E} [ r_p^c ] = w_1 \mu_1 + w_2 \mu_2 - \dfrac{1}{2} ( w_1^2 \sigma_1^2 + 2 w_1 w_2 \rho \sigma_1 \sigma_2 + w_2^2 \sigma_2^2 )\end{split}
\end{equation*}
\sphinxAtStartPar
Using \(w_1\), \(w_2\) as independent variables, for any value of the expected return, the expression of the return itself can be represented as a \sphinxstylestrong{conic section} in the \(w_1,w_2\)\sphinxhyphen{}plane. In particular,
\begin{itemize}
\item {} 
\sphinxAtStartPar
for \(\rho \ne 1\), it’s an \sphinxstylestrong{ellipse} (\(\Delta = B^2 - 4 A C < 0\)),

\item {} 
\sphinxAtStartPar
for \(\rho = 1\), it’s a \sphinxstylestrong{parabola} (\(\Delta = 0\))

\end{itemize}

\sphinxAtStartPar
\sphinxstylestrong{Portfolio allocation.} Some constraints may hold on portfolio allocation:
\begin{itemize}
\item {} 
\sphinxAtStartPar
fully\sphinxhyphen{}invested: \(w_1 + w_2 = 1\)

\item {} 
\sphinxAtStartPar
no short\sphinxhyphen{}selling: \(w_1,\ w_2 \ge 0\)

\item {} 
\sphinxAtStartPar
no leverage: \(w_1, \ w_2 \le 1\)

\end{itemize}

\begin{sphinxuseclass}{cell}
\begin{sphinxuseclass}{tag_hide-input}\begin{sphinxVerbatimOutput}

\begin{sphinxuseclass}{cell_output}
\begin{sphinxVerbatim}[commandchars=\\\{\}]
interactive(children=(FloatSlider(value=\PYGZhy{}0.25, description=\PYGZsq{}rho\PYGZsq{}, max=1.0, min=\PYGZhy{}1.0, step=0.01), FloatSlider(v…
\end{sphinxVerbatim}

\noindent\sphinxincludegraphics{{bd2503f6cb1a73480111e234c1cb6ccab548d7a17cdf88899fbc0bbc3ee0933a}.png}

\end{sphinxuseclass}\end{sphinxVerbatimOutput}

\end{sphinxuseclass}
\end{sphinxuseclass}
\sphinxAtStartPar
This plot represents in \sphinxstylestrong{blue} asset allocations of the balanced 2\sphinxhyphen{}asset portfolio with expected value of the compound return larger than the compound return of any individual asset. \sphinxstylestrong{Black line} represents all the possible allocations of a fully invested portfolio, \(w_1+w_2=1\) with no leaverage \(w_k \le 1\) and not short selling \(w_k \ge 0\).


\section{Comparison of portfolios: realizations of stochastic processes}
\label{\detokenize{code/notebooks/rebalancing:comparison-of-portfolios-realizations-of-stochastic-processes}}
\sphinxAtStartPar
In this section, rebalanced portfolio and buy\sphinxhyphen{}and\sphinxhyphen{}hold portfolio are compared.
Different realizations of these two portfolio strategies are built, and used to build statistics, and discuss their properties in terms of \sphinxstylestrong{compound return}, \sphinxstylestrong{drawdowns},…

\sphinxAtStartPar
\sphinxstylestrong{Note.} Here, 1\sphinxhyphen{}period returns are modelled as \sphinxstyleemphasis{normal random variable} so far. Anyways, it’s possible (and suggested) to implement the most suited random process for modelling the return of the desired assets.


\subsection{Useful functions}
\label{\detokenize{code/notebooks/rebalancing:useful-functions}}
\sphinxAtStartPar
A useful function is introduced here to build correlated random variables with the desired expected values and covariance.

\sphinxAtStartPar
Main Colab notebook can be found here: \sphinxurl{https://colab.research.google.com/drive/1n5py0Zf8i3\_jrTTk0AR7Noq2kBYwpaqx?authuser=1\#scrollTo=gmbjbprjCHto}


\subsection{Realizations}
\label{\detokenize{code/notebooks/rebalancing:realizations}}
\begin{sphinxuseclass}{cell}
\begin{sphinxuseclass}{tag_hide-input}\begin{sphinxVerbatimOutput}

\begin{sphinxuseclass}{cell_output}
\begin{sphinxVerbatim}[commandchars=\\\{\}]
[\PYGZlt{}matplotlib.lines.Line2D at 0x7fe290ffafa0\PYGZgt{}]
\end{sphinxVerbatim}

\noindent\sphinxincludegraphics{{4b1eaaef6e8220cd18b38ce936767781629fd524c2cff29cdef55500037f72aa}.png}

\end{sphinxuseclass}\end{sphinxVerbatimOutput}

\end{sphinxuseclass}
\end{sphinxuseclass}

\subsection{Composite return}
\label{\detokenize{code/notebooks/rebalancing:composite-return}}
\begin{sphinxuseclass}{cell}
\begin{sphinxuseclass}{tag_hide-input}\begin{sphinxVerbatimOutput}

\begin{sphinxuseclass}{cell_output}
\begin{sphinxVerbatim}[commandchars=\\\{\}]
Rebalanced portfolio. Compsite return
 Exp. value: 0.085
 Std. dev. : 0.009
Buy\PYGZhy{}and\PYGZhy{}Hold portfolio. Compsite return
 Exp. value: 0.081
 Std. dev. : 0.013
\end{sphinxVerbatim}

\noindent\sphinxincludegraphics{{360de3c5349f6b72e45166cdbb2dbd16dab54c998f129db61190012ad72856c9}.png}

\end{sphinxuseclass}\end{sphinxVerbatimOutput}

\end{sphinxuseclass}
\end{sphinxuseclass}

\subsection{Maximum drawdown}
\label{\detokenize{code/notebooks/rebalancing:maximum-drawdown}}
\begin{sphinxuseclass}{cell}
\begin{sphinxuseclass}{tag_hide-input}\begin{sphinxVerbatimOutput}

\begin{sphinxuseclass}{cell_output}
\begin{sphinxVerbatim}[commandchars=\\\{\}]
Text(0.5, 1.0, \PYGZsq{}Realization with the deepest drawdown\PYGZsq{})
\end{sphinxVerbatim}

\noindent\sphinxincludegraphics{{75f0409ae63bbcd51f74df188a5ae213145e7cb5a9afc2bde1c477d79ddff95c}.png}

\end{sphinxuseclass}\end{sphinxVerbatimOutput}

\end{sphinxuseclass}
\end{sphinxuseclass}
\sphinxstepscope


\chapter{Sequence risk}
\label{\detokenize{code/notebooks/sequence-risk:sequence-risk}}\label{\detokenize{code/notebooks/sequence-risk::doc}}

\section{Introduction}
\label{\detokenize{code/notebooks/sequence-risk:introduction}}
\sphinxAtStartPar
Sequence risk occurs when investment or withdrawal is distributed in time. These two scenarios may be representative of:
\begin{itemize}
\item {} 
\sphinxAtStartPar
Dollar Cost Averaging (\sphinxstylestrong{DCA}, or \sphinxstylestrong{PAC} in Italian for “Piano di Accumulo di Capitale”)

\item {} 
\sphinxAtStartPar
\sphinxstylestrong{Withdrawal} during old age

\end{itemize}

\sphinxAtStartPar
Sequence in time of 1\sphinxhyphen{}period returns may strongly influence the composite return of a portfolio.


\subsection{Mathematical model}
\label{\detokenize{code/notebooks/sequence-risk:mathematical-model}}
\sphinxAtStartPar
In a continuous\sphinxhyphen{}time model, sequence risk of constant\sphinxhyphen{}amount DCA or withdrawal can be modeled with a Goemetric Brownian Motion with “drift”,
\begin{equation*}
\begin{split}d X_t = \mu X_t \, dt + \sigma X_t \, dW_t + C \, dt  \ ,\end{split}
\end{equation*}
\sphinxAtStartPar
being \(C_t\) the rate of investment (\(> 0\)) or withdrawal (\(< 0\)), \(\mu\), \(\sigma\) the expected value and variance of the rate of return. A discrete\sphinxhyphen{}time counterpart may be
\begin{equation*}
\begin{split}\Delta X_ {n,n+1} = \left( \mu_{n,n+1} + \sigma_{n,n+1} W_{n,n+1} \right) \, X_n + C_{n,n+1} \ , \end{split}
\end{equation*}
\sphinxAtStartPar
with \(\mu_{n,n+1}\), \(\sigma_{n,n+1}\) the expected value and the variance of the 1\sphinxhyphen{}period return, \(W_{n,n+1}\) a unit\sphinxhyphen{}variance random variable representing the distribution of the returns from \(n\) to \(n+1\), and \(C_{n,n+1}\) the investment of withdrawal from \(n\) to \(n+1\).


\subsection{Constant investment or withdrawal rate: analytical solution}
\label{\detokenize{code/notebooks/sequence-risk:constant-investment-or-withdrawal-rate-analytical-solution}}
\sphinxAtStartPar
The solution of the continuous\sphinxhyphen{}time equation with reads
\begin{equation*}
\begin{split}X_t = X_0 e^{\left( \mu - \frac{\sigma^2}{2} \right) t + \sigma W_t } +  C \int_{s=0}^{t} e^{\left( \mu - \frac{\sigma^2}{2} \right)(t-s) + \sigma (W_t - W_s)} \, ds\end{split}
\end{equation*}

\section{Realizations}
\label{\detokenize{code/notebooks/sequence-risk:realizations}}

\subsection{Libraries and parameters}
\label{\detokenize{code/notebooks/sequence-risk:libraries-and-parameters}}
\begin{sphinxuseclass}{cell}
\begin{sphinxuseclass}{tag_hide-input}
\end{sphinxuseclass}
\end{sphinxuseclass}

\subsection{Constant Withdrawals}
\label{\detokenize{code/notebooks/sequence-risk:constant-withdrawals}}
\begin{sphinxuseclass}{cell}
\begin{sphinxuseclass}{tag_hide-input}\begin{sphinxVerbatimOutput}

\begin{sphinxuseclass}{cell_output}
\noindent\sphinxincludegraphics{{915d26d8659d693cfa16366fb3047edf0f4a154b4d641b6c865147619a8482b7}.png}

\end{sphinxuseclass}\end{sphinxVerbatimOutput}

\end{sphinxuseclass}
\end{sphinxuseclass}

\subsection{Dollar Cost Averaging (DCA)}
\label{\detokenize{code/notebooks/sequence-risk:dollar-cost-averaging-dca}}
\begin{sphinxuseclass}{cell}
\begin{sphinxuseclass}{tag_hide-input}\begin{sphinxVerbatimOutput}

\begin{sphinxuseclass}{cell_output}
\noindent\sphinxincludegraphics{{23db936fe6339ca0af2a4e8cbca90489bab3b4780412f4e1f19025710abff86c}.png}

\end{sphinxuseclass}\end{sphinxVerbatimOutput}

\end{sphinxuseclass}
\end{sphinxuseclass}
\sphinxstepscope


\part{Asset classes}

\sphinxstepscope


\chapter{Introduction to asset classes}
\label{\detokenize{ch/assets/intro:introduction-to-asset-classes}}\label{\detokenize{ch/assets/intro:fin-edu-assets-intro}}\label{\detokenize{ch/assets/intro::doc}}
\sphinxstepscope


\chapter{Bonds}
\label{\detokenize{ch/assets/bonds:bonds}}\label{\detokenize{ch/assets/bonds:fin-edu-assets-bonds}}\label{\detokenize{ch/assets/bonds::doc}}
\sphinxAtStartPar
…

\sphinxAtStartPar
Here the most general expression for nominal and real \sphinxstylestrong{yield} are derived as a function of prices, face value of coupon, taxation and year to maturity, both in case of coupon reinvestment or not (reinvestment not always possible); a closed form solution is then derived under some assumptions, like constant (or average) rates; the effect on price and yield of credit rating and rating change, coupon, year to maturity are discussed on both examples and real\sphinxhyphen{}world cases.

\sphinxAtStartPar
Extra:
\begin{itemize}
\item {} 
\sphinxAtStartPar
definition of duration

\item {} 
\sphinxAtStartPar
risks: inflation; reinvesment (at lower rates) for bonds with same maturity and different coupons

\item {} 
\sphinxAtStartPar
inflation linked

\end{itemize}


\section{Constant coupon bonds}
\label{\detokenize{ch/assets/bonds:constant-coupon-bonds}}

\subsection{W/o reinvestment}
\label{\detokenize{ch/assets/bonds:w-o-reinvestment}}
\sphinxAtStartPar
At time \(t_0\) the unit price of a bond is \(p_0\); investing \(Y_0\) allows to buy \(N_0 = \frac{Y_0}{p_0}\) titles; each title has the right of receiving net coupon \(C (1 - t)\), with \(t\) taxation rate, per period (here assumed 1\sphinxhyphen{}year coupon range).
\begin{equation*}
\begin{split}N_0 = \dfrac{Y_0}{p_0} = \dfrac{Y_0}{p_{in}} \dfrac{p_{in}}{p_0}\end{split}
\end{equation*}
\sphinxAtStartPar
W/o reinvestment, the number of titles hold is constant and equal to \(N_0\). As capital \(Y_i\) can be written as the product of unit price and number of bond in portfolio, the DCF of a bond w/o coupon reinvestment reads
\begin{equation*}
\begin{split}\begin{aligned}
  \widetilde{DCF} =
  & = - Y_0 + Y_N \prod_{k=1}^N ( 1 + r_k )^{-1} + \sum_{k=1}^{N} N_0 C (1-t) \prod_{j=1}^{k} (1 + r_j )^{-1} \\ 
  & = N_0 \left[ - p_0 + p_N \prod_{k=1}^N ( 1 + r_k )^{-1} + C (1-t) \sum_{k=1}^{N} \prod_{j=1}^{k} (1 + r_j )^{-1} \right] \ , 
\end{aligned}\end{split}
\end{equation*}
\sphinxAtStartPar
This DCF must be corrected a CF at time \(t_N\) corresponding to tax on capital gain if \(p_n > p_0\), discounted as
\begin{equation*}
\begin{split}- N_0( p_N  - p_0) \,  t \,  \prod_{k=1}^{N} (1+r_k)^{-1}  \qquad  (\text{only if $p_N > p_0$})\end{split}
\end{equation*}
\sphinxAtStartPar
The cumulative real return (if the discount ratio is inflation) is the ratio between the \(DCF\) and the actual value of the investment \(Y_0\),
\begin{equation*}
\begin{split}\widetilde{\dfrac{DCF}{Y_0}} = - 1 + \dfrac{p_N}{p_0} \prod_{k=1}^N ( 1 + r_k )^{-1} + \dfrac{C}{p_0} (1-t) \sum_{k=1}^{N} \prod_{j=1}^{k} (1 + r_j )^{-1}  \end{split}
\end{equation*}
\sphinxAtStartPar
If the discount rate is constant, or the average (which average) discount rate is used, the expression of the cumulative return reads
\begin{equation*}
\begin{split}\begin{aligned}
  \dfrac{\widetilde{DCF}}{Y_0} = - 1 + \dfrac{p_N}{p_0} ( 1 + r )^{-N} + \dfrac{C}{p_0} (1-t) \sum_{k=1}^{N} (1 + r )^{-k}  
\end{aligned}\end{split}
\end{equation*}

\subsection{W/ reinvestment}
\label{\detokenize{ch/assets/bonds:w-reinvestment}}

\begin{savenotes}\sphinxattablestart
\centering
\begin{tabulary}{\linewidth}[t]{|T|T|T|T|T|}
\hline
\sphinxstyletheadfamily 
\sphinxAtStartPar
Time
&\sphinxstyletheadfamily 
\sphinxAtStartPar
Cashflows
&\sphinxstyletheadfamily 
\sphinxAtStartPar
\(\Delta\)Quantity
&\sphinxstyletheadfamily 
\sphinxAtStartPar
Quantity
&\sphinxstyletheadfamily 
\sphinxAtStartPar
DF
\\
\hline
\sphinxAtStartPar
\(0\)
&
\sphinxAtStartPar
\(-Y_0\)
&
\sphinxAtStartPar
\(N_0 = \frac{Y_0}{p_0}\)
&
\sphinxAtStartPar
\(N_0 = \frac{Y_0}{p_0}\)
&
\sphinxAtStartPar
1
\\
\hline
\sphinxAtStartPar
\(1\)
&
\sphinxAtStartPar
\(+N_0 C ( 1-t )\)
&
\sphinxAtStartPar

&
\sphinxAtStartPar

&
\sphinxAtStartPar
\((1+r_1)^{-1}\)
\\
\hline
\sphinxAtStartPar
\(1\)
&
\sphinxAtStartPar
\(-N_0 C ( 1-t )\)
&
\sphinxAtStartPar
\(N_1 = \frac{N_0 C (1-t)}{p_1}\)
&
\sphinxAtStartPar
\(N_{0:1} = N_0+N_1\)
&
\sphinxAtStartPar
\((1+r_1)^{-1}\)
\\
\hline
\sphinxAtStartPar
\(2\)
&
\sphinxAtStartPar
\(+N_{0:1} C ( 1-t )\)
&
\sphinxAtStartPar

&
\sphinxAtStartPar

&
\sphinxAtStartPar
\((1+r_1)^{-1} (1+r_2)^{-1}\)
\\
\hline
\sphinxAtStartPar
\(2\)
&
\sphinxAtStartPar
\(-N_{0:1} C ( 1-t )\)
&
\sphinxAtStartPar
\(N_2 = \frac{N_{0:1} C (1-t)}{p_2}\)
&
\sphinxAtStartPar
\(N_{0:2} = N_0+N_1 + N_2\)
&
\sphinxAtStartPar
\((1+r_1)^{-1} (1+r_2)^{-1}\)
\\
\hline
\sphinxAtStartPar
…
&
\sphinxAtStartPar

&
\sphinxAtStartPar

&
\sphinxAtStartPar

&
\sphinxAtStartPar

\\
\hline
\sphinxAtStartPar
\(T-1\)
&
\sphinxAtStartPar
\(+N_{0:T-2} C ( 1-t )\)
&
\sphinxAtStartPar

&
\sphinxAtStartPar

&
\sphinxAtStartPar
\(\prod_{k=1}^{T-1} (1+r_k)^{-1}\)
\\
\hline
\sphinxAtStartPar
\(T-1\)
&
\sphinxAtStartPar
\(-N_{0:T-2} C ( 1-t )\)
&
\sphinxAtStartPar
\(N_{T-1} = \frac{N_{0:T-2} C (1-t)}{p_{T-1}}\)
&
\sphinxAtStartPar
\(N_{0:T-1} = \sum_{k=0}^{T-1} N_k\)
&
\sphinxAtStartPar
\(\prod_{k=1}^{T-1} (1+r_k)^{-1}\)
\\
\hline
\sphinxAtStartPar
\(T\)
&
\sphinxAtStartPar
\(+N_{0:T-1} C ( 1-t )\)
&
\sphinxAtStartPar

&
\sphinxAtStartPar

&
\sphinxAtStartPar
\(\prod_{k=1}^{T  } (1+r_k)^{-1}\)
\\
\hline
\sphinxAtStartPar
\(T\)
&
\sphinxAtStartPar
\(+N_{0:T-1} p_T\)
&
\sphinxAtStartPar

&
\sphinxAtStartPar

&
\sphinxAtStartPar
\(\prod_{k=1}^{T  } (1+r_k)^{-1}\)
\\
\hline
\end{tabulary}
\par
\sphinxattableend\end{savenotes}

\sphinxAtStartPar
All the cashflows from coupons are immediately reinvested so the DCF is
\begin{equation*}
\begin{split}\begin{aligned}
  DCF 
  & = - Y_0 + \underbrace{N_{0:T-1} \left( p_T + C (1-t)\right)}_{Y_T} \, \underbrace{ \prod_{k=1}^T (1+r_k)^{-1} }_{DF_T} = \\
  & = - Y_0 + Y_T \,  DF_T \ ,
\end{aligned}\end{split}
\end{equation*}
\sphinxAtStartPar
with
\begin{equation*}
\begin{split}\begin{aligned}
  N_{0:T-1}
  & = N_{0:T-2} + N_{T-1} =  N_{0:T-2} + N_{0:T-2} \frac{ C (1-t)}{p_{T-1}} = N_{0:T-2} \left[ 1 + \frac{ C (1-t)}{p_{T-1}} \right] = \\
  & = N_{0:T-3} \left[ 1 + \frac{ C (1-t)}{p_{T-2}} \right] \left[ 1 + \frac{ C (1-t)}{p_{T-1}} \right] = \\
  & = \dots = \\
  & = N_{0:1} \prod_{k=2}^{T-1} \left[ 1 + \frac{ C (1-t)}{p_{k}} \right] = \\ 
  & = N_{0  } \prod_{k=1}^{T-1} \left[ 1 + \frac{ C (1-t)}{p_{k}} \right] \ .
\end{aligned}\end{split}
\end{equation*}
\sphinxAtStartPar
Cumulative discounted return reads
\begin{equation*}
\begin{split}\begin{aligned}
  \dfrac{DCF}{Y_0} 
  & = - 1 + \dfrac{Y_T}{Y_0} DF_{T} = \\
  & = - 1 + \dfrac{N_0}{N_0 \, p_0} \prod_{k=1}^{T-1} \left( 1+ \dfrac{C(1-t)}{p_k} \right) \, ( p_T + C(t-1) ) \, DF_T \\
  & = - 1 + \dfrac{p_T}{p_0} \prod_{k=1}^{T} \left( 1+ \dfrac{C(1-t)}{p_k} \right) \, DF_T \\
  & = - 1 + \dfrac{p_T}{p_0} \prod_{k=1}^{T} \left( \dfrac{ 1+ \frac{C(1-t)}{p_k} }{1+r_k} \right) \ .
\end{aligned}\end{split}
\end{equation*}
\sphinxAtStartPar
Composite discounted return is obtained, after writing the diiscounted cashflow as the difference between discounted cashflow at time \(t_T\) and \(t_0\), \(DCF = Y_T \ DF_T - T_0\),
\begin{equation*}
\begin{split}(1 + DCAGR)^T = \dfrac{Y_T \, DF_T}{Y_0} = \dfrac{DCF}{Y_0} + 1 = \dfrac{p_T}{p_0} \, \prod_{k=1}^{T} \left( \dfrac{ 1+ \frac{C(1-t)}{p_k} }{1+r_k} \right)\end{split}
\end{equation*}\begin{equation*}
\begin{split}DCAGR = \left( \dfrac{p_T}{p_0} \, \prod_{k=1}^{T}  \dfrac{ 1+ \frac{C(1-t)}{p_k} }{1+r_k} \right)^{\frac{1}{T}} - 1\end{split}
\end{equation*}
\sphinxAtStartPar
\sphinxstylestrong{If}%
\begin{footnote}[1]\sphinxAtStartFootnote
It’s a big if. Even if credit rating and inflation are constant throughout the life of the bond, years to maturity decreases and thus \sphinxhyphen{} usually \sphinxhyphen{} the required rate decreases as well.
%
\end{footnote} price of the bond is constant throughout its whole life, \(p_k = 1\), \(\forall k=0:T\), and discount rate \(r\) is constant, the number of held bonds at time \(T-1\) is
\begin{equation*}
\begin{split}N_{0:T-1} = N_0 \left( 1 + C(1-t) \right)^{T-1} \ ,\end{split}
\end{equation*}
\sphinxAtStartPar
the discounted cashflow is
\begin{equation*}
\begin{split}\begin{aligned}
  DCF 
  & = - N_0 + N_0 \left( 1 + C(1-t) \right)^{T-1} ( 1 + C(1-t) ) \left( 1 + r \right)^{-T} = \\
  & = N_0 \left[ - 1 +  \left( \dfrac{ 1 + C(1-t) }{ 1 + r } \right)^{T} \right] \ ,
\end{aligned}\end{split}
\end{equation*}
\sphinxAtStartPar
cumulative discounted return
\begin{equation*}
\begin{split}\dfrac{DCF}{Y_0} = - 1 + \left( \dfrac{ 1 + C(1-t) }{ 1 + r } \right)^{T}\end{split}
\end{equation*}
\sphinxAtStartPar
and the composite discounted return reads
\begin{equation*}
\begin{split}DCAGR = \dfrac{1 + C(1-t)}{1+r} - 1 \ .\end{split}
\end{equation*}

\bigskip\hrule\bigskip


\sphinxstepscope


\chapter{Equity}
\label{\detokenize{ch/assets/equity:equity}}\label{\detokenize{ch/assets/equity:fin-edu-assets-equity}}\label{\detokenize{ch/assets/equity::doc}}
\sphinxAtStartPar
Equity valuation is a mix of \sphinxstylestrong{common sense}, math, expectations and estimation, and cooking: buying shares of a firm (directly or with a fund) means buying a (tiny) share of the company, a business producing goods and/or services, able to generate earnings/free cash flows. When you hold shares, you’re participating to the business and have rights to the profits.
\subsubsection*{Detailed introduction}

\sphinxAtStartPar
Equity valuation blends common sense, mathematics, expectations, estimation—and a bit of art. Buying shares in a company, whether directly or through a fund, means owning a (tiny) stake in a real business that produces goods and/or services and has the potential to generate earnings or free cash flows. As a shareholder, you are not just investing in market prices—you’re becoming a part\sphinxhyphen{}owner of the enterprise. This ownership entitles you to a share of the company’s profits through dividends or capital appreciation. It also comes with certain rights and responsibilities, especially during difficult periods.

\sphinxAtStartPar
When companies face financial stress or pursue growth opportunities, they may issue new shares to raise capital. This can lead to dilution, reducing the percentage ownership of existing shareholders. However, shareholders often have preemptive rights, allowing them to participate in new issuances to maintain their ownership stake. Moreover, owning equity means having a claim on the residual value of the company—what’s left after all debts are paid—in both prosperous and challenging times. Understanding these dynamics is crucial to valuing equity: you’re not just buying into today’s performance, but into a stream of future cash flows and the complex, evolving structure of ownership.

\sphinxAtStartPar
\sphinxstylestrong{Sensitivity analysis} could provide an estimate of the effects of different parameters/assumptions on the final result.

\sphinxAtStartPar
\sphinxstylestrong{Different valuation methods} exist, and can be broadly classified in
\begin{itemize}
\item {} 
\sphinxAtStartPar
\sphinxstylestrong{comparison} approach: P/E, EV/EBITDA, or other indices used to compare companies of the same sector, marked, dimension,…%
\begin{footnote}[1]\sphinxAtStartFootnote
It’s not always possible to find “equivalent” companies for the comparison…; P/E, EV/EBITDA,… whould be projected into the future to keep into account future in the value of a firm.
%
\end{footnote}

\item {} 
\sphinxAtStartPar
\sphinxstylestrong{intrinsic value} approach, based on \sphinxstylestrong{DCF}

\item {} 
\sphinxAtStartPar
…other methods for general firms (cost approach,…); valuation of financials;…

\end{itemize}


\section{Comparison}
\label{\detokenize{ch/assets/equity:comparison}}\label{\detokenize{ch/assets/equity:fin-edu-assets-equity-comparison}}

\section{Intrinsic value}
\label{\detokenize{ch/assets/equity:intrinsic-value}}\label{\detokenize{ch/assets/equity:fin-edu-assets-equity-intrinsic}}\begin{itemize}
\item {} 
\sphinxAtStartPar
Future cash flows are estimated,

\item {} 
\sphinxAtStartPar
CFs are discounted, usually for the \(WACC\) (Weighted Average Cost of Capital) to find the \(NPV\) (net present value) of the \sphinxstylestrong{enterprise value} \(EV\)

\item {} 
\sphinxAtStartPar
Cash and equivalents are added to the \(NPV\) to find the \sphinxstylestrong{equity value}

\end{itemize}

\begin{sphinxadmonition}{note}{\protect\(WACC\protect\)}
\begin{equation*}
\begin{split}WACC = \dfrac{E}{V} R_e + \dfrac{D}{V} R_d (1 - t)\end{split}
\end{equation*}
\sphinxAtStartPar
being \(R_e\) the \sphinxstylestrong{cost of equity} and \(R_d\) the \sphinxstylestrong{cost of debt} (maybe the easiest part to estimated accurately, since the debt structure is usually known/programmed). The factor \((1-t)\) usually appears as interest payments are tax\sphinxhyphen{}deductible.
\end{sphinxadmonition}

\begin{sphinxadmonition}{note}{Equity Risk Premium \protect\(R_e\protect\) \sphinxhyphen{} Sharpe}

\sphinxAtStartPar
Following W.Sharpe, equity risk premium can be estimated as
\begin{equation*}
\begin{split}R_e = R_f + (R_m + R_f) \beta \ ,\end{split}
\end{equation*}
\sphinxAtStartPar
being \(R_f\) the risk\sphinxhyphen{}free rate (usaully 10Y US Treasuries), and \(R_m\) the annual return of the market/sector of the investment, \(\beta\) is a measure of risk or stock volatility of returns of the investment relative to that of the market/sector.
\end{sphinxadmonition}


\bigskip\hrule\bigskip


\sphinxstepscope


\section{Three Financial Statements}
\label{\detokenize{ch/assets/financial-statements:three-financial-statements}}\label{\detokenize{ch/assets/financial-statements:fin-edu-financial-statements}}\label{\detokenize{ch/assets/financial-statements::doc}}
\sphinxAtStartPar
Financial statements are written records that illustrates the business activities and the financial performance of a company. In most cases they are audited to ensure accuracy for tax, financing, or investing purposes.

\sphinxAtStartPar
\sphinxstylestrong{Uses.} \sphinxstyleemphasis{Management} uses them for decision\sphinxhyphen{}making, budgeting and performance evaluation. \sphinxstyleemphasis{Investors} use them to asses profitability, financial health, future performance, and creditworthiness (especially \sphinxstyleemphasis{lenders}).
\begin{itemize}
\item {} 
\sphinxAtStartPar
\sphinxstylestrong{Income statement}: company performance (profit and loss) over a period. Broadly speaking:
\begin{equation*}
\begin{split}\text{net earnings} = ( \text{revenues} - \text{total expenses} ) \times ( 1 - \text{tax rate} ) , \end{split}
\end{equation*}
\sphinxAtStartPar
with total expenses = operative (labor + non\sphinxhyphen{}labor + DA) + Interest (due to debt holders), and “partial earnings” EBITDA, EBIT, EBT with trivial definition (Earnings Before: I:interest, DA: depreciation and amortization, T: tax)

\item {} 
\sphinxAtStartPar
\sphinxstylestrong{Balance sheet}: financial position at a specific point in time, in terms of:
\begin{itemize}
\item {} 
\sphinxAtStartPar
assets: cash and equivalent + acc.receiv. + inventory + PPE (Plant property and equipment, subject to CapEx and depreciation, \(PPE(n) = PPE(n-1) + \text{CapEx}(n) + \text{DA}(n)\)

\item {} 
\sphinxAtStartPar
liabilities: debt + acc.pay.

\item {} 
\sphinxAtStartPar
equity:
\begin{equation*}
\begin{split}\begin{aligned}
       \text{retained earnings}(n) & = \text{retained earnings}(n-1) + \text{net earnings}(n) - \text{dividends}(n) \\
       \text{shareholder equity}(n) & = \text{equity capital}(n) + \text{retained earnings}(n) \ ,
     \end{aligned}\end{split}
\end{equation*}
\sphinxAtStartPar
being \(\text{retained earnings}(n)\) the \sphinxstylestrong{cumulative} retained earnings not distributed to shareholders.

\sphinxAtStartPar
The 2 contributions \(\text{shareholder equity}\), \(\text{total liabilities}\) shows how the  compnay’s asset are financed: either through capital raised or retained earnings (equity), or through debt (liabilities). The \sphinxstylestrong{identity}
\begin{equation*}
\begin{split}\text{total liabilities} + \text{shareholders equity} = \text{total asset}\end{split}
\end{equation*}
\sphinxAtStartPar
must hold in a proper filled balance.

\end{itemize}

\item {} 
\sphinxAtStartPar
\sphinxstylestrong{Cash flow statement} tracks the flows of cash in and out of the business over a period. Cashflows over a period modifies cash,
\begin{equation*}
\begin{split}\begin{aligned}
    \text{closing cash}(n) & = \text{opening cash}(n) + \text{total cashflow}(n) \\
    \text{opening cash}(n) & = \text{closing cash}(n-1)
  \end{aligned}\end{split}
\end{equation*}
\sphinxAtStartPar
Cashflows are usually classified as:
\begin{itemize}
\item {} 
\sphinxAtStartPar
operating CF (DA is added back to net income, since it’s not a cashflow going anywhere; it lowers income, but it’s not a cashflow)

\item {} 
\sphinxAtStartPar
investing CF

\item {} 
\sphinxAtStartPar
financing CF
\begin{equation*}
\begin{split}\begin{aligned}
      \text{ Op.CF}(n) & = \text{net earnings}(n) + \text{DA}(n) - \Delta \text{WC}(n) \\
      \text{Inv.CF}(n) & = \text{investment in PPE}(n) \\
      \text{Fin.CF}(n) & = \text{issuance of debt}(n) + \text{issuance of equity}(n) - \text{dividends}(n) \\
    \end{aligned}\end{split}
\end{equation*}
\sphinxAtStartPar
being \(\text{WC}(n) = \text{acc.rec}(n) + \text{inventory}(n) - \text{acc.pay}(n)\) the \sphinxstylestrong{working capital}.

\end{itemize}

\end{itemize}

\sphinxstepscope


\chapter{ETFs}
\label{\detokenize{ch/assets/etfs:etfs}}\label{\detokenize{ch/assets/etfs:fin-edu-assets-etfs}}\label{\detokenize{ch/assets/etfs::doc}}
\sphinxstepscope


\part{Asset allocation}

\sphinxstepscope


\chapter{Introduction to investing}
\label{\detokenize{ch/investing/intro:introduction-to-investing}}\label{\detokenize{ch/investing/intro:fin-edu-investing-intro}}\label{\detokenize{ch/investing/intro::doc}}
\sphinxstepscope


\chapter{Modern Portfolio Theory}
\label{\detokenize{ch/investing/mpt:modern-portfolio-theory}}\label{\detokenize{ch/investing/mpt:fin-edu-investing-mpt}}\label{\detokenize{ch/investing/mpt::doc}}
\sphinxAtStartPar
Modern portfolio theory results in a strategy of asset allocation minimizing portfolio risk \sphinxhyphen{} measured as volatility \sphinxhyphen{} for a given value of the portfolio expected return.
\subsubsection*{Asset Modeling}

\sphinxAtStartPar
A set of \(N\) assets is available. Their return (over a defined period)%
\begin{footnote}[1]\sphinxAtStartFootnote
How to estimate asset return, at least in terms of expected value and variance? And how to estimate correlation of the random variables?
%
\end{footnote} is represented by a multi\sphinxhyphen{}dimensional random variable, \(\mathbf{X}\), with expected value and variance
\begin{equation*}
\begin{split}\begin{aligned}
  \overline{\mathbf{X}} & := \mathbb{E}\left[ \mathbf{X} \right] \\
  \boldsymbol{\sigma}^2 & := \mathbb{E}\left[ \Delta \mathbf{X} \, \Delta \mathbf{X}^T \right] \\
\end{aligned}\end{split}
\end{equation*}
\sphinxAtStartPar
with \(\Delta \mathbf{X} := \mathbf{X} - \overline{\mathbf{X}}\).
\subsubsection*{Asset allocation. Constraints}

\sphinxAtStartPar
A portfolio, without short or leverage positions on these assets, can be represented with the set of weights (proportion) of the assets \(\mathbf{w}\), with
\begin{equation*}
\begin{split}\begin{aligned}
  & \sum_n w_n = 1 \\
  & 0 \le w_n \le 1 \quad , \qquad \forall n = 1:N  \\
\end{aligned}\end{split}
\end{equation*}\subsubsection*{Portfolio return}

\sphinxAtStartPar
Portfolio return is
\begin{equation*}
\begin{split}\mathbf{X} = \mathbf{w}^T \mathbf{X} \ ,\end{split}
\end{equation*}
\sphinxAtStartPar
From linearity, its expected value reads
\begin{equation*}
\begin{split}\begin{aligned}
  \overline{X} = \mathbb{E} \left[ X \right] =  \mathbb{E} \left[ \mathbf{w}^T \mathbf{X} \right] = \mathbf{w}^T \, \overline{\mathbf{X}} 
\end{aligned}\end{split}
\end{equation*}
\sphinxAtStartPar
and its variance
\begin{equation*}
\begin{split}\begin{aligned}
  \sigma^2
  & = \mathbb{E} \left[ (X-\overline{X})^2 \right]      
    = \mathbb{E} \left[ \mathbf{w}^T \left( \mathbf{X} - \overline{\mathbf{X}} \right)\left( \mathbf{X} - \overline{\mathbf{X}} \right)^T \mathbf{w}  \right] = \\
  & = \mathbf{w}^T \mathbb{E} \left[ \Delta \mathbf{X} \, \Delta \mathbf{X}^T \right] \mathbf{w}
    = \mathbf{w}^T \boldsymbol{\sigma}^2 \, \mathbf{w}
\end{aligned}\end{split}
\end{equation*}\label{ch/investing/mpt:example-0}
\begin{sphinxadmonition}{note}{Example 15.1 (Properties of variance matrix)}



\sphinxAtStartPar
Covariance matrix is symmetric definite positive. Symmetry readily follows
\begin{equation*}
\begin{split}\sigma_{ij} = \mathbb{E}\left[ \Delta X_i \, \Delta X_j \right]\end{split}
\end{equation*}
\sphinxAtStartPar
The matrix is definite positive as

\sphinxAtStartPar
…
\end{sphinxadmonition}
\subsubsection*{Modern Portfolio Theory, as a constrained optimization problem}

\sphinxAtStartPar
Modern portfolio theory has its own “optimal” asset allocation \(\mathbf{w}^*\left(\overline{X}\right)\) \sphinxhyphen{} with the desired expected return as the parameter, fixed during the optimization \sphinxhyphen{} as the asset allocation for which
\begin{equation*}
\begin{split}
  \min_{\mathbf{w}} \sigma^2 \quad \text{s.t.}
\begin{aligned}
  & \qquad \mathbf{w}^T \, \overline{\mathbf{X}} = \overline{X} \\
  & \qquad \mathbf{w}^T \, \mathbf{1} = 1 \\
  & \qquad 0 \le w_i \le 1
\end{aligned}\end{split}
\end{equation*}

\bigskip\hrule\bigskip


\sphinxstepscope


\chapter{Capital Asset Pricing Model}
\label{\detokenize{ch/investing/capm:capital-asset-pricing-model}}\label{\detokenize{ch/investing/capm:fin-edu-investing-capm}}\label{\detokenize{ch/investing/capm::doc}}
\sphinxstepscope


\chapter{Strategic and Tactical Asset Allocation}
\label{\detokenize{ch/investing/strategic-tactical:strategic-and-tactical-asset-allocation}}\label{\detokenize{ch/investing/strategic-tactical:fin-edu-investing-trategic-tactical}}\label{\detokenize{ch/investing/strategic-tactical::doc}}
\sphinxAtStartPar
\sphinxstylestrong{Strategic Asset Allocation, (SAA)}. SAA is a long\sphinxhyphen{}term protfolio strategy, choosing asset class allocation following investor goals and constraints{[}\textasciicircum{}ips{]}. SAA involves \DUrole{xref,myst}{\sphinxstylestrong{rebalancing}}, to keep deviation from the desired asset allocation and volatility of the protfolio within a desired range.



\sphinxAtStartPar
\sphinxstylestrong{Tactical Asset Allocation, (TAA).} A TAA strategy uses a discretionary or quantitative investment model to take advantage of \sphinxstylestrong{inefficiencies or temporary imbalances} among different asset classes. 

\begin{sphinxadmonition}{note}{How many actively managed funds beats the market?}

\sphinxAtStartPar
\sphinxstylestrong{todo}
\end{sphinxadmonition}

\sphinxstepscope


\chapter{Rebalancing}
\label{\detokenize{ch/investing/rebalancing:rebalancing}}\label{\detokenize{ch/investing/rebalancing:fin-edu-investing-rebalancing}}\label{\detokenize{ch/investing/rebalancing::doc}}
\sphinxAtStartPar
\sphinxstylestrong{Reasons.}
\begin{itemize}
\item {} 
\sphinxAtStartPar
risk management:
\begin{itemize}
\item {} 
\sphinxAtStartPar
adjust risk for the period of life, and risk\sphinxhyphen{}level

\item {} 
\sphinxAtStartPar
correct drift towards the asset with highest return, as it affects the

\end{itemize}

\end{itemize}

\sphinxAtStartPar
\sphinxstylestrong{Strategies.}
In a passive investment strategy, rebalancing shoudl be triggered by some rules, to be applied automatically. As an example:
\begin{itemize}
\item {} 
\sphinxAtStartPar
periodic rebalancing: rebalancing with constant time interval

\item {} 
\sphinxAtStartPar
deviation\sphinxhyphen{}triggered: rebalancing when asset allocation deviation from the traget allocation exceeds a prescribed threshold. E.g. approximately 10\% of a 60\sphinxhyphen{}40 portfolio going to 65\sphinxhyphen{}35 (introduction of episode \sphinxstylestrong{152}, as a summary of episode \sphinxstylestrong{117})

\item {} 
\sphinxAtStartPar
using contributions/withdrawals

\end{itemize}

\sphinxAtStartPar
\sphinxstylestrong{Effects of rebalancing.} In different situations one of the following effects occurs:
\begin{itemize}
\item {} 
\sphinxAtStartPar
Volatility is reduced

\item {} 
\sphinxAtStartPar
Risk\sphinxhyphen{}adjusted return improves

\item {} 
\sphinxAtStartPar
Return of the portfolio is increased

\end{itemize}

\sphinxAtStartPar
Often, a rebalanced\sphinxhyphen{}portfolio return is larger than the weighted average of the returns of the assets of the portfolio. Shannon demon is the mathematical reason for that, \sphinxstyleemphasis{for creating return “out of thin air”}.
\label{ch/investing/rebalancing:example-0}
\begin{sphinxadmonition}{note}{Example 18.1 (Shannon demon \sphinxhyphen{} on a coin flip)}



\sphinxAtStartPar
Starting with \(100\)€, and a fair coin with \(50\%\) probability of for each outcome, either \(H\):head or \(T\):tail. If outcome is \(H\) you gain \(50 \%\), if outcome is \(T\) you lose \(33.3 \%\).
\begin{itemize}
\item {} 
\sphinxAtStartPar
If I play with all the money I have, what is the expected amount at the end of the game, for a sufficiently large number of toss?

\end{itemize}

\sphinxAtStartPar
Now, let’s change the strategy: I bet only \(50 \%\) of the amount I have. What’s the expected amount at the end of the game?
\end{sphinxadmonition}
\label{ch/investing/rebalancing:example-1}
\begin{sphinxadmonition}{note}{Example 18.2 (Nassin Taleb, is the coin fair?)}



\sphinxAtStartPar
After 10 tosses with 10 heads, how would you bet on the next toss?
\end{sphinxadmonition}
\label{ch/investing/rebalancing:example-2}
\begin{sphinxadmonition}{note}{Example 18.3 (Kelly criterion)}


\end{sphinxadmonition}
\label{ch/investing/rebalancing:example-3}
\begin{sphinxadmonition}{note}{Example 18.4 (Does rebalancing improve return, thanks to Shannon demon?)}



\sphinxAtStartPar
Yes, for a portfolio with 2 assets with similar returns and low correlation.
E.g.:
\begin{itemize}
\item {} 
\sphinxAtStartPar
\sphinxstylestrong{S\&P500} and \sphinxstylestrong{gold} (50\%\sphinxhyphen{}50\%) from 1972 to 2008 (cherry\sphinxhyphen{}picking?): CAGR with annual rebalancing: 10.3\%, while S\&P: 9.4\% and gold: 8.2\%.

\item {} 
\sphinxAtStartPar
S\&P500 and gold (50\%\sphinxhyphen{}50\%) from 1972 to 2023 (cherry\sphinxhyphen{}picking?): CAGR with annual rebalancing: 10.2\%, while S\&P: 10.5\% and gold: 7.5\%, but with lower volatility, lower max drawdown and a better Sharpe ratio

\item {} 
\sphinxAtStartPar
S\&P500 and Treasury (50\%\sphinxhyphen{}50\%) from 1972 to 2023 (cherry\sphinxhyphen{}picking?): CAGR with annual rebalancing: 9.3\%, while S\&P: 10.5\% and gold: 6.4\%, but with lower volatility, lower max drawdown and a better Sharpe ratio. Without rebalancing: 9.6\% (higher!) as equity has much higher return and drift occurs over 50\sphinxhyphen{}year period.

\item {} 
\sphinxAtStartPar
MSCI World and FTSE G7…

\end{itemize}

\sphinxAtStartPar
No, with 2 assets with 2 assets with very different returns. Anyways, if they have low correlation, rebalancing (may?) reduce volatility, improves risk\sphinxhyphen{}adjusted return, or both.
\end{sphinxadmonition}


\section{Resources}
\label{\detokenize{ch/investing/rebalancing:resources}}\begin{itemize}
\item {} 
\sphinxAtStartPar
{\hyperref[\detokenize{ch/people/the_bull_guests:fin-edu-resources-the-bull}]{\sphinxcrossref{\DUrole{std,std-ref}{The Bull}}}}
\begin{itemize}
\item {} 
\sphinxAtStartPar
\sphinxstylestrong{217.} Il modo migliore per Ribilanciare il portafoglio

\item {} 
\sphinxAtStartPar
\sphinxstylestrong{152.} La magia del ribilanciamento e il demone di Shannon

\item {} 
\sphinxAtStartPar
\sphinxstylestrong{117.} Come ribilanciare il portafoglio (e previsioni per i prossimi 10 anni)

\end{itemize}

\item {} 
\sphinxAtStartPar
{\hyperref[\detokenize{ch/people/list:fin-edu-resources-people-arnott}]{\sphinxcrossref{\DUrole{std,std-ref}{R.Arnott}}}}. Over\sphinxhyphen{}rebalancing

\item {} 
\sphinxAtStartPar
\sphinxhref{https://www.marketsentiment.co/p/shannons-demon}{market sentiment about Shannon demon}

\end{itemize}

\sphinxstepscope


\part{Extra}

\sphinxstepscope


\chapter{Extra and Random}
\label{\detokenize{ch/extra/intro:extra-and-random}}\label{\detokenize{ch/extra/intro:fin-edu-extra-intro}}\label{\detokenize{ch/extra/intro::doc}}
\sphinxstepscope


\chapter{Euristhics and historical correlations}
\label{\detokenize{ch/extra/euristhics:euristhics-and-historical-correlations}}\label{\detokenize{ch/extra/euristhics:fin-edu-extra-euristhics}}\label{\detokenize{ch/extra/euristhics::doc}}

\section{Expected returns in the stoch market}
\label{\detokenize{ch/extra/euristhics:expected-returns-in-the-stoch-market}}

\subsection{Shiller P/E and S\&P500 10\sphinxhyphen{}year annualised forward returns}
\label{\detokenize{ch/extra/euristhics:shiller-p-e-and-s-p500-10-year-annualised-forward-returns}}\begin{itemize}
\item {} 
\sphinxAtStartPar
from Invesco, \sphinxhref{https://www.invesco.com/apac/en/institutional/insights/market-outlook/applied-philosophy-the-shiller-PE-and-SP-500-returns.html}{Applied philosophy: The Shiller P/E and the S\&P500 returns}

\end{itemize}


\subsection{Bogle Expected Return Formula}
\label{\detokenize{ch/extra/euristhics:bogle-expected-return-formula}}\begin{itemize}
\item {} 
\sphinxAtStartPar
Comment by {\hyperref[\detokenize{ch/people/list:fin-edu-resources-people-carlson}]{\sphinxcrossref{\DUrole{std,std-ref}{Ben Carlson}}}}, on his \sphinxhref{https://awealthofcommonsense.com/2025/06/expected-returns-in-the-stock-market/}{website}

\end{itemize}

\sphinxstepscope


\part{People}

\sphinxstepscope


\chapter{Resources, People and Firms}
\label{\detokenize{ch/people/list:resources-people-and-firms}}\label{\detokenize{ch/people/list:fin-edu-resources-list}}\label{\detokenize{ch/people/list::doc}}

\section{Tools}
\label{\detokenize{ch/people/list:tools}}\label{\detokenize{ch/people/list:fin-edu-resources-tools}}
\sphinxAtStartPar
These tools have not been tested and verified here. I decline any responsibility about their use.


\subsection{Simple tools for investors}
\label{\detokenize{ch/people/list:simple-tools-for-investors}}\label{\detokenize{ch/people/list:fin-edu-resources-tools-simple-tools}}
\sphinxAtStartPar
{[}Simple tools for investors{]} (\sphinxurl{https://www.simpletoolsforinvestors.eu/index.shtml}): \sphinxstylestrong{bond} monitor, calculators, tools (minus\sphinxhyphen{}eater, laddering,…)


\subsection{Interactive Asset Allocation \sphinxhyphen{} by Research Affiliates}
\label{\detokenize{ch/people/list:interactive-asset-allocation-by-research-affiliates}}
\sphinxAtStartPar
\sphinxhref{https://interactive.researchaffiliates.com/asset-allocation\#!/?currency=EUR\&model=ER\&scale=Log\&tab=views\&terms=Real\&vs=N4IgxgrgTlCmB2AXAygCwIYAdYGED2ANnlCAFzwQEEA0IqAlgCaMICCAzu7Iu2QNoAmAJzUAzADZqAVinVxABmoAOJdQCMAdkmbVmkWqGyB8xcaPyN1AWqNrJAgUYAsIx6alqrUgdSdq1ALq0DMxsnNwAKgCe2LykfEF0TCzwHFyIOOiIsADmxPSwcQnByQj4EEhQBUUg6GAgiSEp5ZVR0bH8ibAAHpjo8CyM7YWdtD19A7CMmdl5VSPxXb39gy2IUG0xC3wgACKwAG6wRNiMILQAogC2sFA59PA55yAAspSI9AC0axsNtOxgKAQABGwNuADV0AQIAtQGAsLBPlAsvQ8J8CA9YGRQARYAAzRBkRQgKo5VCE0jyAC\%2BtDyB0Qn3YJ0RGPgWNIOPxFOJpPJRJp4DgjHoDKZeGw6Mx2JAuIJRNovO5AswxEQeMIqPYn2wUHVUCu-TALKlHJlXPlJPoZKVVKpQA}{Expected return vs. volatility plot of different asset classes on different time horizons} by {\hyperref[\detokenize{ch/people/list:fin-edu-resources-firms-research-affiliates}]{\sphinxcrossref{\DUrole{std,std-ref}{Research Affiliates}}}}.


\subsection{Gregory Gundersen Blog}
\label{\detokenize{ch/people/list:gregory-gundersen-blog}}
\sphinxAtStartPar
\sphinxhref{https://gregorygundersen.com/blog/}{Gregory Gundersen blog}


\section{People}
\label{\detokenize{ch/people/list:people}}\label{\detokenize{ch/people/list:fin-edu-resources-people}}

\subsection{Arnott, Robert}
\label{\detokenize{ch/people/list:arnott-robert}}\label{\detokenize{ch/people/list:fin-edu-resources-people-arnott}}
\sphinxAtStartPar
Founder of {\hyperref[\detokenize{ch/people/list:fin-edu-resources-firms-research-affiliates}]{\sphinxcrossref{\DUrole{std,std-ref}{Research Affiliates}}}}.


\subsection{Faber, Meb(ane)}
\label{\detokenize{ch/people/list:faber-meb-ane}}\label{\detokenize{ch/people/list:fin-edu-resources-people-faber}}
\sphinxAtStartPar
Co\sphinxhyphen{}founder and Chief Investment Officier at {\hyperref[\detokenize{ch/people/list:fin-edu-resources-firms-cambria}]{\sphinxcrossref{\DUrole{std,std-ref}{Cambria Investment Management}}}}.


\subsection{Wigglesworth, Robin}
\label{\detokenize{ch/people/list:wigglesworth-robin}}\label{\detokenize{ch/people/list:fin-edu-resources-people-wigglesworth}}
\sphinxAtStartPar
Financial Times’ global finance correspondent, author of Trillions a book on the past, present and future of passive investing.


\subsection{Zweig, Jason}
\label{\detokenize{ch/people/list:zweig-jason}}\label{\detokenize{ch/people/list:fin-edu-resources-people-zweig}}
\sphinxAtStartPar
Columnist for the Wall Street Journal since 2008. A Safe Heaven for Intelligent Investor.


\subsection{Ritholtz, Barry}
\label{\detokenize{ch/people/list:ritholtz-barry}}\label{\detokenize{ch/people/list:fin-edu-resources-people-ritholtz}}

\subsection{Maggiulli, Nick}
\label{\detokenize{ch/people/list:maggiulli-nick}}\label{\detokenize{ch/people/list:fin-edu-resources-people-maggiulli}}

\subsection{Carlson, Ben}
\label{\detokenize{ch/people/list:carlson-ben}}\label{\detokenize{ch/people/list:fin-edu-resources-people-carlson}}
\sphinxAtStartPar
Director of Institutional Asset Management at {\hyperref[\detokenize{ch/people/list:fin-edu-resources-firms-ritholtz}]{\sphinxcrossref{\DUrole{std,std-ref}{Ritholtz Wealth Management}}}}. Author of the webiste \sphinxhref{https://awealthofcommonsense.com}{A Wealth of Common Sense}


\subsection{Green, Micheal}
\label{\detokenize{ch/people/list:green-micheal}}\label{\detokenize{ch/people/list:fin-edu-resources-people-green}}
\sphinxAtStartPar
Forseen \sphinxstylestrong{“Volmageddon”} of the 5 February 2018, the collapse of invesrse ETFs or ETP: VIX, SVXY and VMIN, as a consequence of the VIC daily surge from 17 to 37 (approximately +115\%, that erased daily inverse products)

\sphinxAtStartPar
\sphinxstyleemphasis{Even a moderate (\textasciitilde{}4\%) equity sell\sphinxhyphen{}off can unleash devastating volatility swings if the short‑vol market is crowded. Rebalancing demands can worsen volatility when liquidity dries up near market close.}
\begin{itemize}
\item {} 
\sphinxAtStartPar
{\hyperref[\detokenize{ch/people/the_bull_guests:fin-edu-resources-the-bull}]{\sphinxcrossref{\DUrole{std,std-ref}{The Bull}}}}, 192. Micheal Green: perché l’investimento passivo distorce il mercato (e come comportarci)

\end{itemize}


\subsection{Yardeni, Edward}
\label{\detokenize{ch/people/list:yardeni-edward}}\label{\detokenize{ch/people/list:fin-edu-resources-people-yardeni}}

\section{Firms}
\label{\detokenize{ch/people/list:firms}}\label{\detokenize{ch/people/list:fin-edu-resources-firms}}

\subsection{Reserach Affiliates}
\label{\detokenize{ch/people/list:reserach-affiliates}}\label{\detokenize{ch/people/list:fin-edu-resources-firms-research-affiliates}}
\sphinxAtStartPar
Founded by {\hyperref[\detokenize{ch/people/list:fin-edu-resources-people-arnott}]{\sphinxcrossref{\DUrole{std,std-ref}{Robert Arnott}}}}


\subsection{Cambria Investment Management}
\label{\detokenize{ch/people/list:cambria-investment-management}}\label{\detokenize{ch/people/list:fin-edu-resources-firms-cambria}}
\sphinxAtStartPar
Co\sphinxhyphen{}founded by {\hyperref[\detokenize{ch/people/list:fin-edu-resources-people-faber}]{\sphinxcrossref{\DUrole{std,std-ref}{Mebane Faber}}}}


\subsection{Ritholtz Wealth Management}
\label{\detokenize{ch/people/list:ritholtz-wealth-management}}\label{\detokenize{ch/people/list:fin-edu-resources-firms-ritholtz}}
\sphinxAtStartPar
{\hyperref[\detokenize{ch/people/list:fin-edu-resources-people-ritholtz}]{\sphinxcrossref{\DUrole{std,std-ref}{Barry Ritholtz}}}}, {\hyperref[\detokenize{ch/people/list:fin-edu-resources-people-maggiulli}]{\sphinxcrossref{\DUrole{std,std-ref}{Nick Maggiulli}}}}, {\hyperref[\detokenize{ch/people/list:fin-edu-resources-people-carlson}]{\sphinxcrossref{\DUrole{std,std-ref}{Ben Carlson}}}}

\sphinxstepscope


\chapter{The Bull}
\label{\detokenize{ch/people/the_bull_guests:the-bull}}\label{\detokenize{ch/people/the_bull_guests:fin-edu-resources-the-bull}}\label{\detokenize{ch/people/the_bull_guests::doc}}
\sphinxAtStartPar
“The Bull, il tuo podcast di finanza personale”, Riccardo Spada. \sphinxhref{https://www.youtube.com/playlist?list=PL9PLR7E2lKYqup-d4\_7i0XLTXR3arOWAx}{Youtube channel}
\begin{itemize}
\item {} 
\sphinxAtStartPar
Jason Zweig, American financial journalist, columnist for the WSJ since 2008. \sphinxhref{https://jasonzweig.com/}{Jason Zweig \sphinxhyphen{} A Safef Haven for Intelligent Investor}
\begin{itemize}
\item {} 
\sphinxAtStartPar
\sphinxstylestrong{224} what it takes to become an intelligent investor

\end{itemize}

\item {} 
\sphinxAtStartPar
Nick Maggiulli, COO for Ritholtz Wealth Management LCC, financial blogger at \sphinxhref{https://ofdollarsanddata.com/}{Of Dollars And Data \sphinxhyphen{} Act Smarter. Live Richer}. Author of “Just Keep Buying” about the power of compounding.
\begin{itemize}
\item {} 
\sphinxAtStartPar
\sphinxstylestrong{221} Just keep buying

\end{itemize}

\item {} 
\sphinxAtStartPar
Davide Serra, founder and CEO of Algebris Investment
\begin{itemize}
\item {} 
\sphinxAtStartPar
\sphinxstylestrong{216} The change we’re living and consequences for investors

\end{itemize}

\item {} 
\sphinxAtStartPar
William Bernstein, american financial theorist and neurologist
\begin{itemize}
\item {} 
\sphinxAtStartPar
\sphinxstylestrong{214} The 4 pillars of investing and how to manage risk and uncertainty

\end{itemize}

\item {} 
\sphinxAtStartPar
Barry Ritholtz
\begin{itemize}
\item {} 
\sphinxAtStartPar
\sphinxstylestrong{206} How \sphinxstylestrong{not} to invest

\end{itemize}

\item {} 
\sphinxAtStartPar
Robin Wigglesworth
\begin{itemize}
\item {} 
\sphinxAtStartPar
\sphinxstylestrong{203} the case for passive investing: navigative crises through simplicity

\end{itemize}

\item {} 
\sphinxAtStartPar
Ben Carlson, Director of Institutional Asset Management at {\hyperref[\detokenize{ch/people/list:fin-edu-resources-firms-ritholtz}]{\sphinxcrossref{\DUrole{std,std-ref}{Ritholtz Wealth Management}}}}. Author of the webiste \sphinxhref{https://awealthofcommonsense.com}{A Wealth of Common Sense}
\begin{itemize}
\item {} 
\sphinxAtStartPar
\sphinxstylestrong{200} Investing is mostly a matter of common sense (and patience)

\end{itemize}

\item {} 
\sphinxAtStartPar
Robert Arnott
\begin{itemize}
\item {} 
\sphinxAtStartPar
\sphinxstylestrong{196} Fundamental investing

\end{itemize}

\item {} 
\sphinxAtStartPar
Micheal Green
\begin{itemize}
\item {} 
\sphinxAtStartPar
\sphinxstylestrong{192} Why passive investing is distorting the markets

\item {} 
\sphinxAtStartPar
\sphinxhref{https://sites.google.com/site/valentinhaddadresearch/}{Haddad, Valentin}, associate professor at the UCLA: how competitive is stock market? why is asset demand inelastic?

\end{itemize}

\item {} 
\sphinxAtStartPar
Edward Yardeni,
\begin{itemize}
\item {} 
\sphinxAtStartPar
\sphinxstylestrong{183} Why should we be optimistic?

\end{itemize}

\item {} 
\sphinxAtStartPar
Meb Faber,
\begin{itemize}
\item {} 
\sphinxAtStartPar
\sphinxstylestrong{180} Investing with Common Sense

\end{itemize}

\item {} 
\sphinxAtStartPar
Eugene Fama, efficient market, \sphinxstylestrong{factor} investing, and factor ETFs
\begin{itemize}
\item {} 
\sphinxAtStartPar
\sphinxstylestrong{164}

\end{itemize}

\end{itemize}






\renewcommand{\indexname}{Proof Index}
\begin{sphinxtheindex}
\let\bigletter\sphinxstyleindexlettergroup
\bigletter{example\sphinxhyphen{}0}
\item\relax\sphinxstyleindexentry{example\sphinxhyphen{}0}\sphinxstyleindexextra{ch/principles/intro\_nb}\sphinxstyleindexpageref{ch/principles/intro_nb:\detokenize{example-0}}
\indexspace
\bigletter{example\sphinxhyphen{}1}
\item\relax\sphinxstyleindexentry{example\sphinxhyphen{}1}\sphinxstyleindexextra{ch/principles/intro\_nb}\sphinxstyleindexpageref{ch/principles/intro_nb:\detokenize{example-1}}
\indexspace
\bigletter{example\sphinxhyphen{}2}
\item\relax\sphinxstyleindexentry{example\sphinxhyphen{}2}\sphinxstyleindexextra{ch/principles/intro\_nb}\sphinxstyleindexpageref{ch/principles/intro_nb:\detokenize{example-2}}
\indexspace
\bigletter{example\sphinxhyphen{}3}
\item\relax\sphinxstyleindexentry{example\sphinxhyphen{}3}\sphinxstyleindexextra{ch/principles/intro\_nb}\sphinxstyleindexpageref{ch/principles/intro_nb:\detokenize{example-3}}
\end{sphinxtheindex}

\renewcommand{\indexname}{Index}
\printindex
\end{document}