%% Generated by Sphinx.
\def\sphinxdocclass{jupyterBook}
\documentclass[letterpaper,10pt,english]{jupyterBook}
\ifdefined\pdfpxdimen
   \let\sphinxpxdimen\pdfpxdimen\else\newdimen\sphinxpxdimen
\fi \sphinxpxdimen=.75bp\relax
\ifdefined\pdfimageresolution
    \pdfimageresolution= \numexpr \dimexpr1in\relax/\sphinxpxdimen\relax
\fi
%% let collapsible pdf bookmarks panel have high depth per default
\PassOptionsToPackage{bookmarksdepth=5}{hyperref}
%% turn off hyperref patch of \index as sphinx.xdy xindy module takes care of
%% suitable \hyperpage mark-up, working around hyperref-xindy incompatibility
\PassOptionsToPackage{hyperindex=false}{hyperref}
%% memoir class requires extra handling
\makeatletter\@ifclassloaded{memoir}
{\ifdefined\memhyperindexfalse\memhyperindexfalse\fi}{}\makeatother

\PassOptionsToPackage{warn}{textcomp}

\catcode`^^^^00a0\active\protected\def^^^^00a0{\leavevmode\nobreak\ }
\usepackage{cmap}
\usepackage{fontspec}
\defaultfontfeatures[\rmfamily,\sffamily,\ttfamily]{}
\usepackage{amsmath,amssymb,amstext}
\usepackage{polyglossia}
\setmainlanguage{english}



\setmainfont{FreeSerif}[
  Extension      = .otf,
  UprightFont    = *,
  ItalicFont     = *Italic,
  BoldFont       = *Bold,
  BoldItalicFont = *BoldItalic
]
\setsansfont{FreeSans}[
  Extension      = .otf,
  UprightFont    = *,
  ItalicFont     = *Oblique,
  BoldFont       = *Bold,
  BoldItalicFont = *BoldOblique,
]
\setmonofont{FreeMono}[
  Extension      = .otf,
  UprightFont    = *,
  ItalicFont     = *Oblique,
  BoldFont       = *Bold,
  BoldItalicFont = *BoldOblique,
]



\usepackage[Bjarne]{fncychap}
\usepackage[,numfigreset=1,mathnumfig]{sphinx}

\fvset{fontsize=\small}
\usepackage{geometry}


% Include hyperref last.
\usepackage{hyperref}
% Fix anchor placement for figures with captions.
\usepackage{hypcap}% it must be loaded after hyperref.
% Set up styles of URL: it should be placed after hyperref.
\urlstyle{same}

\addto\captionsenglish{\renewcommand{\contentsname}{Introduction}}

\usepackage{sphinxmessages}



        % Start of preamble defined in sphinx-jupyterbook-latex %
         \usepackage[Latin,Greek]{ucharclasses}
        \usepackage{unicode-math}
        % fixing title of the toc
        \addto\captionsenglish{\renewcommand{\contentsname}{Contents}}
        \hypersetup{
            pdfencoding=auto,
            psdextra
        }
        % End of preamble defined in sphinx-jupyterbook-latex %
        

\title{Financial Education - basics}
\date{Dec 10, 2025}
\release{}
\author{basics}
\newcommand{\sphinxlogo}{\vbox{}}
\renewcommand{\releasename}{}
\makeindex
\begin{document}

\pagestyle{empty}
\sphinxmaketitle
\pagestyle{plain}
\sphinxtableofcontents
\pagestyle{normal}
\phantomsection\label{\detokenize{intro::doc}}


\sphinxAtStartPar
This book is meant to collect some notes about financial instruments and methods for financial education, and mainly focused asset allocation.

\sphinxAtStartPar
This material is part of the \sphinxhref{https://basics2022.github.io/bbooks}{\sphinxstylestrong{basics\sphinxhyphen{}books project}}. It is also available as a \DUrole{xref,myst}{.pdf document}.

\begin{sphinxadmonition}{note}{Main goal}

\sphinxAtStartPar
The ultimate goal of this material is to develop an understanding of how to manage personal savings efficiently, in line with one’s own reasonable objectives.

\sphinxAtStartPar
To achieve this, some intermediate goals include:
\begin{itemize}
\item {} 
\sphinxAtStartPar
gaining knowledge of the \sphinxstylestrong{macroeconomic environment}

\item {} 
\sphinxAtStartPar
familiarizing with some of the most common \sphinxstylestrong{investment tools} (mainly bonds and stocks);

\item {} 
\sphinxAtStartPar
getting used to some \sphinxstylestrong{common\sphinxhyphen{}sense} and \sphinxstylestrong{investing principles}: minimizing certain costs when conditions are equal, risk/reward, diversification, liquidity, and other constraints/inefficiencies

\item {} 
\sphinxAtStartPar
learning \sphinxstylestrong{what not to do}

\item {} 
\sphinxAtStartPar
and once the poor choices have been ruled out, evaluating the reasonable options for building and managing an investment portfolio, using mainly {\hyperref[\detokenize{ch/assets/etfs:fin-edu-assets-etfs}]{\sphinxcrossref{\DUrole{std,std-ref}{\sphinxstylestrong{ETF}s}}}} as a natural choice of a (usually) liquid asset providing diversification at low cost, even for small capitals.

\end{itemize}
\end{sphinxadmonition}
\begin{itemize}
\item {} 
\sphinxAtStartPar
Introduction

\begin{itemize}
\item {} 
\sphinxAtStartPar
{\hyperref[\detokenize{ch/summary::doc}]{\sphinxcrossref{Summary}}}

\item {} 
\sphinxAtStartPar
{\hyperref[\detokenize{ch/references::doc}]{\sphinxcrossref{References}}}

\end{itemize}
\end{itemize}
\begin{itemize}
\item {} 
\sphinxAtStartPar
Macroeconomic Context for Investing

\begin{itemize}
\item {} 
\sphinxAtStartPar
{\hyperref[\detokenize{ch/actors::doc}]{\sphinxcrossref{Actors}}}

\item {} 
\sphinxAtStartPar
{\hyperref[\detokenize{code/notebooks/inflation::doc}]{\sphinxcrossref{Inflation}}}

\item {} 
\sphinxAtStartPar
{\hyperref[\detokenize{ch/characteristic-times::doc}]{\sphinxcrossref{Characteristic times in economy}}}

\item {} 
\sphinxAtStartPar
{\hyperref[\detokenize{ch/policy::doc}]{\sphinxcrossref{Policy}}}

\end{itemize}
\end{itemize}
\begin{itemize}
\item {} 
\sphinxAtStartPar
Investing Principles

\begin{itemize}
\item {} 
\sphinxAtStartPar
{\hyperref[\detokenize{ch/principles/intro_nb::doc}]{\sphinxcrossref{Introduction to principles of investing}}}

\end{itemize}
\end{itemize}
\begin{itemize}
\item {} 
\sphinxAtStartPar
Asset classes

\begin{itemize}
\item {} 
\sphinxAtStartPar
{\hyperref[\detokenize{ch/assets/intro::doc}]{\sphinxcrossref{Introduction to asset classes}}}

\item {} 
\sphinxAtStartPar
{\hyperref[\detokenize{ch/assets/bonds::doc}]{\sphinxcrossref{Bonds}}}

\item {} 
\sphinxAtStartPar
{\hyperref[\detokenize{ch/assets/equity::doc}]{\sphinxcrossref{Equity}}}

\item {} 
\sphinxAtStartPar
{\hyperref[\detokenize{ch/assets/etfs::doc}]{\sphinxcrossref{ETFs}}}

\end{itemize}
\end{itemize}

\sphinxstepscope


\part{Introduction}

\sphinxstepscope


\chapter{Summary}
\label{\detokenize{ch/summary:summary}}\label{\detokenize{ch/summary:fin-edu-summary}}\label{\detokenize{ch/summary::doc}}\subsubsection*{Introduction}

\sphinxAtStartPar
Financial goals; money; inflation (BC and inflation target);
\subsubsection*{Asset classes}
\subsubsection*{Asset allocation}

\sphinxstepscope


\chapter{References}
\label{\detokenize{ch/references:references}}\label{\detokenize{ch/references::doc}}
\sphinxAtStartPar
Here some references to othere sources, in order to reasonably organize the contents of this book
\subsubsection*{Investment and Portfolio Management \sphinxhyphen{} RICE \sphinxhyphen{} coursera \sphinxhyphen{} A.Ozoguz, J.Foote}
\subsubsection*{Global Financial Markets}
\begin{itemize}
\item {} 
\sphinxAtStartPar
Intro and Review of Elementary Finance Tools

\item {} 
\sphinxAtStartPar
Financial system and financial assets: fixed income, equity and derivatives

\item {} 
\sphinxAtStartPar
Organization of financial markets and securities trading

\end{itemize}
\subsubsection*{Portfolio Selection and Risk Management}
\begin{itemize}
\item {} 
\sphinxAtStartPar
Intro and R/R: R/R trade\sphinxhyphen{}off

\item {} 
\sphinxAtStartPar
Ptf construction and diversification

\item {} 
\sphinxAtStartPar
Investor choices: utility functions, mean\sphinxhyphen{}variance preferences

\item {} 
\sphinxAtStartPar
Optimal capital allocation and portfolio choice: mean\sphinxhyphen{}variance optimization (Modern Portfolio Theory)

\item {} 
\sphinxAtStartPar
Equilibrium asset princing models: CAPM, return\sphinxhyphen{}beta; multi\sphinxhyphen{}factor models (e.g. Fama\sphinxhyphen{}French)

\end{itemize}
\subsubsection*{Biases and Portfolio Selection}
\begin{itemize}
\item {} 
\sphinxAtStartPar
Efficient Market Hypotesis (EMH), and anomalies

\item {} 
\sphinxAtStartPar
Biases and realistic preferences

\item {} 
\sphinxAtStartPar
Inefficient markets: equity premium, volatility puzzle (?), long\sphinxhyphen{}run reversal to the mean, value effect, momentum

\item {} 
\sphinxAtStartPar
Investor behavior

\end{itemize}
\subsubsection*{Investment Strategies and Portfolio Analysis}
\begin{itemize}
\item {} 
\sphinxAtStartPar
Performance measurement and benchmarking

\item {} 
\sphinxAtStartPar
Active vs passive investing: \(R^*\) risk\sphinxhyphen{}adjusted return measurements: Sharpe, Sortino, Treynor’ratio, Jensens’alpha,…;comparing rhe \(R^*\)

\item {} 
\sphinxAtStartPar
Performance evaluation: style analysis and performance attribution

\end{itemize}
\subsubsection*{Capstone: Build a Winning Investment Portfolio}

\sphinxAtStartPar
Using software for building ptf and assess its properties
\begin{itemize}
\item {} 
\sphinxAtStartPar
…

\end{itemize}

\sphinxstepscope


\part{Macroeconomic Context for Investing}

\sphinxstepscope


\chapter{Actors}
\label{\detokenize{ch/actors:actors}}\label{\detokenize{ch/actors:fin-edu-actors}}\label{\detokenize{ch/actors::doc}}

\section{People}
\label{\detokenize{ch/actors:people}}\label{\detokenize{ch/actors:fin-edu-actors-people}}

\section{Private companies}
\label{\detokenize{ch/actors:private-companies}}\label{\detokenize{ch/actors:fin-edu-actors-firms}}

\section{Government \sphinxhyphen{} public}
\label{\detokenize{ch/actors:government-public}}\label{\detokenize{ch/actors:fin-edu-actors-government}}

\section{Banks}
\label{\detokenize{ch/actors:banks}}\label{\detokenize{ch/actors:fin-edu-actors-banks}}

\subsection{Central banks}
\label{\detokenize{ch/actors:central-banks}}\label{\detokenize{ch/actors:fin-edu-actors-banks-cb}}

\subsection{Investment banks}
\label{\detokenize{ch/actors:investment-banks}}\label{\detokenize{ch/actors:fin-edu-actors-banks-inv}}

\section{Foreign regions}
\label{\detokenize{ch/actors:foreign-regions}}\label{\detokenize{ch/actors:fin-edu-actors-banks-foreign}}
\sphinxstepscope


\chapter{Inflation}
\label{\detokenize{code/notebooks/inflation:inflation}}\label{\detokenize{code/notebooks/inflation:fin-edu-inflation}}\label{\detokenize{code/notebooks/inflation::doc}}
\sphinxAtStartPar
Inflation is the \sphinxstylestrong{rate} at which the general level of prices for goods and sevices changes.

\sphinxAtStartPar
\sphinxstylestrong{Contents.} Definition and {\hyperref[\detokenize{code/notebooks/inflation:fin-edu-inflation-indices}]{\sphinxcrossref{\DUrole{std,std-ref}{inflation indices}}}}, with examples of indices used in Italy: NIC, FOI, IPCA; {\hyperref[\detokenize{code/notebooks/inflation:fin-edu-inflation-ipca}]{\sphinxcrossref{\DUrole{std,std-ref}{components of inflation}}}}, with details of IPCA in Italy; {\hyperref[\detokenize{code/notebooks/inflation:fin-edu-inflation-correlations}]{\sphinxcrossref{\DUrole{std,std-ref}{correlation with other macroeconomic quantities}}}}; {\hyperref[\detokenize{code/notebooks/inflation:fin-edu-inflation-control}]{\sphinxcrossref{\DUrole{std,std-ref}{who controls inflation}}}}; {\hyperref[\detokenize{code/notebooks/inflation:fin-edu-inflation-origin}]{\sphinxcrossref{\DUrole{std,std-ref}{origin of inflation}}}}

\begin{sphinxuseclass}{cell}
\begin{sphinxuseclass}{tag_hide-input}
\end{sphinxuseclass}
\end{sphinxuseclass}
\begin{sphinxuseclass}{cell}
\begin{sphinxuseclass}{tag_hide-input}
\end{sphinxuseclass}
\end{sphinxuseclass}
\begin{sphinxuseclass}{cell}
\begin{sphinxuseclass}{tag_hide-input}
\end{sphinxuseclass}
\end{sphinxuseclass}

\section{Inflation Indices (e.g. in Italy)}
\label{\detokenize{code/notebooks/inflation:inflation-indices-e-g-in-italy}}\label{\detokenize{code/notebooks/inflation:fin-edu-inflation-indices}}
\sphinxAtStartPar
Overall inflation is the the weighted average of inflation on different classes of goods and services, weighted for their share of expenses.

\sphinxAtStartPar
Everyone perceives its own inflation, depending on its expenses. Different indices are usually used within an economy to track inflation for some “average individual”.

\sphinxAtStartPar
Different indices may differ on values of weights, and other “details” like the effect of discounts and public transfers.

\sphinxAtStartPar
As an example, three indices are used in Italy:
\begin{itemize}
\item {} 
\sphinxAtStartPar
\sphinxstylestrong{NIC} (Prezzi al Consumo per l’intera Collettività Nazionale), usually the general

\item {} 
\sphinxAtStartPar
\sphinxstylestrong{FOI} (Prezzi al Consumo per Famiglie di Operai e Impiegati), usually used for contracts, pension and inflation\sphinxhyphen{}linked contracts, ex\sphinxhyphen{}tobacco and lotteries.

\item {} 
\sphinxAtStartPar
\sphinxstylestrong{IPCA} (Indice Armonizzato dei Prezzi al Consumo, HIPC \sphinxstyleemphasis{Harmonized Index of Consumer Prices}), used for comparison and statistics in the EU

\end{itemize}

\begin{sphinxuseclass}{cell}
\begin{sphinxuseclass}{tag_hide-input}
\end{sphinxuseclass}
\end{sphinxuseclass}

\section{Weights and Price Indices of Classes of Goods and Services \sphinxhyphen{} Italy IPCA}
\label{\detokenize{code/notebooks/inflation:weights-and-price-indices-of-classes-of-goods-and-services-italy-ipca}}\label{\detokenize{code/notebooks/inflation:fin-edu-inflation-ipca}}
\sphinxAtStartPar
National and International Institutions for Statistics (in Italy, ISTAT) provide open\sphinxhyphen{}access databases collecting statistics about society and economics, including data about price.

\sphinxAtStartPar
\sphinxstylestrong{ISTAT.} As an example, Italian ISTAT provides data at \sphinxurl{https://esploradati.istat.it/databrowser/\#/it/dw}

\sphinxAtStartPar
All the data we need here is available under the category “Prezzi” \sphinxhyphen{} \sphinxstyleemphasis{Prices}. In order to reach a reasonable stability of the notebook, data have been downloaded, cleaned and stored in a folder on the repository of the project.


\subsection{Inspect Data}
\label{\detokenize{code/notebooks/inflation:inspect-data}}\label{\detokenize{code/notebooks/inflation:fin-edu-inflation-ipca-inspect-data}}
\sphinxAtStartPar
Before producing plots, price indices and weights of level\sphinxhyphen{}4 categories are visually inspected. Data are usually collected in tables.


\subsubsection{Category Price Indices \sphinxhyphen{} Level\sphinxhyphen{}4 IPCA}
\label{\detokenize{code/notebooks/inflation:category-price-indices-level-4-ipca}}\label{\detokenize{code/notebooks/inflation:fin-edu-inflation-ipca-inspect-data-prices}}
\begin{sphinxuseclass}{cell}
\begin{sphinxuseclass}{tag_hide-output}
\begin{sphinxuseclass}{tag_hide-input}
\end{sphinxuseclass}
\end{sphinxuseclass}
\end{sphinxuseclass}

\subsubsection{Category Weights \sphinxhyphen{} Level\sphinxhyphen{}4 IPCA}
\label{\detokenize{code/notebooks/inflation:category-weights-level-4-ipca}}\label{\detokenize{code/notebooks/inflation:fin-edu-inflation-ipca-inspect-data-weights}}
\begin{sphinxuseclass}{cell}
\begin{sphinxuseclass}{tag_hide-input}
\begin{sphinxuseclass}{tag_hide-output}
\end{sphinxuseclass}
\end{sphinxuseclass}
\end{sphinxuseclass}
\begin{sphinxuseclass}{cell}
\begin{sphinxuseclass}{tag_hide-input}
\end{sphinxuseclass}
\end{sphinxuseclass}

\subsection{Plots}
\label{\detokenize{code/notebooks/inflation:plots}}

\subsubsection{Category weights \sphinxhyphen{} Level\sphinxhyphen{}2 IPCA}
\label{\detokenize{code/notebooks/inflation:category-weights-level-2-ipca}}
\sphinxAtStartPar
The weights assigned to IPCA (Harmonized Index of Consumer Prices) categories represent the average expenditure share of households on each category of goods and services. These weights reflect how important each category is in the consumption basket.

\sphinxAtStartPar
These weights are revised annually to account for changing consumer behavior, as one can easily realize acting on the slider of the picture below. They are the weights used in computing the overall inflation \(i\) index, as the weighted sum of inflation \(i_c\) of IPCA categories,
\begin{equation*}
\begin{split}i = \sum_{c \in \text{Cat}} i_c \, w_c \ .\end{split}
\end{equation*}
\begin{sphinxuseclass}{cell}
\begin{sphinxuseclass}{tag_hide-input}\begin{sphinxVerbatimOutput}

\begin{sphinxuseclass}{cell_output}
\end{sphinxuseclass}\end{sphinxVerbatimOutput}

\end{sphinxuseclass}
\end{sphinxuseclass}

\subsubsection{Category Prices \sphinxhyphen{} Level\sphinxhyphen{}2 IPCA}
\label{\detokenize{code/notebooks/inflation:category-prices-level-2-ipca}}
\sphinxAtStartPar
Some categories in IPCA are subject to strong seasonal effects, meaning prices follow recurring patterns during the year.

\sphinxAtStartPar
As an example:
\begin{itemize}
\item {} 
\sphinxAtStartPar
Clothing and Footwear: in July–August, retailers apply seasonal discounts (saldi) in Italy and prices in IPCA do include these discounts when they are actually applied in stores, as it’s shown by seasonal July/August price drops

\item {} 
\sphinxAtStartPar
Fresh fruits and vegetables: prone to seasonal availability, leading to fluctuating prices.

\item {} 
\sphinxAtStartPar
Travel and tourism: prices rise in summer and holidays.

\end{itemize}

\sphinxAtStartPar
Seasonality can obscure underlying inflation trends: that’s why \sphinxstylestrong{seasonally adjusted} inflation is evaluated, see below.

\begin{sphinxuseclass}{cell}
\begin{sphinxuseclass}{tag_hide-input}\begin{sphinxVerbatimOutput}

\begin{sphinxuseclass}{cell_output}
\begin{sphinxVerbatim}[commandchars=\\\{\}]
Index([\PYGZsq{}[00] Indice generale\PYGZsq{},
       \PYGZsq{}[01] \PYGZhy{}\PYGZhy{} prodotti alimentari e bevande analcoliche\PYGZsq{},
       \PYGZsq{}[02] \PYGZhy{}\PYGZhy{} bevande alcoliche e tabacchi\PYGZsq{},
       \PYGZsq{}[03] \PYGZhy{}\PYGZhy{} abbigliamento e calzature\PYGZsq{},
       \PYGZsq{}[04] \PYGZhy{}\PYGZhy{} abitazione, acqua, elettricità, gas e altri combustibili\PYGZsq{},
       \PYGZsq{}[05] \PYGZhy{}\PYGZhy{} mobili, articoli e servizi per la casa\PYGZsq{},
       \PYGZsq{}[06] \PYGZhy{}\PYGZhy{} servizi sanitari e spese per la salute\PYGZsq{}, \PYGZsq{}[07] \PYGZhy{}\PYGZhy{} trasporti\PYGZsq{},
       \PYGZsq{}[08] \PYGZhy{}\PYGZhy{} comunicazioni\PYGZsq{}, \PYGZsq{}[09] \PYGZhy{}\PYGZhy{} ricreazione, spettacoli e cultura\PYGZsq{},
       \PYGZsq{}[10] \PYGZhy{}\PYGZhy{} istruzione\PYGZsq{}, \PYGZsq{}[11] \PYGZhy{}\PYGZhy{} servizi ricettivi e di ristorazione\PYGZsq{},
       \PYGZsq{}[12] \PYGZhy{}\PYGZhy{} altri beni e servizi\PYGZsq{}],
      dtype=\PYGZsq{}object\PYGZsq{}, name=\PYGZsq{}Tempo\PYGZsq{})
\end{sphinxVerbatim}

\end{sphinxuseclass}\end{sphinxVerbatimOutput}

\end{sphinxuseclass}
\end{sphinxuseclass}

\subsubsection{Category Price Changes (Inflation) \sphinxhyphen{}  Level\sphinxhyphen{}2 IPCA}
\label{\detokenize{code/notebooks/inflation:category-price-changes-inflation-level-2-ipca}}
\sphinxAtStartPar
The 12\sphinxhyphen{}month inflation rate (year\sphinxhyphen{}on\sphinxhyphen{}year or YoY) compares prices in a given month to the same month the year before. It’s already less prone to seasonal effects than the month\sphinxhyphen{}to\sphinxhyphen{}month rate.

\sphinxAtStartPar
However, even YoY rates can exhibit seasonal patterns, especially in volatile components like food, energy, and clothing. In order to reduce volatility of inflation indices, it’s possible to use:
\begin{itemize}
\item {} 
\sphinxAtStartPar
\sphinxstylestrong{Core inflation}, as a measure of inflation that excludes the most volatile items (e.g., unprocessed food, energy), in order to provide a smoothed measure of inflation trends.

\item {} 
\sphinxAtStartPar
Statistical filtering, and moving averages

\end{itemize}



\begin{sphinxadmonition}{note}{Energy post\sphinxhyphen{}2022}

\sphinxAtStartPar
Since 2022, prices in the energy and utility sectors have shown exceptional volatility. Different causes may have contributed, like geopolitical tensions (notably, the war in Ukraine), “liberalized” electricity/gas markets in Italy where price caps were adjusted or removed. Inflation in energy and electricity was also influenced by a \sphinxstyleemphasis{base effect} (e.g., very low prices in 2020–2021 followed by spikes in 2022).

\sphinxAtStartPar
Policy interventions like tax reductions and bonuses \sphinxhyphen{} that are not “free” \sphinxhyphen{}, which may or may not be reflected in consumer prices, depending on implementation.

\sphinxAtStartPar
The use of \sphinxstyleemphasis{core inflation} in 2022–2023 was arguable, as energy prices didn’t just spiked and reverted, but was/is quite a long\sphinxhyphen{}term shock (war, sanctions, market and supply restructuring,…); as energy price influences many other sectors, food price rose as well, due to input cost shocks /fretilizers, transports,…) not as a result of seasonality only. Using core inflation and excluding energy and food components masked the true \sphinxstylestrong{cost\sphinxhyphen{}of living} impact on households.
\end{sphinxadmonition}

\begin{sphinxuseclass}{cell}
\begin{sphinxuseclass}{tag_hide-input}\begin{sphinxVerbatimOutput}

\begin{sphinxuseclass}{cell_output}
\end{sphinxuseclass}\end{sphinxVerbatimOutput}

\end{sphinxuseclass}
\end{sphinxuseclass}

\subsubsection{Category contributions to overall inflation \sphinxhyphen{} Level\sphinxhyphen{}2 IPCA}
\label{\detokenize{code/notebooks/inflation:category-contributions-to-overall-inflation-level-2-ipca}}
\begin{sphinxuseclass}{cell}
\begin{sphinxuseclass}{tag_hide-input}
\end{sphinxuseclass}
\end{sphinxuseclass}
\begin{sphinxuseclass}{cell}
\begin{sphinxuseclass}{tag_hide-input}
\begin{sphinxuseclass}{tag_hide-output}
\end{sphinxuseclass}
\end{sphinxuseclass}
\end{sphinxuseclass}
\begin{sphinxuseclass}{cell}
\begin{sphinxuseclass}{tag_hide-input}\begin{sphinxVerbatimOutput}

\begin{sphinxuseclass}{cell_output}
\end{sphinxuseclass}\end{sphinxVerbatimOutput}

\end{sphinxuseclass}
\end{sphinxuseclass}

\section{Correlations in macroeconomics with inflation}
\label{\detokenize{code/notebooks/inflation:correlations-in-macroeconomics-with-inflation}}\label{\detokenize{code/notebooks/inflation:fin-edu-inflation-correlations}}
\sphinxAtStartPar
Some correlations exist%
\begin{footnote}[1]\sphinxAtStartFootnote
…
%
\end{footnote} between inflation and other macroeconocmics quantitites.
\begin{itemize}
\item {} 
\sphinxAtStartPar
\sphinxstylestrong{Phillips Curve}: inverse relation between inflation and unemployment (in the short\sphinxhyphen{}run)

\item {} 
\sphinxAtStartPar
\sphinxstylestrong{Money supply} in the long\sphinxhyphen{}run “\sphinxstyleemphasis{Inflation is a monetary phenomenon}”, M.Friedman.

\end{itemize}


\section{Control of Inflation}
\label{\detokenize{code/notebooks/inflation:control-of-inflation}}\label{\detokenize{code/notebooks/inflation:fin-edu-inflation-control}}
\sphinxAtStartPar
Control of inflation is one of the goals of \sphinxstylestrong{central banks}, like the FED and the ECB.

\sphinxAtStartPar
{\hyperref[\detokenize{ch/actors:fin-edu-actors-banks-cb}]{\sphinxcrossref{\DUrole{std,std-ref}{Central banks}}}} aims at controlling inflation, matching target inflation (usaully set as 2\%) by means of \sphinxstylestrong{monetary policy}:
\begin{itemize}
\item {} 
\sphinxAtStartPar
interest rates (cost of money)

\item {} 
\sphinxAtStartPar
non\sphinxhyphen{}conventional actions, like quantitative easing (QE)/tightening (QT)

\end{itemize}

\sphinxAtStartPar
A goverment may indirectly influence inflation with \sphinxstylestrong{fiscal policy}, as taxation and government spending can influence demand.

\sphinxAtStartPar
\sphinxstylestrong{Credibility} of targets, and actors through their actions and forward guidance may influence inflation as well: expectations influences inflation.


\section{Origin of inflation}
\label{\detokenize{code/notebooks/inflation:origin-of-inflation}}\label{\detokenize{code/notebooks/inflation:fin-edu-inflation-origin}}
\sphinxAtStartPar
Origin of inflation?
\begin{itemize}
\item {} 
\sphinxAtStartPar
{\hyperref[\detokenize{ch/characteristic-times:fin-edu-characteristic-times-short}]{\sphinxcrossref{\DUrole{std,std-ref}{short\sphinxhyphen{}run}}}}, {\hyperref[\detokenize{ch/characteristic-times:fin-edu-characteristic-times-medium}]{\sphinxcrossref{\DUrole{std,std-ref}{medium\sphinxhyphen{}run}}}}: cost\sphinxhyphen{}push, demand\sphinxhyphen{}pull, built\sphinxhyphen{}in (triangle model)

\item {} 
\sphinxAtStartPar
{\hyperref[\detokenize{ch/characteristic-times:fin-edu-characteristic-times-long}]{\sphinxcrossref{\DUrole{std,std-ref}{long\sphinxhyphen{}run}}}}: “inflation is always and everywhere a monetary phenomenon” M.Friedman

\end{itemize}


\bigskip\hrule\bigskip


\sphinxstepscope


\chapter{Characteristic times in economy}
\label{\detokenize{ch/characteristic-times:characteristic-times-in-economy}}\label{\detokenize{ch/characteristic-times:fin-edu-characteristic-times}}\label{\detokenize{ch/characteristic-times::doc}}

\section{The short run}
\label{\detokenize{ch/characteristic-times:the-short-run}}\label{\detokenize{ch/characteristic-times:fin-edu-characteristic-times-short}}

\section{The medium run}
\label{\detokenize{ch/characteristic-times:the-medium-run}}\label{\detokenize{ch/characteristic-times:fin-edu-characteristic-times-medium}}

\section{The long run}
\label{\detokenize{ch/characteristic-times:the-long-run}}\label{\detokenize{ch/characteristic-times:fin-edu-characteristic-times-long}}
\sphinxstepscope


\chapter{Policy}
\label{\detokenize{ch/policy:policy}}\label{\detokenize{ch/policy:fin-edu-policy}}\label{\detokenize{ch/policy::doc}}

\begin{savenotes}\sphinxattablestart
\centering
\begin{tabulary}{\linewidth}[t]{|T|T|T|}
\hline

\sphinxAtStartPar

&\sphinxstyletheadfamily 
\sphinxAtStartPar
Monetary Policy
&\sphinxstyletheadfamily 
\sphinxAtStartPar
Fiscal Policy
\\
\hline
\sphinxAtStartPar
Controlled by
&
\sphinxAtStartPar
CB
&
\sphinxAtStartPar
Government
\\
\hline
\sphinxAtStartPar
Main tools
&
\sphinxAtStartPar
IR, Money supply
&
\sphinxAtStartPar
Taxes, Spending, Transfers
\\
\hline
\sphinxAtStartPar
Speed
&
\sphinxAtStartPar
Usually faster
&
\sphinxAtStartPar
Politically slower, debated
\\
\hline
\sphinxAtStartPar
Focus
&
\sphinxAtStartPar
Inflation, lliquidity, credit
&
\sphinxAtStartPar
Employment, Income distribution
\\
\hline
\sphinxAtStartPar
Independence
&
\sphinxAtStartPar

&
\sphinxAtStartPar

\\
\hline
\end{tabulary}
\par
\sphinxattableend\end{savenotes}


\section{Monetary policy}
\label{\detokenize{ch/policy:monetary-policy}}\label{\detokenize{ch/policy:fin-edu-policy-monetary}}

\section{Fiscal policy}
\label{\detokenize{ch/policy:fiscal-policy}}\label{\detokenize{ch/policy:fin-edu-policy-fiscal}}
\sphinxstepscope


\part{Investing Principles}

\sphinxstepscope


\chapter{Introduction to principles of investing}
\label{\detokenize{ch/principles/intro_nb:introduction-to-principles-of-investing}}\label{\detokenize{ch/principles/intro_nb:fin-edu-principles-intro-nb}}\label{\detokenize{ch/principles/intro_nb::doc}}
\sphinxAtStartPar
Investing is a core part of personal financial management—it’s how individuals navigate uncertainty to meet their financial goals under real\sphinxhyphen{}world constraints. The most basic objective is to preserve the real value of wealth, protecting it against {\hyperref[\detokenize{code/notebooks/inflation:fin-edu-inflation}]{\sphinxcrossref{\DUrole{std,std-ref}{inflation}}}}; more ambitious goals include growing capital to fund retirement, education, or other life plans.

\sphinxAtStartPar
Sound investing requires understanding {\hyperref[\detokenize{ch/principles/intro_nb:fin-edu-principles-return}]{\sphinxcrossref{\DUrole{std,std-ref}{return}}}} and {\hyperref[\detokenize{ch/principles/intro_nb:fin-edu-principles-risk}]{\sphinxcrossref{\DUrole{std,std-ref}{risk}}}} of available assets, and the fundamental {\hyperref[\detokenize{ch/principles/intro_nb:fin-edu-principles-rr}]{\sphinxcrossref{\DUrole{std,std-ref}{R/R trade off}}}}. It also demands attention to \sphinxstylestrong{constraints} such as \sphinxstyleemphasis{liquidity} needs, \sphinxstyleemphasis{time horizon}, \sphinxstyleemphasis{acceptable volatility}, and \sphinxstyleemphasis{risk tolerance}. One of the main principle is {\hyperref[\detokenize{ch/principles/intro_nb:fin-edu-principles-diversification}]{\sphinxcrossref{\DUrole{std,std-ref}{diversification}}}} \sphinxhyphen{} which can reduce risk and, in some cases, enhance returns.

\sphinxAtStartPar
This section introduces the core concepts needed to build a robust investment strategy: how {\hyperref[\detokenize{ch/principles/intro_nb:fin-edu-principles-time-compunding}]{\sphinxcrossref{\DUrole{std,std-ref}{compound returns}}}} shape long\sphinxhyphen{}term growth, how {\hyperref[\detokenize{ch/principles/intro_nb:fin-edu-principles-time-volatility-drag}]{\sphinxcrossref{\DUrole{std,std-ref}{volatility drag}}}} reduces expected performance, and how a clear, principle\sphinxhyphen{}based approaches \sphinxhyphen{} like {\hyperref[\detokenize{ch/principles/intro_nb:fin-edu-principles-rebalancing}]{\sphinxcrossref{\DUrole{std,std-ref}{rebalancing}}}} \sphinxhyphen{} may improve performance under uncertainties.

\sphinxAtStartPar
Given its set of constraints, an informed and intelligent agent, see {\hyperref[\detokenize{ch/principles/intro_nb:fin-edu-principles-asset-allocation}]{\sphinxcrossref{\DUrole{std,std-ref}{Portfolio construction}}}} would take actions that try to maximise return for a given accepted risk, or minimize risk for a given desired return: this behavior can be summarized in choosing actions on a \sphinxstyleemphasis{Pareto front}, i.e. within the set of all Pareto efficient solutions.
\subsubsection*{Sections}


\begin{savenotes}\sphinxattablestart
\centering
\begin{tabulary}{\linewidth}[t]{|T|T|}
\hline
\sphinxstyletheadfamily 
\sphinxAtStartPar
\sphinxstylestrong{Section}
&\sphinxstyletheadfamily 
\sphinxAtStartPar
\sphinxstylestrong{Key Concepts}
\\
\hline
\sphinxAtStartPar
\sphinxstylestrong{1. {\hyperref[\detokenize{ch/principles/intro_nb:fin-edu-principles-return}]{\sphinxcrossref{\DUrole{std,std-ref}{Return}}}}}
&
\sphinxAtStartPar

\\
\hline
\sphinxAtStartPar
\sphinxstylestrong{2. {\hyperref[\detokenize{ch/principles/intro_nb:fin-edu-principles-risk}]{\sphinxcrossref{\DUrole{std,std-ref}{Risk}}}}}
&
\sphinxAtStartPar

\\
\hline
\sphinxAtStartPar
\sphinxstylestrong{3. {\hyperref[\detokenize{ch/principles/intro_nb:fin-edu-principles-rr}]{\sphinxcrossref{\DUrole{std,std-ref}{Risk\sphinxhyphen{}Return Trade\sphinxhyphen{}Off}}}}}
&
\sphinxAtStartPar

\\
\hline
\sphinxAtStartPar
\sphinxstylestrong{4. {\hyperref[\detokenize{ch/principles/intro_nb:fin-edu-principles-diversification}]{\sphinxcrossref{\DUrole{std,std-ref}{Diversification}}}}}
&
\sphinxAtStartPar

\\
\hline
\sphinxAtStartPar
\sphinxstylestrong{5. {\hyperref[\detokenize{ch/principles/intro_nb:fin-edu-principles-asset-allocation}]{\sphinxcrossref{\DUrole{std,std-ref}{Portfolio Construction}}}}}
&
\sphinxAtStartPar

\\
\hline
\sphinxAtStartPar
\sphinxstylestrong{6. {\hyperref[\detokenize{ch/principles/intro_nb:fin-edu-principles-time}]{\sphinxcrossref{\DUrole{std,std-ref}{Time and Compounding}}}}}
&
\sphinxAtStartPar
Compounding and volatility drag
\\
\hline
\sphinxAtStartPar
\sphinxstylestrong{7. {\hyperref[\detokenize{ch/principles/intro_nb:fin-edu-principles-investing}]{\sphinxcrossref{\DUrole{std,std-ref}{Disciplined Investing}}}}}
&
\sphinxAtStartPar
PIC/PAC, rebalancing,…
\\
\hline
\end{tabulary}
\par
\sphinxattableend\end{savenotes}


\section{Return}
\label{\detokenize{ch/principles/intro_nb:return}}\label{\detokenize{ch/principles/intro_nb:fin-edu-principles-return}}
\sphinxAtStartPar
Return is the reward for investing. It can come from \sphinxstylestrong{capital gain} (price increase of assets bought), or \sphinxstylestrong{periodic cashflows}, like interest (from bonds), or dividends (from stocks). Some assets produce predictable return (either nominal, or real), other assets have less predictable returns. Any asset has some level of uncertainty, or {\hyperref[\detokenize{ch/principles/intro_nb:fin-edu-principles-rr}]{\sphinxcrossref{\DUrole{std,std-ref}{risk}}}}%
\begin{footnote}[1]\sphinxAtStartFootnote
Even the most safe assets could undergo some (really) \sphinxstylestrong{rare}, but usually (really) \sphinxstylestrong{catastrophic events}. Just as an example, it’s hard to imagine what could happen even to bonds issued by the most (perceived and priced) safe government or institution, in case of its participation in a war.
%
\end{footnote}.

\sphinxAtStartPar
Most returns are quoted on a \sphinxstylestrong{per\sphinxhyphen{}period} basis \sphinxhyphen{} usually annually \sphinxhyphen{} and expressed as the percentage of the reward over the initial amount of the investment.



\sphinxAtStartPar
For a many\sphinxhyphen{}year investment, single\sphinxhyphen{}period returns {\hyperref[\detokenize{ch/principles/intro_nb:fin-edu-principles-time-compunding}]{\sphinxcrossref{\DUrole{std,std-ref}{\sphinxstylestrong{compound}}}}} over time.


\subsection{Costs}
\label{\detokenize{ch/principles/intro_nb:costs}}\label{\detokenize{ch/principles/intro_nb:fin-edu-principles-return-costs}}
\sphinxAtStartPar
While return are uncertain, at least to a certain level, usually costs \sphinxhyphen{} fees, expenses, taxes \sphinxhyphen{} or part of them, are certain. With equal other conditions, the intelligent investor should reduce costs (known), as higher costs reduce returns w/o changing the level of risk.


\section{Risk}
\label{\detokenize{ch/principles/intro_nb:risk}}\label{\detokenize{ch/principles/intro_nb:fin-edu-principles-risk}}
\sphinxAtStartPar
Risk measures uncertainty and its effects, combining probability of events and consequences of specific events. \sphinxstyleemphasis{All the assets have some systematic and some specific risks}
.



\sphinxAtStartPar
Key measures (\sphinxstyleemphasis{should give info about magnitude, frequency/probability, and duration}) include:
\begin{itemize}
\item {} 
\sphinxAtStartPar
standard deviation or \sphinxstylestrong{volatility}: how much returns may deviate from their expected value),

\item {} 
\sphinxAtStartPar
max loss (usually 100\% can’t be neglected for catastrophic although rare events), value at risk (VaR, max loss with a given probability), drawdown (maximum peak\sphinxhyphen{}to\sphinxhyphen{}trough loss)

\item {} 
\sphinxAtStartPar
time\sphinxhyphen{}to\sphinxhyphen{}recover (time to recover drawdowns, in a temporal perspective)

\end{itemize}

\sphinxAtStartPar
Usually, risk metrics measure uncertainty, without discerning from positive and negative events: these metrics perceive a higher\sphinxhyphen{}than\sphinxhyphen{}expected return as a risk as well. Some metrics instead, see \sphinxstyleemphasis{Sortino ratio} in {\hyperref[\detokenize{ch/principles/intro_nb:fin-edu-principles-rr}]{\sphinxcrossref{\DUrole{std,std-ref}{risk\sphinxhyphen{}return}}}} section, aims at quantifying only negative events as risk.




\section{Risk\sphinxhyphen{}Return Trade Off}
\label{\detokenize{ch/principles/intro_nb:risk-return-trade-off}}\label{\detokenize{ch/principles/intro_nb:fin-edu-principles-rr}}
\begin{sphinxadmonition}{note}{“There’s no free lunch”}

\sphinxAtStartPar
Higher expected returns usually come with higher risk.
\end{sphinxadmonition}

\begin{sphinxadmonition}{note}{…but high risk doesn’t imply high expected return}

\sphinxAtStartPar
Very stupid actions usually implies poor return with high risk. Just as an example, playing Russian roulette for fun implies an expected return worse than an alternative “do\sphinxhyphen{}nothing and have an ice\sphinxhyphen{}cream instead” scenario (at least, if your goal is not to kill yourself, and your return function does not positively weight this outcome) with higher uncertainty on the final status of your health.

\sphinxAtStartPar
Sometimes the same could happen if one plays doing trading with some random meme\sphinxhyphen{}stocks or shit\sphinxhyphen{}coins.
\end{sphinxadmonition}

\sphinxAtStartPar
\sphinxstylestrong{Risk\sphinxhyphen{}adjusted return} provides an indication of the expected return per unit of risk. Common metrics are:
\begin{itemize}
\item {} 
\sphinxAtStartPar
\sphinxstylestrong{Sharpe ratio}, comparing excess return and volatility compared with a “risk\sphinxhyphen{}free” asset \sphinxhyphen{} or a benchmark
\begin{equation*}
\begin{split}S := \dfrac{\mathbb{E}[R-R_b]}{\sqrt{\text{var}[R-R_b]}}\end{split}
\end{equation*}
\item {} 
\sphinxAtStartPar
\sphinxstylestrong{Sortino ratio}
\begin{equation*}
\begin{split}So := \dfrac{\mathbb{E}[R] - T}{\text{DR}} \ ,\end{split}
\end{equation*}
\sphinxAtStartPar
with \(T\) target return, and \(\text{DR}\) the downside deviation, i.e. the deviation w.r.t the target return evaluated only for returns \(r\) lower than the target return \(T\)
\begin{equation*}
\begin{split}\text{DR}^2 = \int_{r=-\infty}^{T} (T-r)^2 \, f(r) \, dr \ ,\end{split}
\end{equation*}
\sphinxAtStartPar
being \(f(r)\) the probability density function of the (continuous) random variable \(R\) representing return

\end{itemize}




\section{Diversification}
\label{\detokenize{ch/principles/intro_nb:diversification}}\label{\detokenize{ch/principles/intro_nb:fin-edu-principles-diversification}}
\sphinxAtStartPar
Diversification spreads risk across different investments so no single event can ruin your portfolio. Diversification works well with assets that are not \sphinxhyphen{} or at least they’re loosely \sphinxhyphen{} correlated: in this case, diversification could increase return per unit of risk.


\section{Portfolio Construction}
\label{\detokenize{ch/principles/intro_nb:portfolio-construction}}\label{\detokenize{ch/principles/intro_nb:fin-edu-principles-asset-allocation}}

\section{Time}
\label{\detokenize{ch/principles/intro_nb:time}}\label{\detokenize{ch/principles/intro_nb:fin-edu-principles-time}}

\subsection{Compound Return}
\label{\detokenize{ch/principles/intro_nb:compound-return}}\label{\detokenize{ch/principles/intro_nb:fin-edu-principles-time-compunding}}
\begin{sphinxuseclass}{cell}
\begin{sphinxuseclass}{tag_hide-input}\begin{sphinxVerbatimOutput}

\begin{sphinxuseclass}{cell_output}
\begin{sphinxVerbatim}[commandchars=\\\{\}]
(Text(0.5, 0, \PYGZsq{}t\PYGZsq{}), None)
\end{sphinxVerbatim}

\noindent\sphinxincludegraphics{{762bf809b74094ddaaa2537222a9e4fe8c7a634837a512931af6778e3b19d5c5}.png}

\end{sphinxuseclass}\end{sphinxVerbatimOutput}

\end{sphinxuseclass}
\end{sphinxuseclass}

\subsubsection{Volatility Drag}
\label{\detokenize{ch/principles/intro_nb:volatility-drag}}\label{\detokenize{ch/principles/intro_nb:fin-edu-principles-time-volatility-drag}}
\begin{sphinxuseclass}{cell}
\begin{sphinxuseclass}{tag_hide-input}\begin{sphinxVerbatimOutput}

\begin{sphinxuseclass}{cell_output}
\noindent\sphinxincludegraphics{{f37eacbc45ab67db5d7c9d52f442a3c6e0639ea2a14430eb481ad90844a084cb}.png}

\end{sphinxuseclass}\end{sphinxVerbatimOutput}

\end{sphinxuseclass}
\end{sphinxuseclass}
\sphinxAtStartPar
\sphinxstylestrong{todo}
\begin{itemize}
\item {} 
\sphinxAtStartPar
\sphinxstyleemphasis{“Time and risk?” Listen to The Logic of Risk}

\end{itemize}


\section{Disciplined Investing}
\label{\detokenize{ch/principles/intro_nb:disciplined-investing}}\label{\detokenize{ch/principles/intro_nb:fin-edu-principles-investing}}

\subsection{Rebalancing}
\label{\detokenize{ch/principles/intro_nb:rebalancing}}\label{\detokenize{ch/principles/intro_nb:fin-edu-principles-rebalancing}}
\sphinxAtStartPar
\sphinxhref{https://colab.research.google.com/drive/1Mi3\_9T7XN7xUl9XNfsdkMfqQTqRFzG8L?authuser=1\#scrollTo=QUq8nMHq3bb5}{Colab Notebook, rebalancing.ipynb}


\subsubsection{Rebalancing premium}
\label{\detokenize{ch/principles/intro_nb:rebalancing-premium}}\label{\detokenize{ch/principles/intro_nb:fin-edu-principles-rebalancing-premium}}

\bigskip\hrule\bigskip


\sphinxstepscope


\part{Asset classes}

\sphinxstepscope


\chapter{Introduction to asset classes}
\label{\detokenize{ch/assets/intro:introduction-to-asset-classes}}\label{\detokenize{ch/assets/intro:fin-edu-assets-intro}}\label{\detokenize{ch/assets/intro::doc}}
\sphinxstepscope


\chapter{Bonds}
\label{\detokenize{ch/assets/bonds:bonds}}\label{\detokenize{ch/assets/bonds:fin-edu-assets-bonds}}\label{\detokenize{ch/assets/bonds::doc}}
\sphinxAtStartPar
…

\sphinxAtStartPar
Here the most general expression for nominal and real \sphinxstylestrong{yield} are derived as a function of prices, face value of coupon, taxation and year to maturity, both in case of coupon reinvestment or not (reinvestment not always possible); a closed form solution is then derived under some assumptions, like constant (or average) rates; the effect on price and yield of credit rating and rating change, coupon, year to maturity are discussed on both examples and real\sphinxhyphen{}world cases.

\sphinxAtStartPar
Extra:
\begin{itemize}
\item {} 
\sphinxAtStartPar
definition of duration

\item {} 
\sphinxAtStartPar
risks: inflation; reinvesment (at lower rates) for bonds with same maturity and different coupons

\item {} 
\sphinxAtStartPar
inflation linked

\end{itemize}


\section{Constant coupon bonds}
\label{\detokenize{ch/assets/bonds:constant-coupon-bonds}}

\subsection{W/o reinvestment}
\label{\detokenize{ch/assets/bonds:w-o-reinvestment}}
\sphinxAtStartPar
At time \(t_0\) the unit price of a bond is \(p_0\); investing \(Y_0\) allows to buy \(N_0 = \frac{Y_0}{p_0}\) titles; each title has the right of receiving net coupon \(C (1 - t)\), with \(t\) taxation rate, per period (here assumed 1\sphinxhyphen{}year coupon range).
\begin{equation*}
\begin{split}N_0 = \dfrac{Y_0}{p_0} = \dfrac{Y_0}{p_{in}} \dfrac{p_{in}}{p_0}\end{split}
\end{equation*}
\sphinxAtStartPar
W/o reinvestment, the number of titles hold is constant and equal to \(N_0\). As capital \(Y_i\) can be written as the product of unit price and number of bond in portfolio, the DCF of a bond w/o coupon reinvestment reads
\begin{equation*}
\begin{split}\begin{aligned}
  \widetilde{DCF} =
  & = - Y_0 + Y_N \prod_{k=1}^N ( 1 + r_k )^{-1} + \sum_{k=1}^{N} N_0 C (1-t) \prod_{j=1}^{k} (1 + r_j )^{-1} \\ 
  & = N_0 \left[ - p_0 + p_N \prod_{k=1}^N ( 1 + r_k )^{-1} + C (1-t) \sum_{k=1}^{N} \prod_{j=1}^{k} (1 + r_j )^{-1} \right] \ , 
\end{aligned}\end{split}
\end{equation*}
\sphinxAtStartPar
This DCF must be corrected a CF at time \(t_N\) corresponding to tax on capital gain if \(p_n > p_0\), discounted as
\begin{equation*}
\begin{split}- N_0( p_N  - p_0) \,  t \,  \prod_{k=1}^{N} (1+r_k)^{-1}  \qquad  (\text{only if $p_N > p_0$})\end{split}
\end{equation*}
\sphinxAtStartPar
The cumulative real return (if the discount ratio is inflation) is the ratio between the \(DCF\) and the actual value of the investment \(Y_0\),
\begin{equation*}
\begin{split}\widetilde{\dfrac{DCF}{Y_0}} = - 1 + \dfrac{p_N}{p_0} \prod_{k=1}^N ( 1 + r_k )^{-1} + \dfrac{C}{p_0} (1-t) \sum_{k=1}^{N} \prod_{j=1}^{k} (1 + r_j )^{-1}  \end{split}
\end{equation*}
\sphinxAtStartPar
If the discount rate is constant, or the average (which average) discount rate is used, the expression of the cumulative return reads
\begin{equation*}
\begin{split}\begin{aligned}
  \dfrac{\widetilde{DCF}}{Y_0} = - 1 + \dfrac{p_N}{p_0} ( 1 + r )^{-N} + \dfrac{C}{p_0} (1-t) \sum_{k=1}^{N} (1 + r )^{-k}  
\end{aligned}\end{split}
\end{equation*}

\subsection{W/ reinvestment}
\label{\detokenize{ch/assets/bonds:w-reinvestment}}

\begin{savenotes}\sphinxattablestart
\centering
\begin{tabulary}{\linewidth}[t]{|T|T|T|T|T|}
\hline
\sphinxstyletheadfamily 
\sphinxAtStartPar
Time
&\sphinxstyletheadfamily 
\sphinxAtStartPar
Cashflows
&\sphinxstyletheadfamily 
\sphinxAtStartPar
\(\Delta\)Quantity
&\sphinxstyletheadfamily 
\sphinxAtStartPar
Quantity
&\sphinxstyletheadfamily 
\sphinxAtStartPar
DF
\\
\hline
\sphinxAtStartPar
\(0\)
&
\sphinxAtStartPar
\(-Y_0\)
&
\sphinxAtStartPar
\(N_0 = \frac{Y_0}{p_0}\)
&
\sphinxAtStartPar
\(N_0 = \frac{Y_0}{p_0}\)
&
\sphinxAtStartPar
1
\\
\hline
\sphinxAtStartPar
\(1\)
&
\sphinxAtStartPar
\(+N_0 C ( 1-t )\)
&
\sphinxAtStartPar

&
\sphinxAtStartPar

&
\sphinxAtStartPar
\((1+r_1)^{-1}\)
\\
\hline
\sphinxAtStartPar
\(1\)
&
\sphinxAtStartPar
\(-N_0 C ( 1-t )\)
&
\sphinxAtStartPar
\(N_1 = \frac{N_0 C (1-t)}{p_1}\)
&
\sphinxAtStartPar
\(N_{0:1} = N_0+N_1\)
&
\sphinxAtStartPar
\((1+r_1)^{-1}\)
\\
\hline
\sphinxAtStartPar
\(2\)
&
\sphinxAtStartPar
\(+N_{0:1} C ( 1-t )\)
&
\sphinxAtStartPar

&
\sphinxAtStartPar

&
\sphinxAtStartPar
\((1+r_1)^{-1} (1+r_2)^{-1}\)
\\
\hline
\sphinxAtStartPar
\(2\)
&
\sphinxAtStartPar
\(-N_{0:1} C ( 1-t )\)
&
\sphinxAtStartPar
\(N_2 = \frac{N_{0:1} C (1-t)}{p_2}\)
&
\sphinxAtStartPar
\(N_{0:2} = N_0+N_1 + N_2\)
&
\sphinxAtStartPar
\((1+r_1)^{-1} (1+r_2)^{-1}\)
\\
\hline
\sphinxAtStartPar
…
&
\sphinxAtStartPar

&
\sphinxAtStartPar

&
\sphinxAtStartPar

&
\sphinxAtStartPar

\\
\hline
\sphinxAtStartPar
\(T-1\)
&
\sphinxAtStartPar
\(+N_{0:T-2} C ( 1-t )\)
&
\sphinxAtStartPar

&
\sphinxAtStartPar

&
\sphinxAtStartPar
\(\prod_{k=1}^{T-1} (1+r_k)^{-1}\)
\\
\hline
\sphinxAtStartPar
\(T-1\)
&
\sphinxAtStartPar
\(-N_{0:T-2} C ( 1-t )\)
&
\sphinxAtStartPar
\(N_{T-1} = \frac{N_{0:T-2} C (1-t)}{p_{T-1}}\)
&
\sphinxAtStartPar
\(N_{0:T-1} = \sum_{k=0}^{T-1} N_k\)
&
\sphinxAtStartPar
\(\prod_{k=1}^{T-1} (1+r_k)^{-1}\)
\\
\hline
\sphinxAtStartPar
\(T\)
&
\sphinxAtStartPar
\(+N_{0:T-1} C ( 1-t )\)
&
\sphinxAtStartPar

&
\sphinxAtStartPar

&
\sphinxAtStartPar
\(\prod_{k=1}^{T  } (1+r_k)^{-1}\)
\\
\hline
\sphinxAtStartPar
\(T\)
&
\sphinxAtStartPar
\(+N_{0:T-1} p_T\)
&
\sphinxAtStartPar

&
\sphinxAtStartPar

&
\sphinxAtStartPar
\(\prod_{k=1}^{T  } (1+r_k)^{-1}\)
\\
\hline
\end{tabulary}
\par
\sphinxattableend\end{savenotes}

\sphinxAtStartPar
All the cashflows from coupons are immediately reinvested so the DCF is
\begin{equation*}
\begin{split}\begin{aligned}
  DCF 
  & = - Y_0 + \underbrace{N_{0:T-1} \left( p_T + C (1-t)\right)}_{Y_T} \, \underbrace{ \prod_{k=1}^T (1+r_k)^{-1} }_{DF_T} = \\
  & = - Y_0 + Y_T \,  DF_T \ ,
\end{aligned}\end{split}
\end{equation*}
\sphinxAtStartPar
with
\begin{equation*}
\begin{split}\begin{aligned}
  N_{0:T-1}
  & = N_{0:T-2} + N_{T-1} =  N_{0:T-2} + N_{0:T-2} \frac{ C (1-t)}{p_{T-1}} = N_{0:T-2} \left[ 1 + \frac{ C (1-t)}{p_{T-1}} \right] = \\
  & = N_{0:T-3} \left[ 1 + \frac{ C (1-t)}{p_{T-2}} \right] \left[ 1 + \frac{ C (1-t)}{p_{T-1}} \right] = \\
  & = \dots = \\
  & = N_{0:1} \prod_{k=2}^{T-1} \left[ 1 + \frac{ C (1-t)}{p_{k}} \right] = \\ 
  & = N_{0  } \prod_{k=1}^{T-1} \left[ 1 + \frac{ C (1-t)}{p_{k}} \right] \ .
\end{aligned}\end{split}
\end{equation*}
\sphinxAtStartPar
Cumulative discounted return reads
\begin{equation*}
\begin{split}\begin{aligned}
  \dfrac{DCF}{Y_0} 
  & = - 1 + \dfrac{Y_T}{Y_0} DF_{T} = \\
  & = - 1 + \dfrac{N_0}{N_0 \, p_0} \prod_{k=1}^{T-1} \left( 1+ \dfrac{C(1-t)}{p_k} \right) \, ( p_T + C(t-1) ) \, DF_T \\
  & = - 1 + \dfrac{p_T}{p_0} \prod_{k=1}^{T} \left( 1+ \dfrac{C(1-t)}{p_k} \right) \, DF_T \\
  & = - 1 + \dfrac{p_T}{p_0} \prod_{k=1}^{T} \left( \dfrac{ 1+ \frac{C(1-t)}{p_k} }{1+r_k} \right) \ .
\end{aligned}\end{split}
\end{equation*}
\sphinxAtStartPar
Composite discounted return is obtained, after writing the diiscounted cashflow as the difference between discounted cashflow at time \(t_T\) and \(t_0\), \(DCF = Y_T \ DF_T - T_0\),
\begin{equation*}
\begin{split}(1 + DCAGR)^T = \dfrac{Y_T \, DF_T}{Y_0} = \dfrac{DCF}{Y_0} + 1 = \dfrac{p_T}{p_0} \, \prod_{k=1}^{T} \left( \dfrac{ 1+ \frac{C(1-t)}{p_k} }{1+r_k} \right)\end{split}
\end{equation*}\begin{equation*}
\begin{split}DCAGR = \left( \dfrac{p_T}{p_0} \, \prod_{k=1}^{T}  \dfrac{ 1+ \frac{C(1-t)}{p_k} }{1+r_k} \right)^{\frac{1}{T}} - 1\end{split}
\end{equation*}
\sphinxAtStartPar
\sphinxstylestrong{If}%
\begin{footnote}[1]\sphinxAtStartFootnote
It’s a big if. Even if credit rating and inflation are constant throughout the life of the bond, years to maturity decreases and thus \sphinxhyphen{} usually \sphinxhyphen{} the required rate decreases as well.
%
\end{footnote} price of the bond is constant throughout its whole life, \(p_k = 1\), \(\forall k=0:T\), and discount rate \(r\) is constant, the number of held bonds at time \(T-1\) is
\begin{equation*}
\begin{split}N_{0:T-1} = N_0 \left( 1 + C(1-t) \right)^{T-1} \ ,\end{split}
\end{equation*}
\sphinxAtStartPar
the discounted cashflow is
\begin{equation*}
\begin{split}\begin{aligned}
  DCF 
  & = - N_0 + N_0 \left( 1 + C(1-t) \right)^{T-1} ( 1 + C(1-t) ) \left( 1 + r \right)^{-T} = \\
  & = N_0 \left[ - 1 +  \left( \dfrac{ 1 + C(1-t) }{ 1 + r } \right)^{T} \right] \ ,
\end{aligned}\end{split}
\end{equation*}
\sphinxAtStartPar
cumulative discounted return
\begin{equation*}
\begin{split}\dfrac{DCF}{Y_0} = - 1 + \left( \dfrac{ 1 + C(1-t) }{ 1 + r } \right)^{T}\end{split}
\end{equation*}
\sphinxAtStartPar
and the composite discounted return reads
\begin{equation*}
\begin{split}DCAGR = \dfrac{1 + C(1-t)}{1+r} - 1 \ .\end{split}
\end{equation*}

\bigskip\hrule\bigskip


\sphinxstepscope


\chapter{Equity}
\label{\detokenize{ch/assets/equity:equity}}\label{\detokenize{ch/assets/equity:fin-edu-assets-equity}}\label{\detokenize{ch/assets/equity::doc}}
\sphinxAtStartPar
\sphinxstylestrong{What’s equity?}
\subsubsection*{Contents}

\sphinxAtStartPar
\sphinxstylestrong{{\hyperref[\detokenize{ch/assets/equity-valuation:fin-edu-assets-equity-valuation}]{\sphinxcrossref{\DUrole{std,std-ref}{Valuation methods}}}}.} Comparison and intrinsic value methods.

\sphinxAtStartPar
\sphinxstylestrong{\DUrole{xref,myst}{Financial statements}.} Introduction to financial statements of a company.

\sphinxAtStartPar
\sphinxstylestrong{\DUrole{xref,myst}{Correlations, plots and fun\sphinxhyphen{}facts}.}

\sphinxstepscope


\section{Equity Valuation}
\label{\detokenize{ch/assets/equity-valuation:equity-valuation}}\label{\detokenize{ch/assets/equity-valuation:fin-edu-assets-equity-valuation}}\label{\detokenize{ch/assets/equity-valuation::doc}}\subsubsection*{Detailed introduction}

\sphinxAtStartPar
Equity valuation blends common sense, mathematics, expectations, estimation—and a bit of art. Buying shares in a company, whether directly or through a fund, means owning a (tiny) stake in a real business that produces goods and/or services and has the potential to generate earnings or free cash flows. As a shareholder, you are not just investing in market prices—you’re becoming a part\sphinxhyphen{}owner of the enterprise. This ownership entitles you to a share of the company’s profits through dividends or capital appreciation. It also comes with certain rights and responsibilities, especially during difficult periods.

\sphinxAtStartPar
When companies face financial stress or pursue growth opportunities, they may issue new shares to raise capital. This can lead to dilution, reducing the percentage ownership of existing shareholders. However, shareholders often have preemptive rights, allowing them to participate in new issuances to maintain their ownership stake. Moreover, owning equity means having a claim on the residual value of the company—what’s left after all debts are paid—in both prosperous and challenging times. Understanding these dynamics is crucial to valuing equity: you’re not just buying into today’s performance, but into a stream of future cash flows and the complex, evolving structure of ownership.

\sphinxAtStartPar
\sphinxstylestrong{Sensitivity analysis} could provide an estimate of the effects of different parameters/assumptions on the final result.

\sphinxAtStartPar
\sphinxstylestrong{Different valuation methods} exist, and can be broadly classified in
\begin{itemize}
\item {} 
\sphinxAtStartPar
\sphinxstylestrong{comparison} approach: P/E, EV/EBITDA, or other indices used to compare companies of the same sector, marked, dimension,…%
\begin{footnote}[1]\sphinxAtStartFootnote
It’s not always possible to find “equivalent” companies for the comparison…; P/E, EV/EBITDA,… whould be projected into the future to keep into account future in the value of a firm.
%
\end{footnote}

\item {} 
\sphinxAtStartPar
\sphinxstylestrong{intrinsic value} approach, based on \sphinxstylestrong{DCF}

\item {} 
\sphinxAtStartPar
…other methods for general firms (cost approach,…); valuation of financials;…

\end{itemize}


\subsection{Comparison}
\label{\detokenize{ch/assets/equity-valuation:comparison}}\label{\detokenize{ch/assets/equity-valuation:fin-edu-assets-equity-valuation-comparison}}

\subsection{Intrinsic value}
\label{\detokenize{ch/assets/equity-valuation:intrinsic-value}}\label{\detokenize{ch/assets/equity-valuation:fin-edu-assets-equity-valuation-intrinsic}}\begin{itemize}
\item {} 
\sphinxAtStartPar
Future cash flows are estimated,

\item {} 
\sphinxAtStartPar
CFs are discounted, usually for the \(WACC\) (Weighted Average Cost of Capital) to find the \(NPV\) (net present value) of the \sphinxstylestrong{enterprise value} \(EV\)

\item {} 
\sphinxAtStartPar
Cash and equivalents are added to the \(NPV\) to find the \sphinxstylestrong{equity value}

\end{itemize}

\begin{sphinxadmonition}{note}{\protect\(WACC\protect\)}
\begin{equation*}
\begin{split}WACC = \dfrac{E}{V} R_e + \dfrac{D}{V} R_d (1 - t)\end{split}
\end{equation*}
\sphinxAtStartPar
being \(R_e\) the \sphinxstylestrong{cost of equity} and \(R_d\) the \sphinxstylestrong{cost of debt} (maybe the easiest part to estimated accurately, since the debt structure is usually known/programmed). The factor \((1-t)\) usually appears as interest payments are tax\sphinxhyphen{}deductible.
\end{sphinxadmonition}

\begin{sphinxadmonition}{note}{Equity Risk Premium \protect\(R_e\protect\) \sphinxhyphen{} Sharpe}

\sphinxAtStartPar
Following W.Sharpe, equity risk premium can be estimated as
\begin{equation*}
\begin{split}R_e = R_f + (R_m + R_f) \beta \ ,\end{split}
\end{equation*}
\sphinxAtStartPar
being \(R_f\) the risk\sphinxhyphen{}free rate (usaully 10Y US Treasuries), and \(R_m\) the annual return of the market/sector of the investment, \(\beta\) is a measure of risk or stock volatility of returns of the investment relative to that of the market/sector.
\end{sphinxadmonition}


\bigskip\hrule\bigskip


\sphinxstepscope


\chapter{ETFs}
\label{\detokenize{ch/assets/etfs:etfs}}\label{\detokenize{ch/assets/etfs:fin-edu-assets-etfs}}\label{\detokenize{ch/assets/etfs::doc}}






\renewcommand{\indexname}{Index}
\printindex
\end{document}